\documentclass[12pt]{article}
\usepackage{amsfonts}
\usepackage[hscale=0.8,vscale=0.8]{geometry}
\usepackage{hyperref}

\begin{document}

\title{Notes on HJM Framework}
\date{Dec. 32, 2999}

\maketitle

This mainly derives from Andersen and Piterbarg's {\it Interest Rate Modeling}, and Brigo and Mercurio's
{\it Interest Rate Models - Theory and Practice}.

\section{Formulation}

  Let $P(t,T)$ denotes the time $t$ price of a zero coupon bond which pays 1 at time $T$, {\it i.e.},
  \begin{equation}
    P(t,T)=E_t^Q\left[\exp\left(-\int_t^T r(u)du\right)\right],
  \end{equation}
  where $Q$ stands for the risk neutral measure, and $r(t)$ is the short rate. Then, under the same measure,
  \begin{equation}
    \label{bond1}
    \frac{dP(t,T)}{P(t,T)}=r(t)dt-\sigma_P^{\prime}(t,T)dW(t),
  \end{equation}
  where $W$ is a $d$-dimensional Brownian motion, $\sigma_P(t,T)$ is a
  $d$-dimensional stochastic process adapted to the filtration generated by $W$,
  and the prime represents matrix transposition. Here, $\sigma_P(t,T)$ must
  satisfy the consistency condition, $\sigma_P(T,T)=0$, since $P(T,T)=1$.
  
  Define the forward discount bond price as
  \begin{equation}
    P(t,T,T+\tau)=\frac{P(t,T+\tau)}{P(t,T)},
  \end{equation}
  and apply the Ito's lemma, then
  \begin{equation}
    \label{bond2}
    \frac{dP(t,T,T+\tau)}{P(t,T,T+\tau)}=-\left[\sigma_P^{\prime}(t,T+\tau)-\sigma_P^{\prime}(t,T)\right]\sigma_P(t,T)dt
                                        -\left[\sigma_P^{\prime}(t,T+\tau)-\sigma_P^{\prime}(t,T)\right]dW(t),
  \end{equation}
  under the risk neutral measure. Under the $T$-forward measure, the forward discount bond price is a martingale,
  which means
  \begin{equation}
    \label{bond3}
    \frac{dP(t,T,T+\tau)}{P(t,T,T+\tau)}=-\left[\sigma_P^{\prime}(t,T+\tau)-\sigma_P^{\prime}(t,T)\right]dW^T(t).
  \end{equation}
  
  Define the forward rate as
  \begin{equation}
    f(t,T)=-\frac{\partial\log P(t,T)}{\partial T},
  \end{equation}
  then from Eq. (\ref{bond1}), the forward rate has the following dynamics,
  \begin{equation}
    \label{rate}
    df(t,T)=\sigma_f^{\prime}(t,T)\sigma_P(t,T)dt+\sigma_f^{\prime}(t,T)dW(t),
  \end{equation}
  where $\sigma_f(t,T)=\partial\sigma_P(t,T)/\partial T$, or equivalently,
  $\sigma_P(t,T)=\int_t^T\sigma_f(t,u)du$.


\section{Markovian Short Rate Dynamics}

  From Eq. (\ref{rate}), the forward rate is given by
  \begin{equation}
    f(t,T)=f(0,T)+\int_0^t\sigma_f^{\prime}(u,T)\left(\int_u^T\sigma_f(u,s)ds\right)du+\int_0^t\sigma_f^{\prime}(u,T)dW(u).
  \end{equation}
  Define the short rate as $r(t)=f(t,t)$, then the short rate dynamics becomes
  \begin{equation}
    r(t)=f(0,t)+\int_0^t\sigma_f^{\prime}(u,t)\left(\int_u^t\sigma_f(u,s)ds\right)du+\int_0^t\sigma_f^{\prime}(u,t)dW(u).
  \end{equation}
  Generally, the short rate process is not Markovian. However, after imposing the separation condition
  $\sigma_f(t,T)=g(t)h(T)$, where $g(t)\equiv g(t;\omega)$ is a $d\times d$ matrix valued function, which can
  be deterministic or stochastic, and $h$ is a
  deterministic $d$-dimensional vector, the short rate will be Markovian. Under this condition,
  \begin{equation}
    r(t)=f(0,t)+h^{\prime}(t)\int_0^tg^{\prime}(u)g(u)\left(\int_u^th(s)ds\right)du+h^{\prime}(t)\int_0^tg^{\prime}(u)dW(u).
  \end{equation}
  
  To cast the short rate dynamics into a convenient form, define
  \begin{equation}
    H(t)={\rm diag}(h_1(t),...,h_d(t))=\left(
      \begin{array}{ccc}
        h_1(t) & & 0 \\
          & \ddots & \\
          0 & & h_d(t)
      \end{array}
    \right),
  \end{equation}
  and
  \begin{equation}
    \kappa(t) = -\frac{dH(t)}{dt}H^{-1}(t).
  \end{equation}
  Here, we have assumed that none of the $h_i$ is zero. Later it will be shown that the diagonal elements of $\kappa(t)$
  are the mean reversion parameter for the short rate dynamics. Now, define
  \begin{equation}
    x(t)=H(t)\int_0^tg^{\prime}(u)g(u)\left(\int_u^th(s)ds\right)du+H(t)\int_0^tg^{\prime}(u)dW(u),
  \end{equation}
  and
  \begin{equation}
    y(t)=H(t)\left(\int_0^tg^{\prime}(u)g(u)du\right)H(t),
  \end{equation}
  then it can be shown
  \begin{equation}
    \frac{dy}{dt} = H(t)g^{\prime}(t)g(t)H(t) - \kappa(t)y(t) - y(t)\kappa(t),
  \end{equation}
  which is deterministic and can be solved numerically and analytically, if possible. Also,
  \begin{eqnarray}
    dx(t) &=& \left[\frac{dH(t)}{dt}\int_0^tg^{\prime}(u)g(u)\left(\int_u^th(s)ds\right)du\right]dt
            + \left[H(t)\left(\int_0^tg^{\prime}(u)g(u)du\right)h(t)dt\right]\nonumber\\
          &&+ \frac{dH(t)}{dt}\left(\int_0^tg^{\prime}(u)dW(u)\right)dt
            + H(t)g^{\prime}(t)dW(t)\nonumber\\
          &=& (y(t)I_{d\times 1}-\kappa(t)x(t))dt+H(t)g^{\prime}(t)dW(t)\nonumber\\
          &=& (y(t)I_{d\times 1}-\kappa(t)x(t))dt+\sigma_x^{\prime}(t)dW(t),
  \end{eqnarray}
  where $I_{d\times 1}=(1,1,...,1)^{\prime}$, and $\sigma_x(t)=g(t)H(t)$. The $x$ and $y$ processes have
  initial conditions $x(0)=0$ and $y(0)=0$, which is evident from their definitions. It can be shown,
  by Ritchken and Sankarasubramanian, that any interest rate derivative will be completely determined
  by the two-state Markovian process $(x(t), y(t))$.
  
  Using the above parameterizations, it can be shown, for forward rate,
  \begin{eqnarray}
    f(t,T) &=& f(0,T)+I_{d\times 1}^{\prime}H(T)\int_0^tg^{\prime}(u)g(u)\left(\int_u^Th(s)ds\right)du
              +I_{d\times 1}^{\prime}H(T)\int_0^tg^{\prime}(u)dW(u)\nonumber\\
           &=& f(0,T)+I_{d\times 1}^{\prime}H(T)\int_0^tg^{\prime}(u)g(u)\left(\int_u^th(s)ds\right)du
              +I_{d\times 1}^{\prime}H(T)\int_0^tg^{\prime}(u)dW(u)\nonumber\\
           && + I_{d\times 1}^{\prime}H(T)\int_0^tg^{\prime}(u)g(u)\left(\int_t^{\prime}h(s)ds\right)du\nonumber\\
           &=& f(0,T)+I_{d\times 1}^{\prime}H(T)H^{-1}(t)\Big(H(t)\int_0^tg^{\prime}(u)g(u)\left(\int_u^th(s)ds\right)du\nonumber\\
           &&\ \ \ \ \ \ \ \ \ \ \ \ \ \ \ \ \ \ \ \ \ \ \ \ \ \ \ \ \ \ \ \ \ \ \ \ \ \ \ +H(t)\int_0^tg^{\prime}(u)dW(u)\Big)\nonumber\\
           && + I_{d\times 1}^{\prime}H(T)H^{-1}(t)H(t)\int_0^tg^{\prime}(u)g(u)\left(\int_t^Th(s)ds\right)du\nonumber\\
           &=& f(0,T)+M^{\prime}(t,T)\left(x(t)+y(t)\int_t^TM(t,u)du\right),
  \end{eqnarray}
  where
  \begin{equation}
    M(t,T)=H(T)H^{-1}(t)I_{d\times 1}.
  \end{equation}
  From this, the short rate is given by
  \begin{equation}
    r(t)=f(t,t)=f(0,t)+I_{d\times 1}^{\prime}x(t)=f(0,t)+\sum_{i=1}^dx_i(t).
  \end{equation}
  
  
  From the above result, we can also derive the bond reconstitution formula. Notice $P(t,T)=\exp\left(-\int_t^Tf(t,u)du\right)$, we have
  \begin{eqnarray}
    P(t,T)&=&\exp\Bigg(-\int_t^Tf(0,u)du-\left(\int_t^TM^{\prime}(t,u)du\right)x(t) \nonumber\\
          &&\ \ \ \ \ \ \ \ \ \ \ \ \ \ \ \ \ \ \ -\int_t^TM^{\prime}(t,u)y(t)\left(\int_t^uM(t,s)ds\right)du\Bigg).
  \end{eqnarray}
  Define
  \begin{equation}
    G(t,T)=\int_t^TM(t,u)du,
  \end{equation}
  and notice that 
  \begin{equation}
    \int_t^TM^{\prime}(t,u)y(t)\left(\int_t^uM(t,s)ds\right)du = \int_t^TM^{\prime}(t,u)y(t)\left(\int_u^TM(t,s)ds\right)du,
  \end{equation}
  the bond reconstitution formula can be rewritten as
  \begin{equation}
    P(t,T)=\frac{P(0,T)}{P(0,t)}\exp\left(-G^{\prime}(t,T)x(t)-\frac{1}{2}G^{\prime}(t,T)y(t)G(t,T)\right).
    \label{reconstitution}
  \end{equation}


\section{Closed Form Results for Gaussian Models}

  In the above formulation, when $\sigma_f(t,T)$ is a deterministic function,
  the resulting models will be Gaussian, where the distribution of the forward rate, and the short rate, will be Gaussian.
  This leads to simplifications to the modeling and pricing of interest rate derivatives, and even closed form expressions
  for several simple derivatives.
    
  \subsection{Bond option}
  
    Consider a call option with expiry $T$ and strike $K$ on a zero coupon bond with maturity $T^*$. At expiry, the
    option pays
    \begin{equation}
      V(T)=\left(P(T,T^*)-K\right)^+.
    \end{equation}
    Using the change of measure trick, the value of the zero coupon bond becomes
    \begin{equation}
      V(t)=E_t^Q\left[\exp\left(-\int_t^Tr(s)ds\right)V(T)\right]=P(t,T)E_t^T\left[\left(P(T,T,T^*)-K\right)^+\right],
    \end{equation}
    under the $T$-forward measure. Then, apply the Black-Scholes formula and take into account of Eq. (\ref{bond3}),
    \begin{eqnarray}
      V(t) &=& P(t,T)\left[P(t,T,T^*)N(d_+)-KN(d_-)\right] \nonumber\\
           &=& P(t,T^*)N(d_+) - KP(t,T)N(d_-),
    \end{eqnarray}
    where
    \begin{equation}
      d_{\pm} = \frac{\displaystyle\log\left(\frac{P(t,T,T^*)}{K}\right)\pm \frac{v}{2}}{\sqrt{v}}
             = \frac{\displaystyle\log\left(\frac{P(t,T^*)}{KP(t,T)}\right)\pm \frac{v}{2}}{\sqrt{v}},
    \end{equation}
    and
    \begin{equation}
      v=\int_t^T\left|\sigma_P(s,T^*)-\sigma_P(s,T)\right|^2ds.
    \end{equation}
  
  \subsection{Caplet}
  
    Now consider a caplet which resets at time $T$, and pays at $T+\tau$ the amount
    \begin{equation}
      V(T+\tau) = \tau\left(L(T,T+\tau)-K\right)^+,
    \end{equation}
    where $L(T,T+\tau)$ is the LIBOR rates spanning from $T$ to $T+\tau$, and $\tau$
    is the year fraction. The value of the caplet at time $t$ is
    \begin{equation}
      V(t)=E_t^Q\left[\exp\left(\int_t^{T+\tau}r(s)ds\right)V(T+\tau)\right].
    \end{equation}
    Use the tower rule,
    \begin{eqnarray}
      V(t)&=&E_t^Q\left[\exp\left(-\int_t^{T}r(s)ds\right)E_T^Q\left[\exp\left(-\int_T^{T+\tau}r(s)ds\right)V(T+\tau)\right]\right]\nonumber\\
          &=&E_t^Q\left[\exp\left(-\int_t^{T}r(s)ds\right)P(T,T+\tau)V(T+\tau)\right].
    \end{eqnarray}
    Notice that $L(T,T+\tau)=\tau^{-1}(1/P(T,T+\tau)-1)$, the above expression becomes
    \begin{eqnarray}
      V(t)&=&E_t^Q\left[\exp\left(-\int_t^{T}r(s)ds\right)\left(1-\left(1+\tau K\right)P(T,T+\tau)\right)^+\right]\nonumber\\
          &=&(1+\tau K)E_t^Q\left[\exp\left(-\int_t^{T}r(s)ds\right)\left(\frac{1}{1+\tau K}-P(T,T+\tau)\right)^+\right].
    \end{eqnarray}
    Now, the caplet price has been turned into a put option on a zero coupon bond. Using the earlier result, the caplet price is
    \begin{eqnarray}
      V(t)&=&(1+\tau K)P(t,T)\left(\frac{1}{1+\tau K}N(-d_-)-P(T,T+\tau)N(-d_+)\right)\nonumber\\
          &=&P(t,T)N(-d_-)-(1+\tau K)P(t,T+\tau)N(-d_+),
    \end{eqnarray}
    where
    \begin{equation}
      d_{\pm} = \frac{\displaystyle\log\left(\frac{(1+\tau K)P(t,T+\tau)}{P(t,T)}\right)\pm \frac{v}{2}}{\sqrt{v}},
    \end{equation}
    and
    \begin{equation}
      v=\int_t^T\left|\sigma_P(s,T+\tau)-\sigma_P(s,T)\right|^2ds.
    \end{equation}
  
  \subsection{Futures}
  
    The futures rate is a martingale under the risk neutral measure, {\it i.e.}, $F(t,T,T+\tau)=E_t^Q[L(T,T,T+\tau)]$.
    This is equivalent to
    \begin{equation}
      F(t,T,T+\tau)=\frac{1}{\tau}E_t^Q\left[\frac{1}{P(T,T+\tau)}-1\right].
    \end{equation}
    The dynamics of $G(t)=1/P(t,T,T+\tau)=P(t,T)/P(t,T+\tau)$ under the risk neutral measure is
    \begin{eqnarray}
      d\left(\frac{1}{P(t,T,T+\tau)}\right)&=&-\frac{dP(t,T,T+\tau)}{P^2(t,T,T+\tau)}+\frac{\left(dP(t,T,T+\tau)\right)^2}{P^3(t,T,T+\tau)}\nonumber\\
                &=&\frac{1}{P(t,T,T+\tau)}\Big(\left[\sigma_P^{\prime}(t,T+\tau)-\sigma_P^{\prime}(t,T)\right]\sigma_P(t,T)dt\nonumber\\
                && \ \ \ \ \ \ \ \ \ \ \ \ \ \ \  \ \ \ \ +\left[\sigma_P^{\prime}(t,T+\tau)-\sigma_P^{\prime}(t,T)\right]dW(t)\Big)\nonumber\\
                && + \frac{1}{P(t,T,T+\tau)}\left[\sigma_P^{\prime}(t,T+\tau)-\sigma_P^{\prime}(t,T)\right]\left[\sigma_P(t,T+\tau)-\sigma_P(t,T)\right]dt\nonumber\\
                &=&\frac{1}{P(t,T,T+\tau)}\Big(\left[\sigma_P^{\prime}(t,T+\tau)-\sigma_P^{\prime}(t,T)\right]\sigma_P(t,T+\tau)dt\nonumber\\
                && \ \ \ \ \ \ \ \ \ \ \ \ \ \ \ \ \ \ \ \ +\left[\sigma_P^{\prime}(t,T+\tau)-\sigma_P^{\prime}(t,T)\right]dW(t)\Big).
    \end{eqnarray}
    Therefore,
    \begin{equation}
      E_t^Q\left[\frac{1}{P(T,T+\tau)}\right]=\frac{1}{P(t,T,T+\tau)}e^{\Omega(t,T)},
    \end{equation}
    where
    \begin{equation}
      \Omega(t,T)=\int_t^T\left[\sigma_P^{\prime}(u,T+\tau)-\sigma_P^{\prime}(u,T)\right]\sigma_P(u,T+\tau)du.
    \end{equation}
    So, the futures rate is
    \begin{equation}
      F(t,T,T+\tau)=\frac{1}{\tau}\left(\frac{1}{P(t,T,T+\tau)}e^{\Omega(t,T)}-1\right).
    \end{equation}
 
  \subsection{European Swaption - Jamshidian Decomposition}

  The above results do not require the Markovianity of the short rate process, and can be applied to general models with
  deterministic forward rate volatility functions. The trick which will be introduced in the following relies on the
  bond reconstitution formula, Eq. (\ref{reconstitution}).

  Consider a payer swaption expiring at $T_0$, with the underlying swap paying annualized coupon $c$ at times
  $T_1<T_2<\cdots<T_N$, with $T_1>T_0$. The swaption payout at $T_0$ is
  \begin{equation}
    V(T_0) = \left(1-P(T_0,T_N)-c\sum_{i=1}^N\tau_iP(T_0,T_i)\right)^+,
  \end{equation}
  where $\tau_i=T_i-T_{i-1}$ is the accrual period. Notice the dependence of the zero coupon bond price on $x$, {\it i.e.},
  $P(T_0,T_N)\equiv P(T_0,T_N;x(T_0))$, through the bond reconstitution formula, Eq. (\ref{reconstitution}). Define a
  critical value $x^*$ by setting the value of the swaption at time $T_0$ to exactly zero,
  \begin{equation}
    P(T_0,T_N;x^*)+c\sum_{i=1}^N\tau_iP(T_0,T_i;x^*)=1.
  \end{equation}
  Also defint the ``strikes" as
  \begin{equation}
    K_i=P(T_0,T_i;x^*),
  \end{equation}
  then
  \begin{equation}
    K_N+c\sum_{i=1}^N\tau_iK_i=1.
  \end{equation}
  Notice that $P(T_0,T_i;x(T_0))$ is monotonically decreasing in $x(T_0)$, we have
  \begin{eqnarray}
    V(T_0)&=&\left(1-P(T_0,T_N;x(T_0))-c\sum_{i=1}^N\tau_iP(T_0,T_i;x(T_0))\right)^+\nonumber\\
          &=&\left(1-P(T_0,T_N;x(T_0))-c\sum_{i=1}^N\tau_iP(T_0,T_i;x(T_0))\right)\mathcal{1}_{x(T_0)>x^*}\nonumber\\
          &=&\left(K_N+c\sum_{i=1}^N\tau_iK_i-P(T_0,T_N;x(T_0))-c\sum_{i=1}^N\tau_iP(T_0,T_i;x(T_0))\right)\mathcal{1}_{x(T_0)>x^*}\nonumber\\
          &=&\left(K_N-P(T_0,T_N;x(T_0))\right)\mathcal{1}_{x(T_0)>x^*}
                    + c\sum_{i=1}^N\tau_i\left(K_i-P(T_0,T_i;x(T_0))\right)\mathcal{1}_{x(T_0)>x^*}\nonumber\\
          &=&\left(K_N-P(T_0,T_N;x(T_0))\right)^+
       + c\sum_{i=1}^N\tau_i\left(K_i-P(T_0,T_i;x(T_0))\right)^+.
  \end{eqnarray}
  Then, this can be evaluated with the bond option price formula.
   
\section{Examples of Classical Gaussian Short Rate Models - One Factor Models}

  By setting $d=1$ in the above formulation, we will obtain one-factor models. Since the matrix products will be reduced to
  scalar products, this will greatly simplify the discussion.
  
  \subsection{Ho-Lee Model}

  The Ho-Lee model can be recovered by taking $h(t)=1$, and $g(t)=\sigma$ which is constant. It is straightforward to show that
  $\kappa(t)=0$, $y(t)=\sigma^2t$,
  \begin{equation}
    dx(t)=\sigma^2tdt+\sigma dW(t),
  \end{equation}
  and $r(t)=f(0,t)+x(t)$. Also, the forward rate volatility is $\sigma_f(t,T)=\sigma$, which is constant too, $\sigma_P(t,T)=\sigma(T-t)$, and the bond
  reconstitution formula is
  \begin{eqnarray}
    P(t,T)&=&\frac{P(0,T)}{P(0,t)}\exp\left(-x(t)(T-t)-\frac{1}{2}\sigma^2(T-t)^2t\right)\nonumber\\
          &=&\frac{P(0,T)}{P(0,t)}\exp\left(-\left(r(t)-f(0,t)\right)(T-t)-\frac{1}{2}\sigma^2(T-t)^2t\right).
  \end{eqnarray}
  
  
  \subsection{Hull-White Model}

  Take $h(t)=\exp\left(-\int_0^t\kappa(u)du\right)$, and $g(t)=\sigma(t)\exp\left(\int_0^t\kappa(u)du\right)$, then we have
  \begin{equation}
    \frac{dy(t)}{dt} = \sigma^2(t) - 2\kappa(t)y(t),
  \end{equation}
  which has the following analytical solution,
  \begin{equation}
    y(t)=\int_0^t\sigma^2(u)\exp\left(-2\int_u^t\kappa(s)ds\right)du,
  \end{equation}
  Then,
  \begin{equation}
    dx(t)=\left(y(t)-\kappa(t)x(t)\right)dt+\sigma(t)dW(t),
  \end{equation}
  which can be shown to have the following solution,
  \begin{equation}
    x(t)=\int_0^ty(u)\exp\left(-\int_u^t\kappa(s)ds\right)du+\int_0^t\sigma(u)\exp\left(-\int_u^t\kappa(s)ds\right)dW(u).
  \end{equation}
  
  \subsection{Mercurio-Moraleda Model}

  Similar to the Vasicek model, set
  \begin{equation}
    \kappa(t)=\lambda-\frac{\gamma}{1+\gamma t},
  \end{equation}
    
    
    
\section{Examples of Classical Gaussian Short Rate Models - Multi Factor Models}

  \subsection{Deficiencies of One-Factor Models}

  

  \subsection{Hull-White Two-Factor Model}
    
    
    
    
    
    
    
    
    
    
    
    
    
    
    
    
    
    


\section{Quasi-Gaussian Models}


\section{Quadratic Gaussian Models}

\end{document}
