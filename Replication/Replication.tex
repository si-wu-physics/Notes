\documentclass[12pt]{article}
\usepackage{amsfonts}
\usepackage{amsmath}
\usepackage[hscale=.8,vscale=.8]{geometry}
\usepackage{hyperref}

\usepackage{amsthm}
\usepackage{enumitem}

\newtheorem{theorem}{Theorem}[section]
\newtheorem{corollary}{Corollary}[theorem]
\newtheorem{lemma}[theorem]{Lemma}

\begin{document}

\title{Notes on Static Hedging}
\date{Dec. 32, 2999}

\maketitle

\section{Replication of European Payoff}

  For a European call option with spot price $S_t$, strike $K$, expiration $T$, its value at $t$ can be determined by
  \begin{equation}
    C(K) = e^{-r\tau}\int_{0}^{+\infty}(S_T-K)^+p(t,S_t;T,S_T)dS_T = e^{-r\tau}\int_{K}^{+\infty}(S_T-K)p(t,S_t;T,S_T)dS_T,
  \end{equation}
  where $r$ is the risk free rate, $p(t,S_t;T,S_T)$ is the transition density for the asset price starting from $S_t$ at $t$ and
  ending with $S_T$ at $T$, and $\tau=T-t$. In the Black-Scholes world, the underlying process is
  \begin{equation}
    \frac{dS_t}{S_t}=(r-q)dt+\sigma dW_t,
  \end{equation}
  where $q$ is the dividend rate. The corresponding transition density is given by
  \begin{equation}
    p(t,S_t;T,S_T)=\frac{1}{S_T\sqrt{2\pi\sigma^2\tau}}
    \exp\left(-\frac{\displaystyle \left(\log\left(\frac{S_T}{S_t}\right)-\left(r-q-\frac{\sigma^2}{2}\right)\tau\right)^2}{2\sigma^2\tau}\right).
    \label{density}
  \end{equation}

  On the other hand, if we have the call option prices for a continuum of strikes, the transition density can be recovered by
  differentiating the call option price twice,
  \begin{equation}
    \frac{\partial^2C(K)}{\partial K^2}=e^{-r\tau}p(t,S_t;T,K).
  \end{equation}
  We can obtain the same result from put option prices, {\it i.e.},
  \begin{equation}
    \frac{\partial^2P(K)}{\partial K^2}=e^{-r\tau}p(t,S_t;T,K).
  \end{equation}

  Given the transition density, the price of a European option with arbitrary payoff $f(S_T)$ at maturity is given by
  \begin{equation}
    V = e^{-r\tau}\int_{0}^{+\infty}f(K)p(t,S_t;T,K)dK.
  \end{equation}
  This option price can also be represented in terms of out-of-the-money (OTM) option prices,
  \begin{equation}
    V = \int_{0}^{F}f(K)\frac{\partial^2P(K)}{\partial K^2}dK + \int_{F}^{+\infty}f(K)\frac{\partial^2C(K)}{\partial K^2}dK,
  \end{equation}
  where $F$ is an arbitrary positive number. Integrating the above equation by parts twice, we have
  \begin{equation}
    V = e^{-r\tau}f(F) + f^{\prime}(F)\big[e^{-q\tau}S_t-e^{-r\tau}F\big] +
           \int_{0}^{F}f^{\prime\prime}(K)P(K)dK + \int_{F}^{+\infty}f^{\prime\prime}(K)C(K)dK.
  \end{equation}
  Here, we have used the call-put parity,
  \begin{equation}
    C(K) - P(K) = e^{-q\tau}S_t-e^{-r\tau}K.
  \end{equation}
  Therefore, if $F$ is chosen to be the forward price $F_T=S_te^{(r-q)\tau}$, the linear term will drop out,
  \begin{equation}
    V = e^{-r\tau}f(F_T) +
           \int_{0}^{F_T}f^{\prime\prime}(K)P(K)dK + \int_{F_T}^{+\infty}f^{\prime\prime}(K)C(K)dK.
  \end{equation}
  This means that we can replicate a European option with arbitry payoff with plain vanilla options. The above intergrals in
  strike have to be discretized in reality, which will introduce some replication errors.


\section{Static Hedging of Barrier Options}

  Consider a down-and-in barrier option with barrier level $H < S_t$ with terminal payoff $f(S_T)$, the option price is given by
  \begin{equation}
    V_{DI} = e^{-r\tau}\int_{0}^{+\infty}f(K)p(t,S_t;T,K; H)dK,
  \end{equation}
  where $p(t,S_t;T,K; H)$ is the transition density from $S_t$ at $t$ to $K$ at $T$, while the underlying process has hit barrier $H$ at
  least once in this time frame. The option price can be decomposed
  into two parts,
  \begin{equation}
    V_{DI} = e^{-r\tau}\int_{0}^{H}f(K)p(t,S_t;T,K)dK + e^{-r\tau}\int_{H}^{+\infty}f(K)p(t,S_t;T,K;H)dK.
    \label{decomp}
  \end{equation}
  The first term is identical to its European counterpart, since the terminal underlying price is below the barrier level, and the underlying
  process must have cross the barrier level at least once. Therefore, the transition density reduces to a plain vanilla one. For the second term,
  assume the density for the first passage time to the barrier $H$ starting from $S_t$ at $t$ is given by $\phi(t,S_t;\delta,H)$, then
  \begin{equation}
    p(t,S_t;T,K;H) = \int_t^T \phi(t,S_t;\delta,H)p(\delta,H;T,K)d\delta.
  \end{equation}
  The second term of Eq. (\ref{decomp}) becomes
  \begin{eqnarray}
      &&e^{-r\tau}\int_{H}^{+\infty}f(K)p(t,S_t;T,K;H)dK \nonumber\\
    = &&e^{-r\tau}\int_t^Td\delta \phi(t,S_t;\delta,H)\int_{H}^{+\infty}dKf(K)p(\delta,H;T,K) \nonumber\\
    = &&e^{-r\tau}\int_t^Td\delta \phi(t,S_t;\delta,H)\int_0^HdK \left(\frac{H}{K}\right)^2f\left(\frac{H^2}{K}\right)p\left(\delta,H;T,\frac{H^2}{K}\right).
  \end{eqnarray}
  From Eq. (\ref{density}), we have
  \begin{eqnarray}
      && \left(\frac{H}{K}\right)^2p\left(\delta,H;T,\frac{H^2}{K}\right) \nonumber\\
    = && \left(\frac{H}{K}\right)^2\frac{1}{\displaystyle \frac{H^2}{K}\sqrt{2\pi\sigma^2\tau}}
    \exp\left(-\frac{\displaystyle \left(\log\left(\frac{H}{K}\right)-\left(r-q-\frac{\sigma^2}{2}\right)(T-\delta)\right)^2}{2\sigma^2(T-\delta)}\right) \nonumber\\
    = && \left(\frac{K}{H}\right)^k\frac{1}{\displaystyle K\sqrt{2\pi\sigma^2\tau}}
    \exp\left(-\frac{\displaystyle \left(\log\left(\frac{K}{H}\right)-\left(r-q-\frac{\sigma^2}{2}\right)(T-\delta)\right)^2}{2\sigma^2(T-\delta)}\right) \nonumber\\
    = && \left(\frac{K}{H}\right)^kp\left(\delta,H;T,K\right),
  \end{eqnarray}
  and the second term of Eq. (\ref{decomp}) becomes
  \begin{eqnarray}
      &&e^{-r\tau}\int_t^Td\delta \phi(t,S_t;\delta,H)\int_0^HdK f\left(\frac{H^2}{K}\right)\left(\frac{K}{H}\right)^kp\left(\delta,H;T,K\right)\nonumber\\
    = &&e^{-r\tau}\int_0^H \left(\frac{K}{H}\right)^kf\left(\frac{H^2}{K}\right)p\left(t, S_t;T,K\right)dK,
  \end{eqnarray}
  where $k=1-2(r-q)/\sigma^2$. Now, the down-and-in option price becomes
  \begin{equation}
    V_{DI} = e^{-r\tau}\int_0^{\infty} \hat{f}(K)p\left(t, S_t;T,K\right)dK,
  \end{equation}
  which means that the down-and-in option with final payoff $f(S_T)$ at maturity can be statically replicated with an European option with final payoff \cite{CarrChou}
  \begin{equation}
    \hat{f}(K) = \left[f(K) + \left(\frac{K}{H}\right)^kf\left(\frac{H^2}{K}\right)\right]\mathcal{I}_{K<H}
  \end{equation}
  at maturity. We can use the exact result from the previous section to further replicate this European option with plain vanilla ones.

  For down-and-out options, the replication strategy can be obtained from the difference between a vanilla option and a down-and-in option with the same barrier
  as the down-and-out option. The is due to the fact that a vanilla option can be decomposed into a down-and-out and a down-and-in options with same features.


\begin{thebibliography}{99}
  \bibitem{CarrChou}
    Peter Carr and Andrew Chou, {\it Breaking Barriers}, Risk {\bf 10}(9), 139 (1997).

\end{thebibliography}


\end{document}