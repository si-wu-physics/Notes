\documentclass[12pt]{article}
\usepackage{amsfonts}
\usepackage{amsmath}
\usepackage[hscale=.8,vscale=.8]{geometry}
\usepackage{hyperref}

\usepackage{amsthm}
\usepackage{enumitem}

\newtheorem{theorem}{Theorem}[section]
\newtheorem{corollary}{Corollary}[theorem]
\newtheorem{lemma}[theorem]{Lemma}

\allowdisplaybreaks

\begin{document}

\title{Notes on Replication}
\date{Dec. 32, 2999}

\maketitle

\section{Replication of a European Payoff}

  For a vanilla European call option with spot price $S_t$, strike $K$, expiration $T$, its value at $t$ can be determined by
  \begin{equation}
    C(K) = e^{-r\tau}\int_{0}^{+\infty}(S_T-K)^+p(t,S_t;T,S_T)dS_T = e^{-r\tau}\int_{K}^{+\infty}(S_T-K)p(t,S_t;T,S_T)dS_T,
  \end{equation}
  where $r$ is the risk free rate, $p(t,S_t;T,S_T)$ is the transition density for the asset price starting from $S_t$ at $t$ and
  ending with $S_T$ at $T$, and $\tau=T-t$. In the Black-Scholes world, the underlying process is
  \begin{equation}
    \frac{dS_t}{S_t}=(r-q)dt+\sigma dW_t,
    \label{BS}
  \end{equation}
  where $q$ is the dividend rate. The corresponding transition density is given by
  \begin{equation}
    p(t,S_t;T,S_T)=\frac{1}{S_T\sqrt{2\pi\sigma^2\tau}}
    \exp\left(-\frac{\displaystyle \left(\log\left(\frac{S_T}{S_t}\right)-\left(r-q-\frac{\sigma^2}{2}\right)\tau\right)^2}{2\sigma^2\tau}\right).
    \label{density}
  \end{equation}

  On the other hand, if we have the call option prices for a continuum of strikes, the transition density can be recovered by
  differentiating the call option price twice,
  \begin{equation}
    \frac{\partial^2C(K)}{\partial K^2}=e^{-r\tau}p(t,S_t;T,K).
    \label{call}
  \end{equation}
  We can obtain the same result from put option prices, {\it i.e.},
  \begin{equation}
    \frac{\partial^2P(K)}{\partial K^2}=e^{-r\tau}p(t,S_t;T,K).
    \label{put}
  \end{equation}

  Given the transition density, the price of a European contract with arbitrary payoff $f(S_T)$ at maturity is given by
  \begin{equation}
    V = e^{-r\tau}\int_{0}^{+\infty}f(K)p(t,S_t;T,K)dK.
    \label{euro}
  \end{equation}
  Using Eqs. (\ref{call}) and (\ref{put}), Eq. (\ref{euro}) can be represented in terms of out-of-the-money (OTM) option prices,
  \begin{equation}
    V = \int_{0}^{F}f(K)\frac{\partial^2P(K)}{\partial K^2}dK + \int_{F}^{+\infty}f(K)\frac{\partial^2C(K)}{\partial K^2}dK,
  \end{equation}
  where $F$ is an arbitrary positive number. Integrating the above equation by parts twice, we have
  \begin{equation}
    V = e^{-r\tau}f(F) + f^{\prime}(F)\big[e^{-q\tau}S_t-e^{-r\tau}F\big] +
           \int_{0}^{F}f^{\prime\prime}(K)P(K)dK + \int_{F}^{+\infty}f^{\prime\prime}(K)C(K)dK.
    \label{replication}
  \end{equation}
  Here, we have used the call-put parity,
  \begin{equation}
    C(K) - P(K) = e^{-q\tau}S_t-e^{-r\tau}K.
  \end{equation}
  Therefore, if $F$ is chosen to be the forward price $F_T=S_te^{(r-q)\tau}$, the linear term will drop out,
  \begin{equation}
    V = e^{-r\tau}f(F_T) +
           \int_{0}^{F_T}f^{\prime\prime}(K)P(K)dK + \int_{F_T}^{+\infty}f^{\prime\prime}(K)C(K)dK.
  \end{equation}
  This means that we can replicate a European contract with arbitry payoff with plain vanilla options.

  Another useful representation of the static replication is given by
  \begin{eqnarray}
    f(S_T)&=&f(F)+f^{\prime}(F)\left[\left(S_T-F\right)^+-\left(F-S_T\right)^+\right]\nonumber\\
            &&\ \ \ \ \ +\int_0^Ff^{\prime\prime}(K)\left(F-S_T\right)^+dK+\int_F^{\infty}f^{\prime\prime}(K)\left(S_T-F\right)^+dK.
    \label{replication2}
  \end{eqnarray}
  This can be viewed as the forward version of Eq. (\ref{replication}).

  \subsection{An Exmaple: Libor-in-arrears}

    For a Libor rate with a start date $T$ and an end date $M$, the forward Libor rate is given by
    \begin{equation}
      L(t)\equiv L(t,T,M)=\frac{1}{\tau}\left(\frac{P(t,T)}{P(t,M)}-1\right),
    \end{equation}
    where $P(t,T)$ is the time $t$ price of a zero-coupon bond maturing at $T$, and $\tau=M-T$. A standard Libor payment
    will be on the end date of the related Libor rate, and the discounted cash flow is
    \begin{eqnarray}
       &&E^Q\left[\frac{B(0)}{B(M)}L(T)\right]=E^M\left[\frac{P(0,M)}{P(M,M)}L(T)\right]\nonumber\\
      =&&P(0,M)E^M\left[L(T)\right]=P(0,M)L(0).
      \label{Libor}
    \end{eqnarray}
    Here, $Q$ and $M$ stand for the risk neutral and $M$-forward measures, respectively. We have also used the fact that $L(t)$
    is a martingale in the $M$-forward measure.

    This relation does not hold any more for a Libor-in-arrears (LIA) payment, which pays immediately after the Libor rate
    is fixed, {\it i.e.}, the discounted cash flow is
    \begin{equation}
      V_{LIA}(0)=E^Q\left[\frac{B(0)}{B(T)}L(T)\right],
    \end{equation}
    where the payment is on the start date $T$, rather than the end date $M$ as in Eq. (\ref{Libor}). If we apply the same
    change-of-numeraire technique as before, and use $M$-forward measure for convenience, we will have
    \begin{eqnarray}
      V_{LIA}(0)&=&E^M\left[\frac{P(0,M)}{P(T,M)}L(T)\right]=P(0,M)E^M\left[\left(1+\tau L(T)\right)L(T)\right]\nonumber\\
                &=&P(0,M)\left(L(0)+\tau E^M\left[L(T)^2\right]\right).
    \end{eqnarray}
    The $E^M\left[L(T)^2\right]$ term can be replicated with call and put options on the Libor rate, known as caplets and floorlets,
    which leads to
    \begin{equation}
      E^M\left[L(T)^2\right] = L(0)^2 + 2\int_{-\infty}^{L(0)}p(0,L(0);T,K)dK + 2\int_{L(0)}^{+\infty}c(0,L(0);T,K)dK,
    \end{equation}
    where $c(0,L(0);T,K)$ and $p(0,L(0);T,K)$ are the undiscounted caplet and floorlet prices with maturity $T$ and strike $K$.


\section{Static Replication of Barrier Options}

  Consider a down-and-in barrier option with barrier level $H < S_t$ with terminal payoff $f(S_T)$, the option price is given by
  \begin{equation}
    V_{DI} = e^{-r\tau}\int_{0}^{+\infty}f(K)p(t,S_t;T,K; H)dK,
  \end{equation}
  where $p(t,S_t;T,K; H)$ is the transition density from $S_t$ at $t$ to $K$ at $T$, while the underlying process has hit barrier $H$ at
  least once in this time frame. The option price can be decomposed
  into two parts,
  \begin{equation}
    V_{DI} = e^{-r\tau}\int_{0}^{H}f(K)p(t,S_t;T,K)dK + e^{-r\tau}\int_{H}^{+\infty}f(K)p(t,S_t;T,K;H)dK.
    \label{decomp}
  \end{equation}
  The first term is identical to its European counterpart, since the terminal underlying price is below the barrier level, and the underlying
  process must have cross the barrier level at least once. Therefore, the transition density reduces to a plain vanilla one. For the second term,
  assume the density for the first passage time to the barrier $H$ starting from $S_t$ at $t$ is given by $\phi(t,S_t;\delta,H)$, then
  \begin{equation}
    p(t,S_t;T,K;H) = \int_t^T \phi(t,S_t;\delta,H)p(\delta,H;T,K)d\delta.
  \end{equation}
  The second term of Eq. (\ref{decomp}) becomes
  \begin{eqnarray}
      &&e^{-r\tau}\int_{H}^{+\infty}f(K)p(t,S_t;T,K;H)dK \nonumber\\
    = &&e^{-r\tau}\int_t^Td\delta \phi(t,S_t;\delta,H)\int_{H}^{+\infty}dKf(K)p(\delta,H;T,K) \nonumber\\
    = &&e^{-r\tau}\int_t^Td\delta \phi(t,S_t;\delta,H)\int_0^HdK \left(\frac{H}{K}\right)^2f\left(\frac{H^2}{K}\right)p\left(\delta,H;T,\frac{H^2}{K}\right).
  \end{eqnarray}
  From Eq. (\ref{density}), we have
  \begin{eqnarray}
      && \left(\frac{H}{K}\right)^2p\left(\delta,H;T,\frac{H^2}{K}\right) \nonumber\\
    = && \left(\frac{H}{K}\right)^2\frac{1}{\displaystyle \frac{H^2}{K}\sqrt{2\pi\sigma^2\left(T-\delta\right)}}
    \exp\left(-\frac{\displaystyle \left(\log\left(\frac{H}{K}\right)-\left(r-q-\frac{\sigma^2}{2}\right)(T-\delta)\right)^2}{2\sigma^2(T-\delta)}\right) \nonumber\\
    = && \left(\frac{K}{H}\right)^k\frac{1}{\displaystyle K\sqrt{2\pi\sigma^2\left(T-\delta\right)}}
    \exp\left(-\frac{\displaystyle \left(\log\left(\frac{K}{H}\right)-\left(r-q-\frac{\sigma^2}{2}\right)(T-\delta)\right)^2}{2\sigma^2(T-\delta)}\right) \nonumber\\
    = && \left(\frac{K}{H}\right)^kp\left(\delta,H;T,K\right),
  \end{eqnarray}
  and the second term of Eq. (\ref{decomp}) becomes
  \begin{eqnarray}
      &&e^{-r\tau}\int_t^Td\delta \phi(t,S_t;\delta,H)\int_0^HdK f\left(\frac{H^2}{K}\right)\left(\frac{K}{H}\right)^kp\left(\delta,H;T,K\right)\nonumber\\
    = &&e^{-r\tau}\int_0^H \left(\frac{K}{H}\right)^kf\left(\frac{H^2}{K}\right)p\left(t, S_t;T,K\right)dK,
  \end{eqnarray}
  where $k=1-2(r-q)/\sigma^2$. Now, the down-and-in option price becomes
  \begin{equation}
    V_{DI} = e^{-r\tau}\int_0^{\infty} \hat{f}(K)p\left(t, S_t;T,K\right)dK,
  \end{equation}
  which means that the down-and-in option with final payoff $f(S)$ at maturity can be statically replicated with
  a European contract with terminal payoff \cite{CarrChou}
  \begin{equation}
    \hat{f}(S) = \left[f(S) + \left(\frac{S}{H}\right)^kf\left(\frac{H^2}{S}\right)\right]\mathcal{I}_{S<H}
  \end{equation}
  at maturity. It can be shown that this terminal payoff will lead to exactly the same expression for the
  down-and-in call option price. In this case, set $f(S)=\max(S-K_c,0)$, and assume $K_c>H$, we can recover
  the down-and-in price through Eq. (\ref{euro}), with the transition density (\ref{density}), {\it i.e.},
  \begin{eqnarray}
    V_{DIC} &=& e^{-r\tau}\int_0^{H^2/K_C}\left(\frac{H^2}{K}\right)^k\left(\frac{H^2}{K}-K_c\right)\nonumber\\
              &&\ \ \ \ \ \ \ \ \ \ \ \ \times\frac{1}{K\sqrt{2\pi\sigma^2\tau}}
    \exp\left(-\frac{\displaystyle \left(\log\left(\frac{K}{S_t}\right)-\left(r-q-\frac{\sigma^2}{2}\right)\tau\right)^2}{2\sigma^2\tau}\right)dK.
  \end{eqnarray}
  By change of variable, $K=H^2/e^v$, the above integration can be transformed to
  \begin{eqnarray}
    V_{DIC} &=& e^{-r\tau}H\int_{\log K_c}^{+\infty}e^{-ku}\left(e^u-K_c\right)\nonumber\\
              &&\ \ \ \ \ \ \ \ \ \ \ \ \times\frac{1}{\sqrt{2\pi\sigma^2\tau}}
    \exp\left(-\frac{\displaystyle \left(u-\log\left(\frac{H^2}{S_t}\right)+\left(r-q-\frac{\sigma^2}{2}\right)\tau\right)^2}{2\sigma^2\tau}\right)du.
  \end{eqnarray}
  Completing the squares, it can be shown that the down-and-in call price is
  \begin{eqnarray}
    V_{DIC} &=& e^{-q\tau}S_t\left(\frac{S_t}{H}\right)^{k-2}
              N\left(\frac{\displaystyle \log\left(\frac{H^2}{S_tK_c}\right)+\left(r-q+\frac{\sigma^2}{2}\right)\tau}{\sigma\sqrt{\tau}}\right)\nonumber\\
            &&\ \ \ \ - e^{-r\tau}K_c\left(\frac{S_t}{H}\right)^{k}
              N\left(\frac{\displaystyle \log\left(\frac{H^2}{S_tK_c}\right)+\left(r-q-\frac{\sigma^2}{2}\right)\tau}{\sigma\sqrt{\tau}}\right).
  \end{eqnarray}


\section{Variance swap}

  A variance swap is a forward contract on future realized variance. At maturity, the floating leg pays
  \begin{equation}
    A\times\frac{N}{n}\sum_{i=1}^{n}\log^2\left(\frac{S_i}{S_{i-1}}\right),
    \label{VarSwapPayoff}
  \end{equation}
  where $A$ is the notional amount of the variance swap, $N$ is an annualization factor and is usually set to
  252, $n$ is the number of monitor dates, and $S$ is the stock price at market close. The fixed leg payment is
  so determined that the variance swap has zero value at inception. In this section, we will discuss the valuation
  and replication of the variance swap, for both continuously and discretely monitored cases. Let us start with
  a very general result.

  \subsection{Profit and loss of a delta-hedged position}

    Consider the price of a European payoff $\phi(S)$ maturing at $T$, $V_{BS}(t,S_t;\sigma_{BS})$, where a constant volatility
    is assumed throughout the life time of the contract.
    For the discounted price, $f(t)=e^{-rt}V_{BS}(t,S_t;\sigma_{BS})$, we have
    \begin{equation}
      f(T)=f(0)+\int_0^Tdf(t)=f(0)+\int_0^Te^{-rt}\left[\left(-rV_{BS}+\frac{\partial V_{BS}}{\partial t}\right)dt+\frac{\partial V_{BS}}{\partial S}dS_t
              +\frac{1}{2}\frac{\partial^2 V_{BS}}{\partial S^2}\left(dS_t\right)^2\right].
    \end{equation}
    Assume that the underlying follows a geometric Brownian motion,
    \begin{equation}
      dS_t = \left(r-q\right)S_tdt + \sigma_tS_tdW_t,
    \end{equation}
    then
    \begin{eqnarray}
      f(T)&=&f(0)+\int_0^Te^{-rt}\left(-rV_{BS}+\frac{\partial V_{BS}}{\partial t}+\left(r-q\right)S_t\frac{\partial V_{BS}}{\partial S}
                                +\frac{1}{2}\sigma_t^2S_t^2\frac{\partial^2 V_{BS}}{\partial S^2}\right)dt\nonumber\\
          &&\quad\quad\quad+\int_0^Te^{-rt}\sigma_tS_t\frac{\partial V_{BS}}{\partial S}dW_t.
    \end{eqnarray}
    Notice that  $V(t,S_t;\sigma_{BS})$ satisfies the Black-Scholes PDE,
    \begin{equation}
      \frac{\partial V_{BS}}{\partial t}+\left(r-q\right)S\frac{\partial V_{BS}}{\partial S}+\frac{1}{2}\sigma_{BS}^2S^2\frac{\partial^2 V_{BS}}{\partial S^2}
      =rV_{BS},
    \end{equation}
    and finally, we have
    \begin{equation}
      f(T)=f(0)+\int_0^Te^{-rt}\frac{S_t^2}{2}\frac{\partial^2 V_{BS}}{\partial S^2}\left(\sigma_t^2-\sigma_{BS}^2\right)dt+\int_0^Te^{-rt}\sigma_tS_t\frac{\partial V_{BS}}{\partial S}dW_t.
      \label{average}
    \end{equation}
    The expected value of the hedged position is then given by
    \begin{equation}
      E\left[f(T)\right]=f(0)+E\left[\int_0^Te^{-rt}\frac{S_t^2}{2}\frac{\partial^2 V_{BS}}{\partial S^2}\left(\sigma_t^2-\sigma_{BS}^2\right)dt\right].
    \end{equation}
    From this, it is clear that, a delta-hedged option is sensitive to the difference between realized and implied variances.
    However, if one wants to trade
    the realized variance, the delta-hedged option is not a good choice, since the dollar gamma is peaked at the strike price.
    It can be shown, for a portfolio
    of options weighted by their inversed strike squared, $1/K^2$, the portfolio will be insensitive to the stock price move.

    As a side result, notice that, in an arbitrage free world, we have
    \begin{equation}
      E\left[f(T)\right]=f(0),
    \end{equation}
    which leads to
    \begin{equation}
      \sigma_{BS}^2=\frac{\displaystyle E\left[\int_0^Te^{-rt}\sigma_t^2S_t^2\frac{\partial^2 V_{BS}}{\partial S^2}dt\right]}
                         {\displaystyle E\left[\int_0^Te^{-rt}S_t^2\frac{\partial^2 V_{BS}}{\partial S^2}dt\right]}.
    \end{equation}
    This result implies that, the implied variance is the average of realized variance,
    weighted by dollar gamma. It can lead to interesting approximation for the implied
    volatility, for example, the so called ``most likely path" method. However, in practice, the approximation is rather crude.

  \subsection{Replication of a continuously monitored variance swap}

    In the limit of continuous monitoring, the floating leg payment of a variance swap becomes the integrated instantaneous
    variance, $\int_0^T\sigma_u^2du$. We can use the result from previous section to determine the replication of a
    continuously monitored variance swap.
    Suppose we risk manage our position with zero implied volatility, {\it i.e.}, $\sigma_{BS}=0$ in Eq. (\ref{average}). Then,
    Eq. (\ref{average}) can be written as
    \begin{equation}
      \int_0^Te^{-rt}\frac{S_t^2}{2}\frac{\partial^2 V_{BS}}{\partial S^2}\sigma_t^2dt
      =f(T)-f(0)-\int_0^Te^{-rt}\sigma_tS_t\frac{\partial V_{BS}}{\partial S}dW_t.
    \end{equation}
    If we assume the underlying follows the stochastic process
    \begin{equation}
      dS_t=(r-q)S_tdt+\sigma_tS_tdW_t,
    \end{equation}
    the last term becomes
    \begin{equation}
      \int_0^Te^{-rt}\frac{S_t^2}{2}\frac{\partial^2 V_{BS}}{\partial S^2}\sigma_t^2dt
      =f(T)-f(0)-\int_0^Te^{-rt}\frac{\partial V_{BS}}{\partial S}\left(dS_t-(r-q)S_tdt\right).
      \label{rep1}
    \end{equation}

    Now, choose
    \begin{equation}
      \phi(S) = 2\left[\log\left(\frac{F}{S}\right)+\frac{S}{F}-1\right],
    \end{equation}
    then its price at time $t$ with spot $S_t$ is
    \begin{equation}
      V_{BS}(t,S_t;\sigma_{BS} = 0) = 2e^{-r(T-t)}\left[\log\left(\frac{F}{S_t}\right)-(r-q)(T-t)+\frac{S_te^{(r-q)(T-t)}}{F}-1\right].
    \end{equation}
    The first and second order derivatives are
    \begin{equation}
      \frac{\partial V_{BS}}{\partial S}=2e^{-r(T-t)}\left[\frac{e^{(r-q)(T-t)}}{F}-\frac{1}{S_t}\right],
    \end{equation}
    and
    \begin{equation}
      \frac{\partial^2 V_{BS}}{\partial S^2}=\frac{2e^{-r(T-t)}}{S_t^2},
    \end{equation}
    repectively. With these derivatives, Eq. (\ref{rep1}) becomes
    \begin{eqnarray}
      \int_0^T\sigma_t^2dt &=& 2\left[\log\left(\frac{F}{S_T}\right)+\frac{S_T}{F}-1\right]\nonumber\\
                            && \ \ \ - 2\left[\log\left(\frac{F}{S_0}\right)-(r-q)T+\frac{S_0e^{(r-q)T}}{F}-1\right]\nonumber\\
                            && \ \ \ - 2\int_0^T\left[\frac{e^{(r-q)(T-t)}}{F}-\frac{1}{S_t}\right]\left(dS_t-(r-q)S_tdt\right).
    \end{eqnarray}
    Expand the last line, we have
    \begin{eqnarray}
      &&\int_0^T\left[\frac{e^{(r-q)(T-t)}}{F}-\frac{1}{S_t}\right]\left(dS_t-(r-q)S_tdt\right)\nonumber\\
      &=& \int_0^T\left[\frac{e^{(r-q)(T-t)}}{F}dS_t-\frac{dS_t}{S_t}-(r-q)S_t\frac{e^{(r-q)(T-t)}}{F}dt+(r-q)dt\right]\nonumber\\
      &=& \frac{S_T}{F} - \frac{S_0e^{(r-q)T}}{F} + \int_0^T\frac{dS_t}{S_t} + (r-q)T,
    \end{eqnarray}
    where we have performed integration by parts for the third term on the second line in the above equations.
    Finally, we have
    \begin{eqnarray}
      \int_0^T\sigma_t^2dt &=& 2\left[\log\left(\frac{F}{S_T}\right)+\frac{S_T}{F}-1\right]
                                     - 2\left[\log\left(\frac{F}{S_0}\right)+\frac{S_0}{F}-1\right]\nonumber\\
                            && \ \ \ - 2\int_0^T\left[\frac{1}{F}-\frac{1}{S_t}\right]dS_t\nonumber\\
                           &=& 2\left[\log\left(\frac{S_0}{S_T}\right)+\int_0^T\frac{dS_t}{S_t}\right].
    \end{eqnarray}
    This means that a continuously monitored variance swap can be replicated by a static position of a log contract,
    and a dynamically traded underlying position.

    Now, write $\log(S_0/S_T)$ as $\log(S_0/S_*)+\log(S_*/S_T)$, and notice that the fair strike of the variance swap is
    the risk neutral expectation of time averaged realized variance, we have
    \begin{eqnarray}
      K_{var} &=& E\left[\frac{1}{T}\int_0^T\sigma_t^2dt\right]\nonumber\\
              &=& \frac{2}{T}\Bigg[(r-q)T-\frac{S_0}{S_*}e^{(r-q)T}+1-\log\left(\frac{S_*}{S_0}\right)\nonumber\\
              &&\ \ \ \ +e^{rT}\int_0^{S_*}P(K)\frac{dK}{K^2}+e^{rT}\int_{S_*}^{+\infty}C(K)\frac{dK}{K^2}\Bigg].
      \label{VarSwapRate}
    \end{eqnarray}

  \subsection{VIX}

    In Eq. (\ref{VarSwapRate}), if we set the separator $S_*$ with the forward price of the underlying, {\it i.e.},
    \begin{equation}
      S_*=F_T=S_0e^{(r-q)T},
    \end{equation}
    the fair strike $K_{var}$ will become
    \begin{eqnarray}
      K_{var} &=& \frac{2e^{rT}}{T}\Bigg[\int_0^{F_T}P(K)\frac{dK}{K^2}+\int_{F_T}^{+\infty}C(K)\frac{dK}{K^2}\Bigg]\nonumber\\
              &=& \frac{2e^{rT}}{T}\Bigg[\int_0^{K_0}P(K)\frac{dK}{K^2}+\int_{K_0}^{+\infty}C(K)\frac{dK}{K^2}
                                        +\int_{K_0}^{F_T}\bigg(P(K)-C(K)\bigg)\frac{dK}{K^2}\Bigg]\nonumber\\
              &=& \frac{2e^{rT}}{T}\Bigg[\int_0^{K_0}P(K)\frac{dK}{K^2}+\int_{K_0}^{+\infty}C(K)\frac{dK}{K^2}\Bigg]
                              +\frac{2}{T}\int_{K_0}^{F_T}\bigg(K-F_T\bigg)\frac{dK}{K^2}\nonumber\\
              &=& \frac{2e^{rT}}{T}\Bigg[\int_0^{K_0}P(K)\frac{dK}{K^2}+\int_{K_0}^{+\infty}C(K)\frac{dK}{K^2}\Bigg]
                              -\frac{1}{T}\bigg(\frac{F_T}{K_0}-1\bigg)^2.
    \end{eqnarray}
    After discretization, the fair strike is
    \begin{equation}
       K_{var} = \frac{2e^{rT}}{T}\sum_i\frac{\Delta K_i}{K_i^2}Q(K_i)-\frac{1}{T}\bigg(\frac{F_T}{K_0}-1\bigg)^2.
    \end{equation}
    Here, $F_T$ is the forward index value derived from option prices. Specifically, from a strip of option prices,
    we determine the forward index value by identifying the strike price $K_c$ at which the absolute difference between
    call and put option prices are the smallest. Here, the call and put option prices are taken to be the average of
    the corresponding bid and ask values. Then, the forward can be determined by
    \begin{equation}
      F_T = K_c + e^{rT}\left(C(K_c)-P(K_c)\right),
    \end{equation}
    where $r$ is the prevailing risk-free rate for option expiry $T$, which is usually taken to be the corresponding
    US treasury yield. $K_0$ will be set to the strike level immediately below the forward value, and $\Delta K_i$ is
    half the difference between the two adjacent strikes below and above $K_i$, {\it i.e.}, $\Delta K_i=(K_{i+1}-K_{i-1})/2$.
    For the highest and lowest strike to be used in the option strip, $\Delta K_i$ is calculated from only one side.
    Now, for $K_i$ greater than $K_0$, $Q(K_i)$ will be set to the out-of-money call price with strike $K_i$. For $K_i$
    less than $K_0$, the out-of-money put price is used. For $K_0$, it is the average of the call and put option prices.

    This is the basic component for the VIX index. For more details, see the VIX White Paper. \cite{VIX}

  \subsection{Discretely monitored variance swap}

    Now, let us consider the floating leg payment of a discretely monitored variance swap \cite{CarrLee}
    \begin{equation}
      V = \frac{N}{n}\sum_{i=1}^n\log^2\left(\frac{S_i}{S_{i-1}}\right).
    \end{equation}
    Define the simple return as
    \begin{equation}
      R_i=\frac{S_i-S_{i-1}}{S_i},
    \end{equation}
    the log return can be Taylor expanded to third order in terms of the simple return,
    \begin{equation}
      \log\left(\frac{S_i}{S_{i-1}}\right)=R_i-\frac{1}{2}R_i^2+\frac{1}{3}R_i^3.
      \label{Taylor}
    \end{equation}
    Squaring both sides, we have
    \begin{equation}
      \log^2\left(\frac{S_i}{S_{i-1}}\right)=R_i^2-R_i^3.
    \end{equation}
    Solving for $R_i^2$ and plug back into Eq. (\ref{Taylor}),
    \begin{equation}
      \log\left(\frac{S_i}{S_{i-1}}\right)=R_i-\frac{1}{2}\log^2\left(\frac{S_i}{S_{i-1}}\right)-\frac{1}{6}R_i^3,
    \end{equation}
    which finally leads to
    \begin{equation}
      \log^2\left(\frac{S_i}{S_{i-1}}\right)=2R_i-2\log\left(\frac{S_i}{S_{i-1}}\right)-\frac{1}{3}R_i^3.
    \end{equation}
    Therefore, the floating leg payment becomes
    \begin{equation}
      \sum_{i=1}^n\log^2\left(\frac{S_i}{S_{i-1}}\right)=\sum_{i=1}^n\frac{2}{S_{i-1}}(S_i-S_{i-1})-2\log\left(\frac{S_n}{S_0}\right)-\frac{1}{3}\sum_{i=1}^nR_i^3.
    \end{equation}
    Using Eq. (\ref{replication2}), $\log S_n$ can be expanded around $S_0$,
    \begin{equation}
      \log S_n = \log S_0 +\frac{1}{S_0}\left(S_n-S_0\right)-\int_0^{S_0}\left(K-S_n\right)^+\frac{dK}{K^2}-\int_{S_0}^{\infty}\left(S_n-K\right)^+\frac{dK}{K^2},
    \end{equation}
    we can obtain the desired decomposition,
    \begin{eqnarray}
      \sum_{i=1}^n\log^2\left(\frac{S_i}{S_{i-1}}\right)&=&2\sum_{i=1}^n\left(\frac{1}{S_{i-1}}-\frac{1}{S_0}\right)(S_i-S_{i-1})\nonumber\\
          && + 2\left[\int_0^{S_0}\left(K-S_n\right)^+\frac{dK}{K^2}+\int_{S_0}^{\infty}\left(S_n-K\right)^+\frac{dK}{K^2}\right]-\frac{1}{3}\sum_{i=1}^nR_i^3
    \end{eqnarray}
    If the third order term sums to a negative number, the variance swap will be mispriced when only the first two terms are
    taken into account for replication.

    If we use the simple return in the variance swap floating leg payment, instead of the log return, we can obtain a similar decomposition,
    \begin{eqnarray}
      \sum_{i=1}^n\left(\frac{S_i}{S_{i-1}}\right)^2&=&2\sum_{i=1}^n\left(\frac{1}{S_{i-1}}-\frac{1}{S_0}\right)(S_i-S_{i-1})\nonumber\\
          && + 2\left[\int_0^{S_0}\left(K-S_n\right)^+\frac{dK}{K^2}+\int_{S_0}^{\infty}\left(S_n-K\right)^+\frac{dK}{K^2}\right]+\frac{2}{3}\sum_{i=1}^nR_i^3,
    \end{eqnarray}
    which differs only by the third order term. We can see that, if the cubed simple return sums to a negative number and the variance swap is priced and risk
    managed with static positions in options, then the variance swap with log return is preferred from a buyer's perspective.


\section{Weighted variance swap}

  The method developed in the previous section can be applied to other types of variance swap. We will show a few examples in
  this section.

  \subsection{Corridor variance swap}

    The floating leg payment of the corridor variance swap is
    \begin{equation}
      A\times\frac{N}{n}\sum_{i=1}^{n}\mathcal{I}_{L\leq S_i \leq H}\log^2\left(\frac{S_i}{S_{i-1}}\right),
    \end{equation}
    {\it i.e.}, the contribution to the realized variance is only taken into account if the stock price falls
    in the corridor $[L,H]$.

    Take \cite{CarrMadan}
    \begin{equation}
      \phi(S) = 2\left[\log\left(\frac{F}{S^{\prime}}\right)+S\left(\frac{1}{F}-\frac{1}{S^{\prime}}\right)\right],
    \end{equation}
    where
    \begin{equation}
      S^{\prime} = {\rm max}\left[L, {\rm min}\left(S,H\right)\right]
    \end{equation}
    is floored at $L$ and capped at $H$, {\it i.e.}, the value of $S^{\prime}$ is restricted in the corridor $[L,H]$.
    Following the same argument as for the variance swap, we have
    \begin{eqnarray}
      \int_0^T\sigma_t^2\mathcal{I}_{L\leq S \leq H}dt &=& 2\left[\log\left(\frac{F}{S_T^{\prime}}\right)+\frac{S_T^{\prime}}{F}-1\right]
                                     - 2\left[\log\left(\frac{F}{S_0}\right)+\frac{S_0}{F}-1\right]\nonumber\\
                            && \ \ \ - 2\int_0^T\left[\frac{1}{F}-\frac{1}{S_t^{\prime}}\right]dS_t.
    \end{eqnarray}

  \subsection{Gamma swap}

    The floating leg payment of the gamma swap is
    \begin{equation}
      A\times\frac{N}{n}\sum_{i=1}^{n}\frac{S_i}{S_0}\log^2\left(\frac{S_i}{S_{i-1}}\right).
    \end{equation}

    Take
    \begin{equation}
      \phi(S) = \frac{2}{S_0}\left[S\log\left(\frac{S}{F}\right)-S+F\right],
    \end{equation}
    then
    \begin{eqnarray}
      \int_0^T\frac{S}{S_0}\sigma_t^2dt &=& \frac{2}{S_0}\left[S_T\log\left(\frac{S_T}{F}\right)-S_T+F\right]
                                     - \frac{2}{S_0}\left[S_0\log\left(\frac{S_0}{F}\right)-S_0+F\right]\nonumber\\
                            && \ \ \ - \frac{2}{S_0}\int_0^T\log\left(\frac{S_t}{F}\right)dS_t.
    \end{eqnarray}


\section{Volatility swap}

  A volatility swap is a forward contract on future realized volatility.  The volatility swap payoff lies closely
  to the variance swap, and the floating leg payment is
  \begin{equation}
    A\sqrt{\times\frac{N}{n}\sum_{i=1}^{n}\log^2\left(\frac{S_i}{S_{i-1}}\right)},
  \end{equation}
  {\it i.e.}, the square root of Eq. (\ref{VarSwapPayoff}). Due to the square root, volatility swap is more complicated
  than variance swap in valuation and hedging. The valuation of a volatility swap generally requires a stochastic volatility
  model, and depends on the correlation between the underlying and the stochastic volatility processes. However, as shown by
  Carr and Lee \cite{CarrLee2}, there exists model-free replication strategies for the volatility swaps, insensitive to the correlation in
  the first oder.

  \subsection{Valuation under stoachstic volatility}
    For a European payoff $F(S,\omega)$, where $\omega$ represents other stochastic factors that affect the payoff,
    define
    \begin{equation}
      F^{BS}\left(S, \sigma, \omega\right) = \int_0^{+\infty}F(Sy,\omega)
                          \frac{1}{y\sqrt{2\pi\sigma^2}}\exp\left(-\frac{\left(y+\sigma^2/2\right)^2}{2\sigma^2}\right)dy.
    \end{equation}
    This gives the Black-Scholes result if there is no $\omega$ dependence.

    Consider the following stochastic process for an underlying,
    \begin{equation}
      dS_t=\sqrt{1-\rho^2}\sigma_tS_tdW_{1t}+\rho\sigma_tS_tdW_{2t},
      \label{BS1}
    \end{equation}
    where the volatility process $\sigma_t$ is independent on the Brownian motion $W_1$, and the underlying is correlated
    with the volatility with correlation $\rho$. Let $X_t\equiv\log(S_t/S_0)$, then
    \begin{equation}
      dX_t=-\frac{1-\rho^2}{2}\sigma_t^2dt+\sqrt{1-\rho^2}\sigma_tdW_{1t}-\frac{\rho^2}{2}\sigma_t^2dt+\rho\sigma_tS_tdW_{2t}.
    \end{equation}
    Conditioning on $W_1$, $X_t$ is normally distributed,
    \begin{equation}
      X_T\sim N\left(X_t+\log M_{t,T}(\rho)-\frac{1-\rho^2}{2}\bar\sigma_{t,T}^2dt, \sqrt{1-\rho^2}\bar\sigma_{t,T}\right),
    \end{equation}
    where
    \begin{equation}
      M_{t,T}(\rho) = \exp\left(-\frac{\rho^2}{2}\int_t^T\sigma_u^2du+\rho\int_t^T\sigma_udW_u\right),
    \end{equation}
    and
    \begin{equation}
      \bar\sigma_{t,T}^2=\int_t^T\sigma_u^2du.
    \end{equation}
    Now, apply repeated expectation, we have
    \begin{equation}
      E_t\left[F(S_T)\right]=E_t\left[F^{BS}\left(S_tM_{t,T}(\rho),\bar\sigma_{t,T}\sqrt{1-\rho^2}\right)\right].
    \end{equation}
    Clearly, this result depends on the correlation $\rho$. If we can find a payoff $F(S)$ which can render a correlation-insensitive
    result, or at least correlation-insensitive locally around $\rho=0$, then we can have (nearly) model-free valuation for the volatility
    swap. This can be achieved by expanding the above result around $\rho=0$,
    \begin{eqnarray}
             E_t\left[F(S_T)\right]
      &=&E_t\left[F^{BS}\left(S_tM_{t,T}(\rho),\bar\sigma_{t,T}\sqrt{1-\rho^2}\right)\right]\nonumber\\
      &=&E_t\left[F^{BS}(S_t,\bar\sigma_{t,T})\right]
              + \rho S_tE_t\left[\frac{\partial F^{BS}(S_t,\bar\sigma_{t,T})}{\partial S}
                              \int_t^T\sigma_udW_u\right].
      \label{immune}
    \end{eqnarray}
    If $\partial F^{BS}/\partial S$ does not depend on $\bar\sigma_{t,T}$, then the expectation in the last line will be zero,
    and the above result will be correlation-insensitive, with an error of order $O(\rho^2)$.

    \subsection{Characteristic function of realized variance}
    Define
    \begin{equation}
      X_t\equiv\log\frac{S_t}{S_0},
    \end{equation}
    and
    \begin{equation}
      \langle X \rangle_t\equiv\int_0^t\sigma_u^2du.
    \end{equation}
    Assuming the underlying process Eq. (\ref{BS1}), these are related by
    \begin{equation}
      \langle X \rangle_t=-2X_t+\int_0^t\frac{2}{S_u}dS_u.
    \end{equation}
    Consider the realized variance from $t$ to $T$,
    \begin{equation}
      X_T-X_t=-\frac{1}{2}\left(\langle X \rangle_T - \langle X \rangle_t\right) + \int_t^T\sigma_udW_u,
    \end{equation}
    which means that $X_T-X_t$ is normally distributed, {\it i.e.},
    \begin{equation}
      X_T-X_t\sim N\left(-\frac{1}{2}\left(\langle X \rangle_T - \langle X \rangle_t\right), \langle X \rangle_T - \langle X \rangle_t\right).
    \end{equation}
    Therefore,
    \begin{equation}
      E_t\left[\exp\left\{p\left(X_T-X_t\right)\right\}\right] =
      E_t\left[\exp\left\{\frac{1}{2}\left(p^2-p\right)\left(\langle X \rangle_T - \langle X \rangle_t\right)\right\}\right].
    \end{equation}
    Denote $\lambda=(p^2-p)/2$, we can solve $p$ for $\lambda$, and get
    \begin{equation}
      p=\frac{1}{2}\pm\sqrt{\frac{1}{4}+2\lambda}.
    \end{equation}
    So, the characteristic function for the realized variance is
    \begin{equation}
    E_t\left[\exp\left\{\lambda\left(\langle X \rangle_T - \langle X \rangle_t\right)\right\}\right]
       = E_t\left[\exp\left\{p_{\pm}\left(X_T-X_t\right)\right\}\right].
    \end{equation}

    \subsection{Correlation-insensitive replication of a volatility swap}
      Now, let us consider the replication and valuation of the volatility swap. Use the following identity,
      \begin{equation}
        \sqrt{q} = \frac{1}{2\sqrt{\pi}}\int_0^{+\infty}\frac{1-2^{-zq}}{z^{3/2}}dz,
      \end{equation}
      the floating leg payment of the volatility swap can be expressed as
      \begin{equation}
        E_t\left[\sqrt{\langle X \rangle_T - \langle X \rangle_t+q}\right]
          =\frac{1}{2\sqrt{\pi}}E_t\left[\int_0^{+\infty}
          \left(1-\exp\left\{-z\left(\langle X \rangle_T - \langle X \rangle_t+q\right)\right\}\right)\right]
          \frac{dz}{z^{3/2}}.
      \end{equation}
      Here, we have considered a very general scenario, so that we can handle both a newly issued volatility
      swap and a seasoned one.

      The integration and the expectation in the above equation can be exchanged, then
      \begin{eqnarray}
                          E_t\left[\sqrt{\langle X \rangle_T - \langle X \rangle_t+q}\right]
        &=&\frac{1}{2\sqrt{\pi}}\int_0^{+\infty}\sum_{\pm}\theta_{\pm}\left(1-e^{-zq}E_t\left[\exp\left\{p_{\pm}\left(X_T-X_t\right)\right\}\right]\right)\frac{dz}{z^{3/2}}\nonumber\\
        &=&\frac{1}{2\sqrt{\pi}}E_t\left[\int_0^{+\infty}\sum_{\pm}\theta_{\pm}\left(1-e^{-zq}\exp\left\{p_{\pm}\left(X_T-X_t\right)\right\}\right)\frac{dz}{z^{3/2}}\right].
      \end{eqnarray}
      Here,
      \begin{equation}
        p_{\pm}=\frac{1}{2}\pm\sqrt{\frac{1}{4}-2z},
      \end{equation}
      and $\theta_++\theta_-=1$. We can use the freedom in choosing $\theta_{\pm}$ to achieve the correlation immunity. With zero correlation, $\rho=0$,
      $S_T$ is lognormally distributed. The expectation in the above equation can be transform into an integration with respect to a lognormal density with
      parameters $(-v/2, v)$. Then, the derivative in Eq. (\ref{immune}) becomes
      \begin{eqnarray}
        \frac{\partial F^{BS}(S_t)}{\partial s} &=& \frac{1}{2\sqrt{\pi}}\left.\frac{\partial}{\partial s}\right|_{s=S_t}\int_0^{+\infty}
                    \left[\int_0^{+\infty}\sum_{\pm}\theta_{\pm}\left(1-e^{-zq}\left(sy/S_t\right)^{p_{\pm}}\right)\frac{dz}{z^{3/2}}\right]\phi_v(y)dy\nonumber\\
                    &=&\frac{1}{2\sqrt{\pi}}\int_0^{+\infty}\int_0^{+\infty}\frac{-e^{-zq}\left(\theta_+ p_+ y^{p_+}+\theta_- p_- y^{p_-}\right)}{S_tz^{3/2}}\phi_v(y)dzdy.
      \end{eqnarray}
      Since
      \begin{equation}
        \int_0^{+\infty}y^{p_+}\phi_v(y)dy = \int_0^{+\infty}y^{p_-}\phi_v(y)dy,
      \end{equation}
      the correlation immunity can be achieved by
      \begin{equation}
        \theta_+ p_+ + \theta_- p_- = 0,
      \end{equation}
      which leads to
      \begin{equation}
        \theta_{\pm}=\frac{1}{2}\mp\frac{1}{2\sqrt{1-8z}}.
      \end{equation}

      Finally, the correlation-immune price of a volatility swap is given by
      \begin{equation}
        E_t\left[\sqrt{\langle X \rangle_T}\right] = E_t\left[G_{SVS}\left(S_T,S_t,\langle X \rangle_t\right)\right],
      \end{equation}
      where
      \begin{equation}
        G_{SVS}\left(S_T,S_t,q\right) = \frac{1}{2\sqrt{\pi}}\int_t^{+\infty}\left(\theta_+\left[1-e^{-zq}\left(\frac{S_T}{S_t}\right)^{p_+}\right]
                                                                                + \theta_-\left[1-e^{-zq}\left(\frac{S_T}{S_t}\right)^{p_-}\right]\right)\frac{dz}{z^{3/2}}.
      \end{equation}

      At inception, the above expression can be simplified,
      \begin{equation}
        E_0\left[\sqrt{\langle X \rangle_T}\right] = E_0\left[\psi(S_T)\right],
      \end{equation}
      where $\psi(S)=\phi(\log(S/S_0))$, and
      \begin{equation}
        \phi(x)=\sqrt{\frac{\pi}{2}}e^{x/2}\left|xI_0\left(\frac{x}{2}\right)-xI_1\left(\frac{x}{2}\right)\right|.
      \end{equation}
      Here, $I_{\nu}$ is the modified Bessel function of order $\nu$.


\begin{thebibliography}{99}
  \bibitem{CarrChou}
    Peter Carr and Andrew Chou, {\it Breaking Barriers}, Risk {\bf 10}(9), 139 (1997).

  \bibitem{CarrLee}
    Peter Carr and Roger Lee, {\it Volatility Derivatives}, Annu. Rev. Financ. Econ. {\bf 1}, 319 (2009).

  \bibitem{CarrMadan}
    Peter Carr and Dilip Madan, {\it Towards a Theory of Volatility Trading}.

  \bibitem{VIX}
    {\it VIX White Paper}, \url{https://www.cboe.com/micro/vix/vixwhite.pdf}.

  \bibitem{CarrLee2}
    Peter Carr and Roger Lee, {\it Robust Replication of Volatility Derivatives}, \url{https://math.uchicago.edu/~rl/rrvd.pdf}.

\end{thebibliography}


\end{document}