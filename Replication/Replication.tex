\documentclass[12pt]{article}
\usepackage{amsfonts}
\usepackage{amsmath}
\usepackage[hscale=.8,vscale=.8]{geometry}
\usepackage{hyperref}

\usepackage{amsthm}
\usepackage{enumitem}

\newtheorem{theorem}{Theorem}[section]
\newtheorem{corollary}{Corollary}[theorem]
\newtheorem{lemma}[theorem]{Lemma}

\begin{document}

\title{Notes on Static Replication}
\date{Dec. 32, 2999}

\maketitle

\section{Replication of a European Payoff}

  For a vanilla European call option with spot price $S_t$, strike $K$, expiration $T$, its value at $t$ can be determined by
  \begin{equation}
    C(K) = e^{-r\tau}\int_{0}^{+\infty}(S_T-K)^+p(t,S_t;T,S_T)dS_T = e^{-r\tau}\int_{K}^{+\infty}(S_T-K)p(t,S_t;T,S_T)dS_T,
  \end{equation}
  where $r$ is the risk free rate, $p(t,S_t;T,S_T)$ is the transition density for the asset price starting from $S_t$ at $t$ and
  ending with $S_T$ at $T$, and $\tau=T-t$. In the Black-Scholes world, the underlying process is
  \begin{equation}
    \frac{dS_t}{S_t}=(r-q)dt+\sigma dW_t,
    \label{BS}
  \end{equation}
  where $q$ is the dividend rate. The corresponding transition density is given by
  \begin{equation}
    p(t,S_t;T,S_T)=\frac{1}{S_T\sqrt{2\pi\sigma^2\tau}}
    \exp\left(-\frac{\displaystyle \left(\log\left(\frac{S_T}{S_t}\right)-\left(r-q-\frac{\sigma^2}{2}\right)\tau\right)^2}{2\sigma^2\tau}\right).
    \label{density}
  \end{equation}

  On the other hand, if we have the call option prices for a continuum of strikes, the transition density can be recovered by
  differentiating the call option price twice,
  \begin{equation}
    \frac{\partial^2C(K)}{\partial K^2}=e^{-r\tau}p(t,S_t;T,K).
    \label{call}
  \end{equation}
  We can obtain the same result from put option prices, {\it i.e.},
  \begin{equation}
    \frac{\partial^2P(K)}{\partial K^2}=e^{-r\tau}p(t,S_t;T,K).
    \label{put}
  \end{equation}

  Given the transition density, the price of a European contract with arbitrary payoff $f(S_T)$ at maturity is given by
  \begin{equation}
    V = e^{-r\tau}\int_{0}^{+\infty}f(K)p(t,S_t;T,K)dK.
    \label{euro}
  \end{equation}
  Using Eqs. (\ref{call}) and (\ref{put}), Eq. (\ref{euro}) can be represented in terms of out-of-the-money (OTM) option prices,
  \begin{equation}
    V = \int_{0}^{F}f(K)\frac{\partial^2P(K)}{\partial K^2}dK + \int_{F}^{+\infty}f(K)\frac{\partial^2C(K)}{\partial K^2}dK,
  \end{equation}
  where $F$ is an arbitrary positive number. Integrating the above equation by parts twice, we have
  \begin{equation}
    V = e^{-r\tau}f(F) + f^{\prime}(F)\big[e^{-q\tau}S_t-e^{-r\tau}F\big] +
           \int_{0}^{F}f^{\prime\prime}(K)P(K)dK + \int_{F}^{+\infty}f^{\prime\prime}(K)C(K)dK.
    \label{replication}
  \end{equation}
  Here, we have used the call-put parity,
  \begin{equation}
    C(K) - P(K) = e^{-q\tau}S_t-e^{-r\tau}K.
  \end{equation}
  Therefore, if $F$ is chosen to be the forward price $F_T=S_te^{(r-q)\tau}$, the linear term will drop out,
  \begin{equation}
    V = e^{-r\tau}f(F_T) +
           \int_{0}^{F_T}f^{\prime\prime}(K)P(K)dK + \int_{F_T}^{+\infty}f^{\prime\prime}(K)C(K)dK.
  \end{equation}
  This means that we can replicate a European contract with arbitry payoff with plain vanilla options.

  Another usefull representation of the static replication is given by
  \begin{eqnarray}
    f(S_T)&=&f(F)+f^{\prime}(F)\left[\left(S_T-F\right)^+-\left(F-S_T\right)^+\right]\nonumber\\
            &&\ \ \ \ \ +\int_0^Ff^{\prime\prime}(K)\left(F-S_T\right)^+dK+\int_F^{\infty}f^{\prime\prime}(K)\left(S_T-F\right)^+dK.
  \end{eqnarray}
  This can be viewed as the forward version of Eq. (\ref{replication}).


\section{Static Replication of Barrier Options}

  Consider a down-and-in barrier option with barrier level $H < S_t$ with terminal payoff $f(S_T)$, the option price is given by
  \begin{equation}
    V_{DI} = e^{-r\tau}\int_{0}^{+\infty}f(K)p(t,S_t;T,K; H)dK,
  \end{equation}
  where $p(t,S_t;T,K; H)$ is the transition density from $S_t$ at $t$ to $K$ at $T$, while the underlying process has hit barrier $H$ at
  least once in this time frame. The option price can be decomposed
  into two parts,
  \begin{equation}
    V_{DI} = e^{-r\tau}\int_{0}^{H}f(K)p(t,S_t;T,K)dK + e^{-r\tau}\int_{H}^{+\infty}f(K)p(t,S_t;T,K;H)dK.
    \label{decomp}
  \end{equation}
  The first term is identical to its European counterpart, since the terminal underlying price is below the barrier level, and the underlying
  process must have cross the barrier level at least once. Therefore, the transition density reduces to a plain vanilla one. For the second term,
  assume the density for the first passage time to the barrier $H$ starting from $S_t$ at $t$ is given by $\phi(t,S_t;\delta,H)$, then
  \begin{equation}
    p(t,S_t;T,K;H) = \int_t^T \phi(t,S_t;\delta,H)p(\delta,H;T,K)d\delta.
  \end{equation}
  The second term of Eq. (\ref{decomp}) becomes
  \begin{eqnarray}
      &&e^{-r\tau}\int_{H}^{+\infty}f(K)p(t,S_t;T,K;H)dK \nonumber\\
    = &&e^{-r\tau}\int_t^Td\delta \phi(t,S_t;\delta,H)\int_{H}^{+\infty}dKf(K)p(\delta,H;T,K) \nonumber\\
    = &&e^{-r\tau}\int_t^Td\delta \phi(t,S_t;\delta,H)\int_0^HdK \left(\frac{H}{K}\right)^2f\left(\frac{H^2}{K}\right)p\left(\delta,H;T,\frac{H^2}{K}\right).
  \end{eqnarray}
  From Eq. (\ref{density}), we have
  \begin{eqnarray}
      && \left(\frac{H}{K}\right)^2p\left(\delta,H;T,\frac{H^2}{K}\right) \nonumber\\
    = && \left(\frac{H}{K}\right)^2\frac{1}{\displaystyle \frac{H^2}{K}\sqrt{2\pi\sigma^2\left(T-\delta\right)}}
    \exp\left(-\frac{\displaystyle \left(\log\left(\frac{H}{K}\right)-\left(r-q-\frac{\sigma^2}{2}\right)(T-\delta)\right)^2}{2\sigma^2(T-\delta)}\right) \nonumber\\
    = && \left(\frac{K}{H}\right)^k\frac{1}{\displaystyle K\sqrt{2\pi\sigma^2\left(T-\delta\right)}}
    \exp\left(-\frac{\displaystyle \left(\log\left(\frac{K}{H}\right)-\left(r-q-\frac{\sigma^2}{2}\right)(T-\delta)\right)^2}{2\sigma^2(T-\delta)}\right) \nonumber\\
    = && \left(\frac{K}{H}\right)^kp\left(\delta,H;T,K\right),
  \end{eqnarray}
  and the second term of Eq. (\ref{decomp}) becomes
  \begin{eqnarray}
      &&e^{-r\tau}\int_t^Td\delta \phi(t,S_t;\delta,H)\int_0^HdK f\left(\frac{H^2}{K}\right)\left(\frac{K}{H}\right)^kp\left(\delta,H;T,K\right)\nonumber\\
    = &&e^{-r\tau}\int_0^H \left(\frac{K}{H}\right)^kf\left(\frac{H^2}{K}\right)p\left(t, S_t;T,K\right)dK,
  \end{eqnarray}
  where $k=1-2(r-q)/\sigma^2$. Now, the down-and-in option price becomes
  \begin{equation}
    V_{DI} = e^{-r\tau}\int_0^{\infty} \hat{f}(K)p\left(t, S_t;T,K\right)dK,
  \end{equation}
  which means that the down-and-in option with final payoff $f(S)$ at maturity can be statically replicated with
  a European contract with terminal payoff \cite{CarrChou}
  \begin{equation}
    \hat{f}(S) = \left[f(S) + \left(\frac{S}{H}\right)^kf\left(\frac{H^2}{S}\right)\right]\mathcal{I}_{S<H}
  \end{equation}
  at maturity. We can use the exact result from the previous section to replicate this European payoff with vanilla options.

  For down-and-out options, the replication strategy can be obtained from the difference between a
  vanilla option and a down-and-in option with the same barrier
  as the down-and-out option. The is due to the fact that a vanilla option can be decomposed into a 
  down-and-out and a down-and-in options with same features.

\section{Variance swap}

  A variance swap is a forward contract on future realized variance. At maturity, the floating leg pays
  \begin{equation}
    A\times\frac{N}{n}\sum_{i=1}^{n}\log^2\left(\frac{S_i}{S_{i-1}}\right),
  \end{equation}
  where $A$ is the notional amount of the variance swap, $N$ is an annualization factor and is usually set to
  252, $n$ is the number of monitor dates, and $S$ is the stock price at market close. The fixed leg payment is
  so determined that the variance swap has zero value at inception. In this section, we will discuss the valuation
  and replication of the variance swap, for both continuously and discretely monitored cases. Let us start with
  a very general result.

  \subsection{Profit and loss of a delta-hedged position}

    Consider the price of a European payoff $\phi(S)$ maturing at $T$, $V_{BS}(t,S_t;\sigma_{BS})$, where a constant volatility
    is assumed throughout the life time of the contract.
    For the discounted price, $f(t)=e^{-rt}V_{BS}(t,S_t;\sigma_{BS})$, we have
    \begin{equation}
      f(T)=f(0)+\int_0^Tdf(t)=f(0)+\int_0^Te^{-rt}\left[\left(-rV_{BS}+\frac{\partial V_{BS}}{\partial t}\right)dt+\frac{\partial V_{BS}}{\partial S}dS_t
              +\frac{1}{2}\frac{\partial^2 V_{BS}}{\partial S^2}\left(dS_t\right)^2\right].
    \end{equation}
    Assume that the underlying follows a geometric Brownian motion,
    \begin{equation}
      dS_t = \left(r-q\right)S_tdt + \sigma_tS_tdW_t,
    \end{equation}
    then
    \begin{eqnarray}
      f(T)&=&f(0)+\int_0^Te^{-rt}\left(-rV_{BS}+\frac{\partial V_{BS}}{\partial t}+\left(r-q\right)S_t\frac{\partial V_{BS}}{\partial S}
                                +\frac{1}{2}\sigma_t^2S_t^2\frac{\partial^2 V_{BS}}{\partial S^2}\right)dt\nonumber\\
          &&\quad\quad\quad+\int_0^Te^{-rt}\sigma_tS_t\frac{\partial V_{BS}}{\partial S}dW_t.
    \end{eqnarray}
    Notice that  $V(t,S_t;\sigma_{BS})$ satisfies the Black-Scholes PDE,
    \begin{equation}
      \frac{\partial V_{BS}}{\partial t}+\left(r-q\right)S\frac{\partial V_{BS}}{\partial S}+\frac{1}{2}\sigma_{BS}^2S^2\frac{\partial^2 V_{BS}}{\partial S^2}
      =rV_{BS},
    \end{equation}
    and finally, we have
    \begin{equation}
      f(T)=f(0)+\int_0^Te^{-rt}\frac{S_t^2}{2}\frac{\partial^2 V_{BS}}{\partial S^2}\left(\sigma_t^2-\sigma_{BS}^2\right)dt+\int_0^Te^{-rt}\sigma_tS_t\frac{\partial V_{BS}}{\partial S}dW_t.
      \label{average}
    \end{equation}
    The expected value of the hedged position is then given by
    \begin{equation}
      E\left[f(T)\right]=f(0)+E\left[\int_0^Te^{-rt}\frac{S_t^2}{2}\frac{\partial^2 V_{BS}}{\partial S^2}\left(\sigma_t^2-\sigma_{BS}^2\right)dt\right].
    \end{equation}
    From this, it is clear that, a delta-hedged option is sensitive to the difference between realized and implied variances.
    However, if one wants to trade
    the realized variance, the delta-hedged option is not a good choice, since the dollar gamma is peaked at the strike price.
    It can be shown, for a portfolio
    of options weighted by their inversed strike squared, $1/K^2$, the portfolio will be insensitive to the stock price move.

    As a side result, notice that, in an arbitrage free world, we have
    \begin{equation}
      E\left[f(T)\right]=f(0),
    \end{equation}
    which leads to
    \begin{equation}
      \sigma_{BS}^2=\frac{\displaystyle E\left[\int_0^Te^{-rt}\sigma_t^2S_t^2\frac{\partial^2 V_{BS}}{\partial S^2}dt\right]}
                         {\displaystyle E\left[\int_0^Te^{-rt}S_t^2\frac{\partial^2 V_{BS}}{\partial S^2}dt\right]}.
    \end{equation}
    This result implies that, the implied variance is the average of realized variance,
    weighted by dollar gamma. It can lead to interesting approximation for the implied
    volatility, for example, the so called ``most likely path" method. However, in practice, the approximation is rather crude.

  \subsection{Static replication of variance swap}

    Suppose we risk manage our position with zero implied volatility, {\it i.e.}, $\sigma_{BS}=0$ in Eq. (\ref{average}). Then,
    Eq. (\ref{average}) can be written as
    \begin{equation}
      \int_0^Te^{-rt}\frac{S_t^2}{2}\frac{\partial^2 V_{BS}}{\partial S^2}\sigma_t^2dt
      =f(T)-f(0)-\int_0^Te^{-rt}\sigma_tS_t\frac{\partial V_{BS}}{\partial S}dW_t.
    \end{equation}
    If we assume the underlying follows the stochastic process
    \begin{equation}
      dS_t=(r-q)S_tdt+\sigma_tS_tdW_t,
    \end{equation}  
    the last term becomes
    \begin{equation}
      \int_0^Te^{-rt}\frac{S_t^2}{2}\frac{\partial^2 V_{BS}}{\partial S^2}\sigma_t^2dt
      =f(T)-f(0)-\int_0^Te^{-rt}\frac{\partial V_{BS}}{\partial S}\left(dS_t-(r-q)S_tdt\right).
      \label{rep1}
    \end{equation}

    Now, choose
    \begin{equation}
      \phi(S) = 2\left[\log\left(\frac{F}{S}\right)+\frac{S}{F}-1\right],
    \end{equation}
    then its price at time $t$ with spot $S_t$ is
    \begin{equation}
      V_{BS}(t,S_t;\sigma_{BS} = 0) = 2e^{-r(T-t)}\left[\log\left(\frac{F}{S_t}\right)-(r-q)(T-t)+\frac{S_te^{(r-q)(T-t)}}{F}-1\right].
    \end{equation}
    The first and second order derivatives are
    \begin{equation}
      \frac{\partial V_{BS}}{\partial S}=2e^{-r(T-t)}\left[\frac{e^{(r-q)(T-t)}}{F}-\frac{1}{S_t}\right],
    \end{equation}
    and
    \begin{equation}
      \frac{\partial^2 V_{BS}}{\partial S^2}=\frac{2e^{-r(T-t)}}{S_t^2},
    \end{equation}
    repectively. With these derivatives, Eq. (\ref{rep1}) becomes
    \begin{eqnarray}
      \int_0^T\sigma_t^2dt &=& 2\left[\log\left(\frac{F}{S_T}\right)+\frac{S_T}{F}-1\right]\nonumber\\
                            && \ \ \ - 2\left[\log\left(\frac{F}{S_0}\right)-(r-q)T+\frac{S_0e^{(r-q)T}}{F}-1\right]\nonumber\\
                            && \ \ \ - 2\int_0^T\left[\frac{e^{(r-q)(T-t)}}{F}-\frac{1}{S_t}\right]\left(dS_t-(r-q)S_tdt\right).
    \end{eqnarray}
    Expand the last line, we have
    \begin{eqnarray}
      &&\int_0^T\left[\frac{e^{(r-q)(T-t)}}{F}-\frac{1}{S_t}\right]\left(dS_t-(r-q)S_tdt\right)\nonumber\\
      &=& \int_0^T\left[\frac{e^{(r-q)(T-t)}}{F}dS_t-\frac{dS_t}{S_t}-(r-q)S_t\frac{e^{(r-q)(T-t)}}{F}dt+(r-q)dt\right]\nonumber\\
      &=& \frac{S_T}{F} - \frac{S_0e^{(r-q)T}}{F} + \int_0^T\frac{dS_t}{S_t} + (r-q)T,
    \end{eqnarray}
    where we have performed integration by parts for the third term on the second line in the above equations.
    Finally, we have
    \begin{eqnarray}
      \int_0^T\sigma_t^2dt &=& 2\left[\log\left(\frac{F}{S_T}\right)+\frac{S_T}{F}-1\right]
                                     - 2\left[\log\left(\frac{F}{S_0}\right)+\frac{S_0}{F}-1\right]\nonumber\\
                            && \ \ \ - 2\int_0^T\left[\frac{1}{F}-\frac{1}{S_t}\right]dS_t\nonumber\\
                           &=& 2\left[\log\left(\frac{S_0}{S_T}\right)+\int_0^T\frac{dS_t}{S_t}\right].
    \end{eqnarray}
    Now, write $\log(S_0/S_T)$ as $\log(S_0/S_*)+\log(S_*/S_T)$, and notice that the fair strike of the variance swap is
    the risk neutral expectation of time averaged realized variance, we have
    \begin{eqnarray}
      K_{var} &=& E\left[\frac{1}{T}\int_0^T\sigma_t^2dt\right]\nonumber\\
              &=& \frac{2}{T}\Bigg[(r-q)T-\frac{S_0}{S_*}e^{(r-q)T}-1-\log\left(\frac{S_*}{S_0}\right)\nonumber\\
              &&\ \ \ \ +e^{rT}\int_0^{S_*}P(K)\frac{dK}{K^2}+e^{rT}\int_{S_*}^{+\infty}C(K)\frac{dK}{K^2}\Bigg].
    \end{eqnarray}


  \subsection{Discretely observed variance swap}

    \begin{equation}
      V = \frac{N}{n}\sum_{i=1}^n\log^2\left(\frac{S_i}{S_{i-1}}\right)
    \end{equation}

    \begin{equation}
      R_i=\frac{S_i-S_{i-1}}{S_i}
    \end{equation}

    \begin{equation}
      \log\left(\frac{S_i}{S_{i-1}}\right)=R_i-\frac{1}{2}R_i^2+\frac{1}{3}R_i^3
    \end{equation}

    \begin{equation}
      \log^2\left(\frac{S_i}{S_{i-1}}\right)=R_i^2-R_i^3
    \end{equation}

    \begin{equation}
      \log\left(\frac{S_i}{S_{i-1}}\right)=R_i-\frac{1}{2}\log^2\left(\frac{S_i}{S_{i-1}}\right)-\frac{1}{6}R_i^3
    \end{equation}

    \begin{equation}
      \log^2\left(\frac{S_i}{S_{i-1}}\right)=2R_i-2\log\left(\frac{S_i}{S_{i-1}}\right)-\frac{1}{3}R_i^3
    \end{equation}

    \begin{equation}
      \sum_{i=1}^n\log^2\left(\frac{S_i}{S_{i-1}}\right)=\sum_{i=1}^n\frac{2}{S_{i-1}}(S_i-S_{i-1})-2\log\left(\frac{S_n}{S_0}\right)-\frac{1}{3}\sum_{i=1}^nR_i^3
    \end{equation}



\section{Weighted variance swap}

  The method developed in the previous section can be applied to other types of variance swap. We will show a few examples in
  this section.

  \subsection{Corridor variance swap}

    Take
    \begin{equation}
      \phi(S) = 2\left[\log\left(\frac{F}{S^{\prime}}\right)+S\left(\frac{1}{F}-\frac{1}{S^{\prime}}\right)\right],
    \end{equation}
    where
    \begin{equation}
      S^{\prime} = S\mathcal{I}_{L\leq S \leq H}
    \end{equation}
    is floored at $L$ and capped at $H$, {\it i.e.}, the value of $S^{\prime}$ is restricted in the corridor $[L,H]$.

  \subsection{Gamma swap}


\section{Volatility swap}

  \begin{equation}
    F^{BS}\left(s, \sigma, \omega\right) = \int_0^{+\infty}F(sy,\omega)\frac{1}{y\sqrt{2\pi\sigma^2}}\exp\left(-\frac{\left(y+\sigma^2/2\right)^2}{2\sigma^2}\right)dy,
  \end{equation}

  \begin{equation}
    dS_t=\sqrt{1-\rho^2}\sigma_tS_tdW_{1t}+\rho\sigma_tS_tdW_{2t},
  \end{equation}

  \begin{equation}
    E_t\left[F(S_T)\right]=E_t\left[F^{BS}\left(S_tM_{t,T}(\rho),\bar\sigma_{t,T}\sqrt{1-\rho^2}\right)\right],
  \end{equation}

  \begin{equation}
    M_{t,T}(\rho) = \exp\left(-\frac{\rho^2}{2}\int_t^T\sigma_u^2du+\rho\int_t^T\sigma_udW_u\right),
  \end{equation}

  \begin{equation}
    \bar\sigma_{t,T}^2=\int_t^T\sigma_u^2du.
  \end{equation}

  \begin{eqnarray}
           E_t\left[F(S_T)\right]
    &=&E_t\left[F^{BS}\left(S_tM_{t,T}(\rho),\bar\sigma_{t,T}\sqrt{1-\rho^2}\right)\right]\nonumber\\
    &=&E_t\left[F^{BS}(S_t,\bar\sigma_{t,T})\right]
              + \rho S_tE_t\left[\frac{\partial F^{BS}(S_t,\bar\sigma_{t,T})}{\partial s}\int_t^T\sigma_udW_u\right],
  \end{eqnarray}

  \begin{equation}
    X_T-X_t=-\frac{1}{2}\left(\langle X \rangle_T - \langle X \rangle_t\right) + \int_t^T\sigma_udW_u,
  \end{equation}

  \begin{equation}
    X_T-X_t\sim N\left(-\frac{1}{2}\left(\langle X \rangle_T - \langle X \rangle_t\right), \langle X \rangle_T - \langle X \rangle_t\right)
  \end{equation}

  \begin{equation}
    E_t\left[\exp\left\{p\left(X_T-X_t\right)\right\}\right] = E_t\left[\exp\left\{\frac{1}{2}\left(p^2-p\right)\left(\langle X \rangle_T - \langle X \rangle_t\right)\right\}\right]
  \end{equation}

  $\lambda=(p^2-p)/2$

  $p=\frac{1}{2}\pm\sqrt{\frac{1}{4}+2\lambda}$


  \begin{equation}
    \sqrt{q} = \frac{1}{2\sqrt{\pi}}\int_0^{+\infty}\frac{1-2^{-zq}}{z^{3/2}}dz
  \end{equation}

  \begin{eqnarray}
                      E_t\left[\sqrt{\langle X \rangle_T - \langle X \rangle_t+q}\right]
    &=&\frac{1}{2\sqrt{\pi}}E_t\left[\int_0^{+\infty}\left(1-\exp\left\{-z\left(\langle X \rangle_T - \langle X \rangle_t+q\right)\right\}\right)\right]\frac{dz}{z^{3/2}}\nonumber\\
    &=&\frac{1}{2\sqrt{\pi}}\int_0^{+\infty}\sum_{\pm}\theta_{\pm}\left(1-e^{-zq}E_t\left[\exp\left\{p_{\pm}\left(X_T-X_t\right)\right\}\right]\right)\frac{dz}{z^{3/2}}\nonumber\\
    &=&\frac{1}{2\sqrt{\pi}}E_t\left[\int_0^{+\infty}\sum_{\pm}\theta_{\pm}\left(1-e^{-zq}\exp\left\{p_{\pm}\left(X_T-X_t\right)\right\}\right)\frac{dz}{z^{3/2}}\right]
  \end{eqnarray}

  $\theta_{\pm}=\frac{1}{2}\mp\frac{1}{2\sqrt{1-8z}}$

  \begin{equation}
    E_t\left[\langle X \rangle_T\right] = E_t\left[G_{SVS}\left(S_T,S_t,\langle X \rangle_t\right)\right]
  \end{equation}

  \begin{equation}
    G_{SVS}\left(S_T,S_t,q\right) = \frac{1}{2\sqrt{\pi}}\int_t^{+\infty}\left(\theta_+\left[1-e^{-zq}\left(\frac{S_T}{S_t}\right)^{p_+}\right]
                                                                            + \theta_-\left[1-e^{-zq}\left(\frac{S_T}{S_t}\right)^{p_-}\right]\right)\frac{dz}{z^{3/2}}
  \end{equation}

  \begin{equation}
    E_0\left[\langle X \rangle_T\right] = E_0\left[\psi(S_T)\right]
  \end{equation}

  $\psi(S)=\phi(\log(S/S_0))$

  \begin{equation}
    \phi(x)=\sqrt{\frac{\pi}{2}}e^{x/2}\left|xI_0\left(\frac{x}{2}\right)-xI_1\left(\frac{x}{2}\right)\right|
  \end{equation}


\begin{thebibliography}{99}
  \bibitem{CarrChou}
    Peter Carr and Andrew Chou, {\it Breaking Barriers}, Risk {\bf 10}(9), 139 (1997).

\end{thebibliography}


\end{document}