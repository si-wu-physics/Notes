\documentclass[12pt]{article}
\usepackage{amsfonts}
\usepackage{amsmath}
\usepackage[hscale=.8,vscale=.8]{geometry}
\usepackage{hyperref}

\usepackage{amsthm}
\usepackage{enumitem}

\newtheorem{theorem}{Theorem}[section]
\newtheorem{corollary}{Corollary}[theorem]
\newtheorem{lemma}[theorem]{Lemma}

\begin{document}

\title{Notes on Feller Condition}
\date{Dec. 32, 2999}

\maketitle

\section{Fokker-Planck Equation}

  Consider the following parabolic partial differential equation (PDE) \cite{Feller}
  \begin{equation}
    \frac{\partial u(t,x)}{\partial t} = -\frac{\partial}{\partial x}\left((bx+c)u(t,x)\right)
                                       + \frac{\partial^2}{\partial x^2}\left(axu(t,x)\right),
    \label{FP}
  \end{equation}
  with $x>0$. This can be viewed as the corresponding Fokker-Planck equation for the Cox-Ingersoll-Ross (CIR)
  process
  \begin{equation}
    dx_t=\kappa(\theta-x_t)dt + \sigma\sqrt{x_t}dW_t,
  \end{equation}
  with $a=\sigma^2/2$, $b=-\kappa$, and $c=\kappa\theta$.

  We want to find the solution of the PDE (\ref{FP}), with certain initial condition,
  \begin{equation}
    u(t,x) = \phi(x).
    \label{init0}
  \end{equation}
  In addition, we want to the solution is positive, $u(t,x)\ge 0$, and norm preserving or decreasing, that is
  \begin{equation}
    \int_0^{+\infty}u(t,x)dx \le \int_0^{+\infty}\phi(x)dx.
  \end{equation}
  In particular, we are
  interested in the fundamental solution of the PDE (\ref{FP}), {\it i.e.}, the initial condition is given
  by
  \begin{equation}
    u(0,x)=\delta(x-\xi),
    \label{init}
  \end{equation}
  where $\delta(x)$ is the Dirac delta function. To this end, introduce the Laplace transform of $u(t,x)$ as
  \begin{equation}
    v(t,\lambda)=\int_0^{+\infty}e^{-\lambda x}u(t,x)dx,
  \end{equation}
  for $\lambda > 0$. We denote the Laplace transform of the initial condition as
  \begin{equation}
    \pi(\lambda) = \int_0^{+\infty}e^{-\lambda x}\phi(x)dx,
  \end{equation}
  and for the fundamental solution, $\pi(\lambda) = e^{-\lambda \xi}$.
  Consider the Laplace transform of the right hand side of Eq. (\ref{FP}),
  \begin{eqnarray}
      && \int_0^{+\infty} e^{-\lambda x}\bigg[\big(axu(t,x)\big)_{xx}-\big((bx+c)u(t,x)\big)_x\bigg]dx \nonumber\\
    = && \left. e^{-\lambda x}\bigg[\big(axu(t,x)\big)_{x}-\big((bx+c)u(t,x)\big)\bigg]\right|_0^{+\infty}\nonumber\\
      && \quad\quad +\lambda \int_0^{+\infty} e^{-\lambda x}\bigg[\big(axu(t,x)\big)_{x}-\big((bx+c)u(t,x)\big)\bigg]dx \nonumber\\
    = && f(t) + \lambda(b-\lambda a)v_{\lambda} - c\lambda v,
  \end{eqnarray}
  where
  \begin{equation}
    f(t) = \lim_{x\rightarrow 0}\left[(bx+c)u-(axu)_x\right],
  \end{equation}
  is the flux at $x=0$ and generally cannot be arbitrarily specified.

  After the Laplace transform, Eq. (\ref{FP}) becomes
  \begin{equation}
    v_t + \lambda(\lambda a - b) v_{\lambda} = f(t) - c\lambda v.
    \label{FP2}
  \end{equation}
  This first order PDE can be solved by the method of characteristics.
  The characteristics can be determined by integrating
  \begin{equation}
    dt = \frac{d\lambda}{\lambda(a\lambda - b)},
  \end{equation}
  which leads to the following characteristic
  \begin{equation}
    \label{char}
    e^{-bt}\frac{a\lambda - b}{\lambda} = C_1,
  \end{equation}
  or equivalently,
  \begin{equation}
    \lambda(t) = -\frac{be^{-bt}}{C_1-ae^{-bt}}.
  \end{equation}
  Then, Eq. (\ref{FP2}) becomes an ordinary differential equation,
  \begin{equation}
    \frac{dv}{dt}-\frac{bce^{-bt}}{C_1-ae^{-bt}}v=f(t).
  \end{equation}
  It can be integrated,
  \begin{equation}
    v(t,\lambda) = \left|C_1-ae^{-bt}\right|^{c/a}\left\{C_2+\int_0^t\frac{f(\tau)}{\left|C_1-ae^{-b\tau}\right|^{c/a}}d\tau\right\}.
  \end{equation}
  To determine the unknown constants, assume $C_2=A(C_1)$, where $A(y)$ is an arbitrary function.
  For the initial condition $v(0,\lambda)=\pi(\lambda)$,
  we have
  \begin{equation}
    \pi(\lambda) = \left|\frac{b}{\lambda}\right|^{c/a}A\left(a-\frac{b}{\lambda}\right),
  \end{equation}
  which gives
  \begin{equation}
    A(y) = \left|a-y\right|^{-c/a}\pi\left(\frac{b}{a-y}\right).
  \end{equation}
  Now, we have
  \begin{equation}
    v(t,\lambda) = \left|C_1-ae^{-bt}\right|^{c/a}\left\{\left|a-C_1\right|^{-c/a}
                    \pi\left(\frac{b}{a-C_1}\right)+\int_0^t
                    \frac{f(\tau)}{\left|C_1-ae^{-b\tau}\right|^{c/a}}d\tau\right\}.
  \end{equation}
  Using Eq. (\ref{char}), the general solution becomes
  \begin{equation}
    v(t,\lambda)=\left(\frac{b}{\lambda a(e^{bt}-1)+b}\right)^{c/a}\pi\left(\frac{\lambda be^{bt}}{\lambda a(e^{bt}-1)+b}\right)
                    +\int_0^tf(\tau)\left(\frac{b}{\lambda a(e^{b(t-\tau)}-1)+b}\right)^{c/a}d\tau.
    \label{FP4}
  \end{equation}

\section{Some exact results}

  In this section, we are going to gather some exact results regarding the general solution (\ref{FP4}) without proof.
  The implication of these results will be discussed.

  \begin{lemma}
    If $c\le 0$, then $v(t,\lambda)$ in Eq. (\ref{FP4}) is the Laplace transform of a solution of $u(t,x)$ only if $f(t)$
    satisfies
    \begin{equation}
      \label{current}
      \pi\left(\frac{b}{a(1-e^{-bt})}\right)
                    +\int_0^tf(\tau)\left(\frac{e^{bt}-1}{e^{b(t-\tau)}-1}\right)^{c/a}d\tau = 0.
    \end{equation}
  \end{lemma}
  This is due to the fact that $u(t,x)$ is integrable near $x=0$, which is equivalent to $v(t,\lambda)\rightarrow 0$ as
  $\lambda\rightarrow +\infty$. Then, there is only one solution for the original Fokker-Planck equation (\ref{FP}) with
  arbitrary initial condition (\ref{init0}), with the boundary condition prescribed by (\ref{current}).

  \begin{lemma}
    If $c>0$ and $f(t)$ is continuous for $t\ge 0$, then $v(t,\lambda)$ in Eq. (\ref{FP4}) is the Laplace transform of a solution of $u(t,x)$ with initial valules $\phi(x)$.
    The solution is positive preserving at least whenever $f(t)\ge 0$.
  \end{lemma}

  \begin{lemma}
    If $c\ge 0$ and $\phi(x)$ is non-negative, then Eq. (\ref{FP4})
    with $f(t)\equiv 0$ defines a non-negative solution $u(t,x)$ with
    initial values $\phi(x)$. For this solution,
    \begin{equation*}
      \lim_{x\rightarrow 0}u(t,x)=
      \begin{cases}
        \infty & {\rm if} \quad 0<c<a, \\
        0      & {\rm if} \quad c>a,   \\
        \frac{b}{a(e^{bt}-1)}\pi\left(\frac{be^{bt}}{a(e^{bt}-1)}\right) & {\rm if} \quad c=a.
      \end{cases}
    \end{equation*}
  \end{lemma}

  \begin{lemma}
    If $c>a$ and $f(t)\ge 0$ but $f(t)$ does not vanish identically, then solution $u(t,x)$ is norm increasing. If $f(t_0)<0$, the
    $u(t_0,x)<0$.
  \end{lemma}

  From the above results, we can draw the following conclusions:
  \begin{enumerate}[noitemsep]
    \item If $c\le 0$, there is one solution for each initial condition $\phi(x)$. The solution is positive and norm decreasing, and is
          absorbing at the $x=0$ boundary;

    \item If $0<c<a$, there is one positive and norm preserving solution, with vanishing flux at the $x=0$ boundary, {\it i.e.}, the reflecting
          boundary condition. There are infinitely many positive and norm decreasing solutions, with only one solution which is finite
          at the $x=0$ boundary, {\it i.e.}, absorbing boundary condition;

    \item If $c\ge a$, the exists one positive and norm preserving solution, with vanishing value and flux at the $x=0$ boundary.
          No boundary condition can be specified at the $x=0$ boundary, and the boundary is unattainable.
  \end{enumerate}

\section{Fundamental solutions}

  In the following, we will consider the absorbing and reflection boundary conditions and find the explicit form of the
  fundamental solutions to the Fokker-Planck equation (\ref{FP}) with initial condition (\ref{init}).

  \subsection{Absorbing boundary condition}

    If $c\le a$, we can impose the absorbing boundary condition, (\ref{current}).
    For the fundamental solution, we have $\pi(\lambda)=e^{-\lambda\xi}$, and Eq. (\ref{current}) becomes
    \begin{equation}
      \int_0^tf(\tau)\left(\frac{e^{bt}-1}{e^{b(t-\tau)}-1}\right)^{c/a}d\tau = -\exp\left(-\frac{b\xi}{a(1-e^{-bt})}\right).
      \label{current2}
    \end{equation}
    Define
    \begin{equation}
      \frac{1}{z}=1-e^{-bt}, \quad\quad \frac{1}{\zeta}=1-e^{-b\tau},
    \end{equation}
    Eq. (\ref{current2}) becomes
    \begin{equation}
      \int_z^{+\infty}f(\tau)\left(\frac{\zeta}{\zeta-z}\right)^{c/a}\frac{d\zeta}{b\zeta(\zeta-1)}=-\exp\left(-\frac{b\xi z}{a}\right).
      \label{current3}
    \end{equation}
    Let
    \begin{equation}
      g(\zeta) = f(\tau)\frac{\zeta^{c/a}}{b\zeta(\zeta-1)},
    \end{equation}
    and let $g(\zeta)$ have the following functional form
    \begin{equation}
      g(\zeta) = Be^{-A\zeta},
    \end{equation}
    then the right hand side of  Eq. (\ref{current3}) becomes
    \begin{equation}
      B\int_z^{+\infty}e^{-A\zeta}(\zeta-z)^{-c/a}d\zeta = \frac{\displaystyle B\Gamma\left(1-\frac{c}{a}\right)}{\displaystyle A^{1-c/a}}e^{-Az}.
    \end{equation}
    Matching terms, we can see that
    \begin{equation}
      A = \frac{b\xi}{a}, \quad\quad B = -\frac{1}{\displaystyle \Gamma\left(1-\frac{c}{a}\right)}\left(\frac{b\xi}{a}\right)^{1-c/a}.
    \end{equation}
    Therefore,
    \begin{equation}
      f(t)=-\frac{b}{\displaystyle \Gamma\left(1-\frac{c}{a}\right)}\frac{e^{-bt}}{1-e^{-bt}}\left(\frac{b\xi}{a(1-e^{-bt})}\right)^{1-c/a}
            \exp\left(-\frac{b\xi}{a(1-e^{-bt})}\right).
    \end{equation}

    With some manipulation, the fundamental solution can be written in the following convenient form,
    \begin{equation}
      v(t,\lambda) = \left(\frac{b}{\lambda a(e^{bt}-1)+b}\right)^{c/a}\exp\left(-\frac{\lambda b\xi e^{bt}}{\lambda a(e^{bt}-1)+b}\right)
                      \Gamma\left(1-\frac{c}{a};\frac{b\xi e^{bt}}{a(e^{bt}-1)}\frac{b}{\lambda a(e^{bt}-1)+b}\right),
      \label{solution1}
    \end{equation}
    where
    \begin{equation}
      \Gamma(n;z)=\frac{1}{\Gamma(n)}\int_0^ze^{-x}x^{n-1}dx
    \end{equation}
    is the incomplete Gamma function. From this, we can see that
    \begin{equation}
      \int_0^{+\infty}u(t,x)dx = \lim_{\lambda\rightarrow 0}v(t,\lambda) = \Gamma\left(1-\frac{c}{a};\frac{\lambda b\xi e^{bt}}{a(e^{bt}-1)}\right),
    \end{equation}
    which is strictly less than 1. This means that the $x=0$ boundary serves as an absorbing boundary, and once it is hit, the solution will stay there.

    We can also find the explicit form of the fundamental solution via inverse Laplace transform. Again, let
    \begin{equation}
      \label{sub}
      A = \frac{b\xi e^{bt}}{a(e^{bt}-1)}, \quad\quad \frac{1}{z} = \frac{b}{\lambda a(e^{bt}-1)+b},
    \end{equation}
    the incomplete Gamma function in Eq. (\ref{solution1}) can be expressed as
    \begin{eqnarray}
      \frac{1}{\displaystyle \Gamma\left(1-\frac{c}{a}\right)}\int_0^{A/z}e^{-x}x^{n-1}dx
      &=& \frac{1}{\displaystyle \Gamma\left(1-\frac{c}{a}\right)}\left(\frac{A}{z}\right)^{1-c/a}\int_0^1 e^{-Au/z}u^{-c/a}du \nonumber\\
      &=& \frac{1}{\displaystyle \Gamma\left(1-\frac{c}{a}\right)}\left(\frac{A}{z}\right)^{1-c/a}e^{-A/z}\int_0^1 e^{Av/z}(1-v)^{-c/a}dv.
    \end{eqnarray}
    Then,
    \begin{equation}
      v(t,\lambda) = \frac{e^{-A}A^{1-c/a}}{\displaystyle \Gamma\left(1-\frac{c}{a}\right)}\int_0^1 (1-v)^{-c/a}\frac{e^{Av/z}}{z}dv.
    \end{equation}
    The inverse Laplace transform can be applied to $e^{Av/z}/z$, which yields
    \begin{eqnarray}
      \int e^{\lambda x}\frac{e^{Av/z}}{z}d\lambda &=& \frac{b}{a(e^{bt}-1)}\exp\left(-\frac{bx}{a(e^{bt}-1)}\right)
                                                          \int \exp\left(\frac{bxz}{a(e^{bt}-1)}\right)\frac{e^{Av/z}}{z}dz \nonumber\\
                                                   &=& \frac{b}{a(e^{bt}-1)}\exp\left(-\frac{bx}{a(e^{bt}-1)}\right)I_0\left(2\left(\frac{Avbx}{a(e^{bt}-1)}\right)^{1/2}\right) \nonumber\\
                                                   &=& \frac{b}{a(e^{bt}-1)}\exp\left(-\frac{bx}{a(e^{bt}-1)}\right)I_0\left(\frac{2b}{a(e^{bt}-1)}\left(e^{bt}\xi v x\right)^{1/2}\right),
    \end{eqnarray}
    where we have used the fact that the inverse Laplace transform of $e^{Av/z}/z$ is $I_0(2\sqrt{Avx})$, where $I_{\nu}(x)$ is the modified Bessel
    function of order $\nu$. Finally, we have
    \begin{eqnarray}
      \label{solution4}
      u(t,x) &=& \frac{1}{\displaystyle \Gamma\left(1-\frac{c}{a}\right)}\frac{b}{a(e^{bt}-1)}\left(\frac{b\xi}{a(1-e^{-bt})}\right)^{1-c/a}
                 \exp\left(-\frac{b(x+\xi e^{bt})}{a(e^{bt}-1)}\right) \nonumber\\
              && \times\int_0^1(1-v)^{-c/a}I_0\left(\frac{2b}{a(1-e^{-bt})}\left(e^{-bt}\xi v x\right)^{1/2}\right)dv.
    \end{eqnarray}
    Using the series expansion of the modified Bessel function,
    \begin{equation}
      I_{\nu}(x) = \sum_{r=0}^{\infty}\frac{1}{r!\Gamma(r+1+\nu)}\left(\frac{x}{2}\right)^{2r+\nu},
    \end{equation}
    the integral in (\ref{solution4}) can be written as
    \begin{eqnarray}
       && \int_0^1(1-v)^{-c/a}I_0\left(\frac{2b}{a(1-e^{-bt})}\left(e^{-bt}\xi v x\right)^{1/2}\right)dv \nonumber\\
      &=& \sum_{r=0}^{\infty}\frac{1}{r!\Gamma(r+1)}\left(\frac{2b}{a(1-e^{-bt})}\left(e^{-bt}\xi x\right)^{1/2}\right)^{2r}\int_0^1(1-v)^{-c/a}v^rdv \nonumber\\
      &=& \Gamma\left(1-\frac{c}{a}\right)\sum_{r=0}^{\infty}\frac{1}{\displaystyle r!\Gamma\left(r-\frac{c}{a}+2\right)}
              \left(\frac{2b}{a(1-e^{-bt})}\left(e^{-bt}\xi x\right)^{1/2}\right)^{2r} \nonumber\\
      &=& \Gamma\left(1-\frac{c}{a}\right)\left(\frac{b}{a(1-e^{-bt})}\right)^{-1+c/a}\left(e^{-bt}\xi x\right)^{\frac{c-a}{2a}}I_{1-\frac{c}{a}}
              \left(\frac{2b}{a(1-e^{-bt})}\left(e^{-bt}\xi x\right)^{1/2}\right).
    \end{eqnarray}
    Then,
    \begin{equation}
      u(t,x) = \frac{b}{a(e^{bt}-1)}\exp\left(-\frac{b(x+\xi e^{bt})}{a(e^{bt}-1)}\right)
                \left(e^{-bt}\frac{x}{\xi}\right)^{\frac{c-a}{2a}}I_{1-\frac{c}{a}}\left(\frac{2b}{a(1-e^{-bt})}\left(e^{-bt}\xi x\right)^{1/2}\right).
    \end{equation}

  \subsection{Reflection boundary condition}

    If $0<c\le a$,
    we can also impose the reflecting boundary condition on Eq. (\ref{FP4}), $f(t)\equiv 0$, which will render a positive
    and norm preserving solution. To find the 
    explicit form of the fundamental solution, we can apply the inverse Laplace transform on Eq. (\ref{FP4}), with $f(t)\equiv 0$.
    Use the same substitution as in Eq. (\ref{sub}),
    we have
    \begin{equation}
      v(t,\lambda) = e^{-A}z^{-c/a}\exp\left(\frac{A}{z}\right).
    \end{equation}
    The same manipulation of the inverse Laplace transform as in last section leads to
    \begin{eqnarray}
      &&\int e^{\lambda x} \frac{e^{A/z}}{z^{c/a}}d\lambda\nonumber\\
      &=& \frac{b}{a(e^{bt}-1)}\exp\left(-\frac{bx}{a(e^{bt}-1)}\right)\int \exp\left(\frac{bxz}{a(e^{bt}-1)}\right)\frac{e^{A/z}}{z^{c/a}}dz \nonumber\\
      &=& \frac{b}{a(e^{bt}-1)}\exp\left(-\frac{bx}{a(e^{bt}-1)}\right)\left(\frac{bx}{Aa(e^{bt}-1)}\right)^{\frac{c-a}{2a}}I_{-1+\frac{c}{a}}\left(2\left(\frac{Abx}{a(e^{bt}-1)}\right)^{1/2}\right) \nonumber\\
      &=& \frac{b}{a(e^{bt}-1)}\exp\left(-\frac{bx}{a(e^{bt}-1)}\right)\left(e^{-bt}\frac{x}{\xi}\right)^{\frac{c-a}{2a}}I_{-1+\frac{c}{a}}\left(\frac{2b}{a(1-e^{-bt})}\left(e^{-bt}\xi x\right)^{1/2}\right),
    \end{eqnarray}
    where we have use the fact that the inverse Laplace transform of $e^{A/z}/z^{1+\nu}$ is $(x/A)^{\nu/2}I_{\nu}(2\sqrt{Ax})$, 
    and the fundamental solution is then given by
    \begin{equation}
      u(t,x)=\frac{b}{a(e^{bt}-1)}\exp\left(-\frac{b(x+\xi e^{bt})}{a(e^{bt}-1)}\right)\left(e^{-bt}\frac{x}{\xi}\right)^{\frac{c-a}{2a}}I_{-1+\frac{c}{a}}\left(\frac{2b}{a(1-e^{-bt})}\left(e^{-bt}\xi x\right)^{1/2}\right).
    \end{equation}



\section{CEV model}

  For the underlying asset price, the CEV model assumes the following stochastic process,
  \begin{equation}
    dS_t=\sigma S_t^{\beta}dW_t,
  \end{equation}
  where $\sigma>0$ is the volatility, and $\beta>0$. Define
  \begin{equation}
    X_t=\frac{S_t^{2(1-\beta)}}{\sigma^2(1-\beta)^2},
  \end{equation}
  and apply the Ito's lemma, we have
  \begin{equation}
    dX_t = \delta dt + 2\sqrt{X_t}dW_t,
  \end{equation}
  where
  \begin{equation}
    \delta = \frac{1-2\beta}{1-\beta}.
  \end{equation}
  Thus, the CEV model has been reduced to the problem that we have studied in the past few sections, with
  $a=2$, $b=0$, and $c=\delta$.

  The absorbing boundary condition can be imposed when $c\le a$, or $\delta\le 2$, which is equivalent to
  $0<\beta\le 1$. The transition density for the transformed state variable $X$ is given by
  \begin{equation}
    p(X_T,T;X_0)=\frac{1}{2T}\exp\left(-\frac{X_T+X_0}{2T}\right)\left(\frac{X_T}{X_0}\right)^{-\nu/2}
                    I_{\nu}\left(\frac{\sqrt{X_TX_0}}{T}\right),
  \end{equation}
  where
  \begin{equation}
    \nu=\frac{1}{2(1-\beta)}.
  \end{equation}

  The reflecting boundary condition can be imposed when $0<c\le a$, or $0<\delta\le 2$, which is equivalent to
  $0<\beta\le \frac{1}{2}$. The transition density for the transformed state variable $X$ is given by
  \begin{equation}
    p(X_T,T;X_0)=\frac{1}{2T}\exp\left(-\frac{X_T+X_0}{2T}\right)\left(\frac{X_T}{X_0}\right)^{-\nu/2}
                    I_{-\nu}\left(\frac{\sqrt{X_TX_0}}{T}\right).
  \end{equation}

\section{Simulation of the CIR process}


\begin{thebibliography}{99}
  \bibitem{Feller}
    W. Feller, {\it Two Singular Diffusion Problems}, Annals of Mathematics {\bf 54}, 173 (1951).

\end{thebibliography}


\end{document}