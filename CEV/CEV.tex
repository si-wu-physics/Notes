\documentclass[12pt]{article}
\usepackage{amsfonts}
\usepackage[hscale=.8,vscale=.8]{geometry}
\usepackage{hyperref}

\begin{document}

\title{Notes on CEV Model}
\date{Dec. 32, 2999}

\maketitle

\section{Fokker-Planck Equation}

  Assume the forward process follows the CEV process,
  \begin{equation}
    dS_t=\sigma S_t^{\beta}dW_t,
    \label{CEV}
  \end{equation}
  with $\beta < 1$, the corresponding Fokker-Planck equation is given by
  \begin{equation}
    -\frac{\partial p(0,s_0;t,s)}{\partial t}+\frac{1}{2}\sigma^2\frac{\partial^2}{\partial s^2}
    \left(s^{2\beta}p(0,s_0;t,s)\right) = 0,
    \label{FP}
  \end{equation}
  with the initial condition
  \begin{equation}
    p(0,s_0;0,s) = \delta(s-s_0),
    \label{initial}
  \end{equation}
  where $\delta(x)$ is the Dirac delta function. Since the CEV process is time homogeneous, we do not need
  to make the initial time explicit, and we will write $p(t,s)\equiv p(0,s_0;t,s)$.

  Define the Laplace transform of the forward transition density $p(t,s)$ as \cite{Lesniewski}
  \begin{equation}
    p(t,s)=\int_0^{+\infty}e^{-\lambda t}g(\lambda, s)d\lambda,
    \label{Laplace}
  \end{equation}
  the Fokker-Planck equation (\ref{FP}) becomes
  \begin{equation}
    \frac{\partial^2g(\lambda,s)}{\partial s^2}+\frac{4\beta}{s}\frac{\partial g(\lambda,s)}{\partial s}
    + \left(\frac{2\lambda}{\sigma^2 s^{2\beta}}+\frac{2\beta(2\beta-1)}{s^2}\right)g(\lambda, s) = 0.
    \label{FP2}
  \end{equation}

  Following \cite{Bessel}, if $u(x)$ satisfies the Bessel equation of order $\nu$,
  \begin{equation}
    u^{\prime\prime}+\frac{1}{x}u^{\prime}+\left(1-\frac{\nu^2}{x^2}\right)u=0,
  \end{equation}
  and let $y(x)=x^au(bx^c)$, then $y(x)$ satisfies
  \begin{equation}
    y^{\prime\prime}+\frac{1-2a}{x}y^{\prime}+\left(b^2c^2x^{2c-2}+\frac{a^2-\nu^2c^2}{x^2}\right)y=0.
  \end{equation}
  Comparing the coefficients with Eq. (\ref{FP2}), we have
  \begin{equation}
    a=\frac{1}{2}\left(1-4\beta\right),\quad\quad
    b=\frac{2\nu\sqrt{2\lambda}}{\sigma}=\frac{\sqrt{2\lambda}}{\sigma(1-\beta)},\quad\quad
    c=\frac{1}{2\nu}=1-\beta,\quad\quad
    \nu=\frac{1}{2(1-\beta)}.
  \end{equation}
  Then, the general solution to Eq. (\ref{FP2}) is given by
  \begin{equation}
    g(\lambda, s)=s^{(1-4\beta)/2}\left[A_{\nu}(\lambda)J_{\nu}\left(\frac{2\nu\sqrt{2\lambda}}{\sigma}s^{\frac{1}{2\nu}}\right)
                                       +B_{\nu}(\lambda)Y_{\nu}\left(\frac{2\nu\sqrt{2\lambda}}{\sigma}s^{\frac{1}{2\nu}}\right)\right],
    \label{solution}
  \end{equation}
  where $J_{\nu}(x)$ and $Y_{\nu}(x)$ are the Bessel functions of the first and second kind, respectively. To determine the
  coefficients $A_{\nu}(\lambda)$ and $B_{\nu}(\lambda)$, and subsequently the forward transition density $p(t,s)$, boundary condition
  at $s=0$ is needed. To this end, we can write down the asymptotic expansion of Eq. (\ref{solution}),
  \begin{eqnarray}
    s^{2\beta}g(\lambda,s)&\sim& \left(\left(A_{\nu}(\lambda)+B_{\nu}(\lambda)\cot(\nu\pi)\right)
                  \left(\frac{\nu\sqrt{2\lambda}}{\sigma}\right)^{\nu}s\right)\left[\frac{1}{\Gamma(1+\nu)}
              -\frac{1}{\Gamma(2+\nu)}\left(\frac{\nu\sqrt{2\lambda}}{\sigma}\right)^{2}s^{\frac{1}{\nu}}\right]\nonumber\\
              &&-\frac{B_{\nu}(\lambda)}{\sin(\nu\pi)}\left(\frac{\nu\sqrt{2\lambda}}{\sigma}\right)^{-\nu}
              \left[\frac{1}{\Gamma(1-\nu)}
              -\frac{1}{\Gamma(2-\nu)}\left(\frac{\nu\sqrt{2\lambda}}{\sigma}\right)^{2}s^{\frac{1}{\nu}}\right].
  \end{eqnarray}
  Here, we have used the series representation of the Bessel function of the first and second kind
  \begin{equation}
    J_{\nu}(x)=\sum_{m=0}^{+\infty}\frac{(-1)^m}{m!\Gamma(m+\nu+1)}\left(\frac{x}{2}\right)^{2m+\nu},
  \end{equation}
  and
  \begin{equation}
    Y_{\nu}(x)=\frac{J_{\nu}(x)\cos(\nu\pi)-J_{-\nu}(x)}{\sin(\nu\pi)}.
  \end{equation}

  The asysmptotics of $p(t,s)$ for small $s$ have the form of $p_A(t,s)\sim s^{1-2\beta}$ and $p_R(t,s)\sim s^{-2\beta}$.
  The latter exists only $\beta < 1/2$. Otherwise, the norm will diverge around 0.
  Consider the Fokker-Planck equation (\ref{FP}), it can be expressed in the following current conservation form,
  \begin{equation}
    \frac{\partial p(t,s)}{\partial t} + \frac{\partial j(t,s)}{\partial s} = 0,
  \end{equation}
  with the current given by
  \begin{equation}
    j(t,s) = -\frac{1}{2}\sigma^2\frac{\partial }{\partial s}\left(s^{2\beta}p(t,s)\right).
  \end{equation}
  With the asymptotics of $p(t,s)$, we can see that the current is a constant for $p_A(t,s)$, and vanishes for $p_R(t,s)$.
  These correspond to an absorbing boundary condition for $p_A(t,s)$, and a reflecting boundary condition for $p_R(t,s)$.

  Now, consider an arbitrary function $f(s)$. From the Fokker-Planck equation (\ref{FP}), we
  have
  \begin{equation}
    \frac{\partial}{\partial t}\left(\int f(s)p(t,s)ds\right) = \frac{1}{2}\sigma^2\int f(s)\frac{\partial^2}{\partial s^2}
    \left(s^{2\beta}p(t,s)\right)ds.
  \end{equation}
  Integrating the right hand side by parts twice, we have
  \begin{eqnarray}
    \frac{\partial}{\partial t}\left(\int f(s)p(t,s)ds\right) &=& \frac{1}{2}\sigma^2\int f^{\prime\prime}(s)s^{2\beta}p(t,s)ds\nonumber \\
          && + \left . \frac{1}{2}\sigma^2\left[f^{\prime}(s)s^{2\beta}p(t,s) - f(s)\frac{\partial}{\partial s}
                \left(s^{2\beta}p(t,s)\right)\right]\right|_{s=0}.
    \label{conservation}
  \end{eqnarray}
  Let $f(s)\equiv 1$, it can be seen that the right hand side of Eq. (\ref{conservation}) does not vanish for absorbing
  boundary condition, and is zero for reflecting boundary condition. Therefore, the total probability is not conserved for
  the absorbing boundary condition, and is conserved for the reflecting boundary condition.

  Now consider $f(s)\equiv s$. The right hand side of Eq. (\ref{conservation}) will vanish for absorbing
  boundary condition, implying that $S_t$ is a martingale under the absorbing boundary condition, but not for reflecting
  boundary condition.

  Finally, let us consider the call payoff, $f(s)=(s-K)^+$. Then, Eq. (\ref{conservation}) becomes
  \begin{equation}
    \frac{\partial}{\partial t}\left(\int f(s)p(t,s)ds\right) = \frac{1}{2}\sigma^2K^{2\beta}p(t,K),
  \end{equation}
  irrespective of the boundary condition, which gives the call option price as
  \begin{equation}
    C(t,S_t,T,K) = \int (s-K)^+p(T,s)ds = (S_t-K)^++\frac{1}{2}\sigma^2\int_t^T K^{2\beta}p(u,K)du.
    \label{TimeValue}
  \end{equation}
  However, for the put payoff, there will be an additional boundary term for the reflecting boundary condition.



\section{Absorbing boundary condition}

  The absorbing boundary condition implies that $B_{\nu}(\lambda)=0$, which leads to
  \begin{equation}
    g(\lambda, s)=s^{(1-4\beta)/2}A_{\nu}(\lambda)J_{\nu}\left(\frac{2\nu\sqrt{2\lambda}}{\sigma}s^{\frac{1}{2\nu}}\right).
  \end{equation}
  To determine the coefficient $A_{\nu}(\lambda)$, initial condition will be needed. From Eq. (\ref{Laplace}), the transition
  density is given by
  \begin{equation}
    p_A(t,s)=\int_0^{+\infty}e^{-\lambda t}g(\lambda,s)d\lambda
          =s^{(1-4\beta)/2}\int_0^{+\infty}e^{-\lambda t}
          A_{\nu}(\lambda)J_{\nu}\left(\frac{2\nu\sqrt{2\lambda}}{\sigma}s^{\frac{1}{2\nu}}\right)d\lambda.
  \end{equation}
  Take into account of the initial condition (\ref{initial}), we have
  \begin{equation}
    \delta(s-s_0)=s^{(1-4\beta)/2}\int_0^{+\infty}
          A_{\nu}(\lambda)J_{\nu}\left(\frac{2\nu\sqrt{2\lambda}}{\sigma}s^{\frac{1}{2\nu}}\right)d\lambda.
  \end{equation}
  Now, multiply both sides by
  $$
    \sqrt{s}J_{\nu}\left(\frac{2\nu\sqrt{2\lambda}}{\sigma}s^{\frac{1}{2\nu}}\right),
  $$
  and integrate over $s$, we will have
  \begin{eqnarray}
      \sqrt{s_0}J_{\nu}\left(\frac{2\nu\sqrt{2\lambda}}{\sigma}s_0^{\frac{1}{2\nu}}\right)
    &=&  \int_0^{+\infty}d\lambda^{\prime} A_{\nu}(\lambda^{\prime})
         \int_0^{+\infty}ds s^{1-2\beta}
         J_{\nu}\left(\frac{2\nu\sqrt{2\lambda^{\prime}}}{\sigma}s^{\frac{1}{2\nu}}\right)
         J_{\nu}\left(\frac{2\nu\sqrt{2\lambda}}{\sigma}s^{\frac{1}{2\nu}}\right)\nonumber\\
    &=& 2\nu \int_0^{+\infty}d\lambda^{\prime} A_{\nu}(\lambda^{\prime})
         \int_0^{+\infty}du u
         J_{\nu}\left(\frac{2\nu\sqrt{2\lambda^{\prime}}}{\sigma}u\right)
         J_{\nu}\left(\frac{2\nu\sqrt{2\lambda}}{\sigma}u\right) \nonumber\\
    &=& 2\nu \int_0^{+\infty}d\lambda^{\prime} A_{\nu}(\lambda^{\prime})
          \frac{\displaystyle \delta\left(\frac{2\nu\sqrt{2\lambda}}{\sigma} - \frac{2\nu\sqrt{2\lambda^{\prime}}}{\sigma}\right)}
            {\displaystyle \frac{2\nu\sqrt{2\lambda}}{\sigma}}\nonumber\\
    &=& \frac{\sigma^2}{2\nu}\int_0^{+\infty}d\lambda^{\prime} A_{\nu}(\lambda^{\prime})
          \delta(\lambda - \lambda^{\prime})\nonumber\\
    &=& \frac{\sigma^2}{2\nu}A_{\nu}(\lambda).
  \end{eqnarray}
  Here, we have used the Hankel transform
  \begin{equation}
    \int_0^{+\infty}rJ_{\nu}(kr)J_{\nu}(k^{\prime}r)=\frac{\delta(k-k^{\prime})}{k},
  \end{equation}
  and the property of delta function,
  \begin{equation}
    \delta\left(f(x)\right)=\frac{\delta(x)}{|f^{\prime}(0)|}.
  \end{equation}
  Now, we have
  $$
    A_{\nu}(\lambda)=\frac{2\nu}{\sigma^2}\sqrt{s_0}J_{\nu}\left(\frac{2\nu\sqrt{2\lambda}}{\sigma}s_0^{\frac{1}{2\nu}}\right),
  $$
  the transition density can be calculated,
  \begin{eqnarray}
    p_A(t,s) &=& \frac{2\nu}{\sigma^2}\left(s_0s^{1-4\beta}\right)^{1/2}\int_0^{+\infty}e^{-\lambda t}
                J_{\nu}\left(\frac{2\nu\sqrt{2\lambda}}{\sigma}s^{\frac{1}{2\nu}}\right)
                J_{\nu}\left(\frac{2\nu\sqrt{2\lambda}}{\sigma}s_0^{\frac{1}{2\nu}}\right)d\lambda\nonumber\\
           &=& \frac{4\nu}{\sigma^2}\left(s_0s^{1-4\beta}\right)^{1/2}\int_0^{+\infty}e^{-tu^2} u
                J_{\nu}\left(\frac{\sqrt{8}\nu}{\sigma}s^{\frac{1}{2\nu}}u\right)
                J_{\nu}\left(\frac{\sqrt{8}\nu}{\sigma}s_0^{\frac{1}{2\nu}}u\right)du\nonumber\\
           &=& \frac{2\nu}{\sigma^2t}\left(s_0s^{1-4\beta}\right)^{1/2}
               \exp\left(-\frac{2\nu^2}{\sigma^2t}\left(s_0^{\frac{1}{\nu}}+s^{\frac{1}{\nu}}\right)\right)
               I_{\nu}\left(\frac{4\nu^2}{\sigma^2t}\left(s_0s\right)^{\frac{1}{2\nu}}\right),
    \label{density1}
  \end{eqnarray}
  where we have used the result \cite{DLMF1}
  \begin{equation}
    \int_0^{+\infty}e^{-tu^2}uJ_{\nu}(au)J_{\nu}(bu)du=\frac{1}{2t}\exp\left(-\frac{a^2+b^2}{4t}\right)I_{\nu}\left(\frac{ab}{2t}\right),
  \end{equation}
  and $I_{\nu}(x)$ is the modified Bessel function of the first kind.

  Finally, the transion density is given by
  \begin{equation}
    p_A(t,s)=\frac{\left(s_0s^{1-4\beta}\right)^{1/2}}{(1-\beta)\sigma^2t}
    \exp\left(-\frac{s_0^{2(1-\beta)}+s^{2(1-\beta)}}{2(1-\beta)^2\sigma^2t}\right)
    I_{\nu}\left(\frac{\left(s_0s\right)^{1-\beta}}{(1-\beta)^2\sigma^2t}\right).
  \end{equation}



\section{Reflecting boundary condition}

  Consider the reflecting boundary condition. It can be shown, to impose this boundary condition, we must have
  \begin{equation}
    A_{\nu}(\lambda)+B_{\nu}(\lambda)\cot(\nu\pi)=0,
  \end{equation}
  and
  \begin{equation}
    \frac{1}{\nu}-1\geq 0,
  \end{equation}
  which is equivalent to
  \begin{equation}
    \beta \leq \frac{1}{2}.
  \end{equation}
  Now, we have
  \begin{equation}
    g(\lambda,x)=-s^{(1-4\beta)/2}\frac{B_{\nu}(\lambda)}{\sin(\nu\pi)}J_{-\nu}\left(\frac{2\nu\sqrt{2\lambda}}{\sigma}s^{\frac{1}{2\nu}}\right).
  \end{equation}
  Proceed in the same way as the absorbing boundary condition, the transition density is then given by
  \begin{equation}
    p_R(t,s)=\frac{\left(s_0s^{1-4\beta}\right)^{1/2}}{(1-\beta)\sigma^2t}
    \exp\left(-\frac{s_0^{2(1-\beta)}+s^{2(1-\beta)}}{2(1-\beta)^2\sigma^2t}\right)
    I_{-\nu}\left(\frac{\left(s_0s\right)^{1-\beta}}{(1-\beta)^2\sigma^2t}\right).
  \end{equation}


% \section{European option pricing}


\section{Integral representation of European call option price}

  To preclude arbitrage opportunity, we are going to consider the European call option price under the absorbing boundary condition.
  The price for a European call option with expiry $T$, forward $S$, and strike $K$ is given by
  \begin{equation}
    C_A(T,S,K) = (S-K)^+ + \frac{\sigma^2}{2}K^{2\beta}\int_0^Tp_A(0,S;\tau,K)d\tau,
    \label{callprice}
  \end{equation}
  with the density $p(0,S;t,K)$ given by Eq. (\ref{density1}). To this end, we can use the integral representation of the modified
  Bessel function \cite{DLMF2},
  \begin{equation}
    I_{\nu}(z) = \frac{1}{\pi}\int_0^{\pi}e^{z\cos\phi}\cos(\nu\phi)d\phi
                - \frac{\sin(\nu\pi)}{\pi}\int_0^{+\infty}e^{-z\cosh \psi - \nu \psi}d\psi
  \end{equation}
  Define
  \begin{equation}
    q_S=\frac{S^{1-\beta}}{1-\beta}, \quad q_K=\frac{K^{1-\beta}}{1-\beta},
  \end{equation}
  the integration term of Eq. (\ref{callprice}) becomes
  \begin{eqnarray}
    && \nu\sqrt{SK}\int_0^T\frac{d\tau}{\tau}\exp\left(-\frac{q_K^2+q_S^2}{2\sigma^2\tau}\right)
    \Bigg(\frac{1}{\pi}\int_0^{\pi}\exp\left(\frac{q_Kq_S\cos\phi}{\sigma^2\tau}\right)\cos(\nu\phi)d\phi\nonumber\\
    && \quad\quad\quad\quad\quad\quad\quad\quad\quad\quad\quad\quad\quad\quad\quad\quad
            - \frac{\sin(\nu\pi)}{\pi}\int_0^{+\infty}\exp\left(-\frac{q_Kq_S\cosh \psi}{\sigma^2\tau}\right)e^{-\nu \psi}d\psi\Bigg)\nonumber\\
    = && \nu\sqrt{SK}\int_0^{\sigma^2 T}\frac{d\tau}{\tau}\exp\left(-\frac{q_K^2+q_S^2}{2\tau}\right)
    \Bigg(\frac{1}{\pi}\int_0^{\pi}\exp\left(\frac{q_Kq_S\cos\phi}{\tau}\right)\cos(\nu\phi)d\phi\nonumber\\
    && \quad\quad\quad\quad\quad\quad\quad\quad\quad\quad\quad\quad\quad\quad\quad\quad
            - \frac{\sin(\nu\pi)}{\pi}\int_0^{+\infty}\exp\left(-\frac{q_Kq_S\cosh \psi}{\tau}\right)e^{-\nu \psi}d\psi\Bigg).
  \end{eqnarray}
  Now, consider the first integration of the above expression. Change the variable $y=q_Kq_S/\tau$, define
  $b=(q_K^2+q_S^2)/2q_Kq_S$, and integrate by parts, we have
  \begin{eqnarray}
    && \frac{\nu\sqrt{SK}}{\pi}\int_0^{\sigma^2 T}\frac{d\tau}{\tau}\exp\left(-\frac{q_K^2+q_S^2}{2\tau}\right)
       \int_0^{\pi}\exp\left(\frac{q_Kq_S\cos\phi}{\tau}\right)\cos(\nu\phi)d\phi\nonumber\\
    = && \frac{\sqrt{SK}}{\pi}\int_{\frac{q_Kq_S}{\sigma^2 T}}^{+\infty}\frac{dy}{y}
       \int_0^{\pi}e^{-(b-\cos\phi)y}d\sin(\nu\phi)\nonumber\\
    = && \frac{\sqrt{SK}}{\pi}\int_{\frac{q_Kq_S}{\sigma^2 T}}^{+\infty}dy
         \left[\left.\frac{\sin(\nu\phi)}{y}e^{-(b-\cos\phi)y}\right|_0^{\pi}
              +\int_0^{\pi}\sin\phi\sin(\nu\phi)e^{-(b-\cos\phi)y}d\phi\right]\nonumber\\
    = && \frac{\sqrt{SK}}{\pi}\int_{\frac{q_Kq_S}{\sigma^2 T}}^{+\infty}\frac{\sin(\nu\pi)}{y}e^{-(b+1)y}dy\nonumber\\
      && \quad\quad\quad\quad+ \frac{\sqrt{SK}}{\pi}\int_0^{\pi}\frac{\sin\phi\sin(\nu\phi)}{b-\cos\phi}
       \exp\left(-\frac{q_Kq_S}{\sigma^2T}\left(b-\cos\phi\right)\right)d\phi.
  \end{eqnarray}
  Similarly, the second integration can be manipulated to give
  \begin{eqnarray}
    && -\frac{\sqrt{SK}}{\pi}\int_{\frac{q_Kq_S}{\sigma^2 T}}^{+\infty}\frac{\sin(\nu\pi)}{y}e^{-(b+1)y}dy\nonumber\\
      && \quad\quad\quad\quad+ \frac{\sqrt{SK}\sin(\nu\pi)}{\pi}\int_0^{+\infty}\frac{e^{-\nu \psi}\sinh \psi}{b+\cosh \psi}
       \exp\left(-\frac{q_Kq_S}{\sigma^2T}\left(b+\cosh \psi\right)\right)d\psi.
  \end{eqnarray}
  Finally, the call option price is
  \begin{eqnarray}
    C_A(T,S,K) &=& (S-K)^+ \nonumber\\
             &+&\frac{\sqrt{SK}}{\pi}\Bigg(\int_0^{\pi}\frac{\sin\phi\sin(\nu\phi)}{b-\cos\phi}
                        \exp\left(-\frac{q_Kq_S}{\sigma^2T}\left(b-\cos\phi\right)\right)d\phi\nonumber\\
             &&\quad\quad\quad\quad\quad + \sin(\nu\pi)\int_0^{+\infty}\frac{e^{-\nu \psi}\sinh \psi}{b+\cosh \psi}
       \exp\left(-\frac{q_Kq_S}{\sigma^2T}\left(b+\cosh \psi\right)\right)d\psi\Bigg).
    \label{CEVAbsorbingCall}
  \end{eqnarray}

  For completeness, we also record the result for reflecting boundary condition as
  \begin{eqnarray}
    C_R(T,S,K) &=& (S-K)^+ \nonumber\\
             &+&\frac{\sqrt{SK}}{\pi}\Bigg(\int_0^{\pi}\frac{\sin\phi\sin(\nu\phi)}{b-\cos\phi}
                        \exp\left(-\frac{q_Kq_S}{\sigma^2T}\left(b-\cos\phi\right)\right)d\phi\nonumber\\
             &&\quad\quad\quad\quad\quad + \sin(\nu\pi)\int_0^{+\infty}\frac{e^{\nu \psi}\sinh \psi}{b+\cosh \psi}
       \exp\left(-\frac{q_Kq_S}{\sigma^2T}\left(b+\cosh \psi\right)\right)d\psi\Bigg).
  \end{eqnarray}

\section{Free boundary CEV}

  The CEV process (\ref{CEV}) can be modified to allow negative forward in the following way,
  \begin{equation}
    dS_t=\sigma \left|S_t\right|^{\beta}dW_t,
    \label{FreeBoundaryCEV}
  \end{equation}
  with $0<\beta<1/2$. The corresponding Fokker-Planck equation is then given by
  \begin{equation}
    -\frac{\partial p(t,s)}{\partial t}+\frac{1}{2}\sigma^2\frac{\partial^2}{\partial s^2}
    \left(\left|s\right|^{2\beta}p(t,s)\right) = 0,
    \label{FBCEVFP}
  \end{equation}
  with initial condition (\ref{initial}). The solution to the Fokker-Planck equation (\ref{FBCEVFP})
  can be constructed as
  \begin{equation}
    p(t,s)=\frac{1}{2}\bigg(p_R\left(t,\left|s\right|\right)+{\rm sgn}(s)p_A\left(t,\left|s\right|\right)\bigg),
  \end{equation}
  where $p_A(t,s)$ and $p_R(t,s)$ are the solution to the Fokker-Plack equation (\ref{FP}) under
  the absorbing and reflecting boundary conditions, repectively. Following the time value of the option (\ref{TimeValue}),
  it can be shown that the call option price in the free boundary CEV model is
  \begin{eqnarray}
    C(T,S,K) &=& (S-K)^+ \nonumber\\
             &+&\frac{\sqrt{\left|SK\right|}}{\pi}\Bigg({\bf 1}_{K\geq 0}\int_0^{\pi}\frac{\sin\phi\sin(\nu\phi)}{b-\cos\phi}
                        \exp\left(-\frac{\bar{q}\left(b-\cos\phi\right)}{\sigma^2T}\right)d\phi\nonumber\\
             &&\quad\quad\quad\quad\quad + \sin(\nu\pi)\int_0^{+\infty}
                \frac{d\psi}{b+\cosh \psi}
       \exp\left(-\frac{\bar{q}\left(b+\cosh \psi\right)}{\sigma^2T}\right)\nonumber\\
             &&\quad\quad\quad\quad\quad\quad\quad\quad\quad\quad \times
                  \bigg({\bf 1}_{K\geq 0}\cosh(\nu \psi)+{\bf 1}_{K< 0}\sinh(\nu \psi)\bigg)\sinh \psi\Bigg),
  \end{eqnarray}
  where
  \begin{equation}
    \bar{q}=q_Kq_S,\quad b=\frac{q_K^2+q_S^2}{2q_Kq_S},\quad q_K=\frac{|K|^{1-\beta}}{1-\beta},
    \quad q_S=\frac{|S|^{1-\beta}}{1-\beta}.
  \end{equation}



\begin{thebibliography}{99}
  \bibitem{Lesniewski}
    See \url{http://www.lesniewski.us/papers/working/NotesOnCEV.pdf}.

  \bibitem{Bessel}
    See \url{http://mathworld.wolfram.com/BesselDifferentialEquation.html}.

  \bibitem{DLMF1}
    See \url{http://dlmf.nist.gov/10.22.E67}.

  \bibitem{DLMF2}
    See \url{http://dlmf.nist.gov/10.32.E4}.
\end{thebibliography}


\end{document}
