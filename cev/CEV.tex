\documentclass[12pt]{article}
\usepackage{amsfonts}
\usepackage[hscale=.8,vscale=.8]{geometry}
\usepackage{hyperref}

\begin{document}

\title{Notes on CEV Model}
\date{Dec. 32, 2999}

\maketitle

\section{Fokker-Planck Equation}

  Assume the forward process follows the CEV process,
  \begin{equation}
    dS_t=\sigma S_t^{\beta}dW_t,
    \label{CEV}
  \end{equation}
  with $\beta < 1$, the corresponding Fokker-Planck equation is given by
  \begin{equation}
    -\frac{\partial p(0,s_0;t,s)}{\partial t}+\frac{1}{2}\sigma^2\frac{\partial^2}{\partial s^2}
    \left(s^{2\beta}p(0,s_0;t,s)\right) = 0,
    \label{FP}
  \end{equation}
  with the initial condition
  \begin{equation}
    p(0,s_0;0,s) = \delta(s-s_0),
    \label{initial}
  \end{equation}
  where $\delta(x)$ is the Dirac delta function. Since the CEV process is time homogeneous, we do not need
  to make the initial time explicit, and we will write $p(t,s)\equiv p(0,s_0;t,s)$.

  Define the Laplace transform of the forward transition density $p(t,s)$ as \cite{Lesniewski}
  \begin{equation}
    p(t,s)=\int_0^{+\infty}e^{-\lambda t}g(\lambda, s)d\lambda,
    \label{Laplace}
  \end{equation}
  the Fokker-Planck equation (\ref{FP}) becomes
  \begin{equation}
    \frac{\partial^2g(\lambda,s)}{\partial s^2}+\frac{4\beta}{s}\frac{\partial g(\lambda,s)}{\partial s}
    + \left(\frac{2\lambda}{\sigma^2 s^{2\beta}}+\frac{2\beta(2\beta-1)}{s^2}\right)g(\lambda, s) = 0.
    \label{FP2}
  \end{equation}

  Following \cite{Bessel}, if $u(x)$ satisfies the Bessel equation of order $\nu$,
  \begin{equation}
    u^{\prime\prime}+\frac{1}{x}u^{\prime}+\left(1-\frac{\nu^2}{x^2}\right)u=0,
  \end{equation}
  and let $y(x)=x^au(bx^c)$, then $y(x)$ satisfies
  \begin{equation}
    y^{\prime\prime}+\frac{1-2a}{x}y^{\prime}+\left(b^2c^2x^{2c-2}+\frac{a^2-\nu^2c^2}{x^2}\right)y=0.
  \end{equation}
  Comparing the coefficients with Eq. (\ref{FP2}), we have
  \begin{equation}
    a=\frac{1}{2}\left(1-4\beta\right),\quad\quad
    b=\frac{2\nu\sqrt{2\lambda}}{\sigma}=\frac{\sqrt{2\lambda}}{\sigma(1-\beta)},\quad\quad
    c=\frac{1}{2\nu}=1-\beta,\quad\quad
    \nu=\frac{1}{2(1-\beta)}.
  \end{equation}
  Then, the general solution to Eq. (\ref{FP2}) is given by
  \begin{equation}
    g(\lambda, s)=s^{(1-4\beta)/2}\left[A_{\nu}(\lambda)J_{\nu}\left(\frac{2\nu\sqrt{2\lambda}}{\sigma}s^{\frac{1}{2\nu}}\right)
                                       +B_{\nu}(\lambda)Y_{\nu}\left(\frac{2\nu\sqrt{2\lambda}}{\sigma}s^{\frac{1}{2\nu}}\right)\right],
    \label{solution}
  \end{equation}
  where $J_{\nu}(x)$ and $Y_{\nu}(x)$ are the Bessel functions of the first and second kind, respectively. To determine the
  coefficients $A_{\nu}(\lambda)$ and $B_{\nu}(\lambda)$, and subsequently the forward transition density $p(t,s)$, boundary condition
  at $s=0$ is needed. To this end, we can write down the asymptotic expansion of Eq. (\ref{solution}),
  \begin{eqnarray}
    s^{2\beta}g(\lambda,s)&\sim& \left(\left(A_{\nu}(\lambda)+B_{\nu}(\lambda)\cot(\nu\pi)\right)
                  \left(\frac{\nu\sqrt{2\lambda}}{\sigma}\right)^{\nu}s\right)\left[\frac{1}{\Gamma(1+\nu)}
              -\frac{1}{\Gamma(2+\nu)}\left(\frac{\nu\sqrt{2\lambda}}{\sigma}\right)^{2}s^{\frac{1}{\nu}}\right]\nonumber\\
              &&-\frac{B_{\nu}(\lambda)}{\sin(\nu\pi)}\left(\frac{\nu\sqrt{2\lambda}}{\sigma}\right)^{-\nu}
              \left[\frac{1}{\Gamma(1-\nu)}
              -\frac{1}{\Gamma(2-\nu)}\left(\frac{\nu\sqrt{2\lambda}}{\sigma}\right)^{2}s^{\frac{1}{\nu}}\right].
  \end{eqnarray}
  Here, we have used the series representation of the Bessel function of the first and second kind
  \begin{equation}
    J_{\nu}(x)=\sum_{m=0}^{+\infty}\frac{(-1)^m}{m!\Gamma(m+\nu+1)}\left(\frac{x}{2}\right)^{2m+\nu},
  \end{equation}
  and
  \begin{equation}
    Y_{\nu}(x)=\frac{J_{\nu}(x)\cos(\nu\pi)-J_{-\nu}(x)}{\sin(\nu\pi)}.
  \end{equation}

  The asysmptotics of $p(t,s)$ for small $s$ have the form of $p_A(t,s)\sim s^{1-2\beta}$ and $p_R(t,s)\sim s^{-2\beta}$.
  The latter exists only $\beta < 1/2$. Otherwise, the norm will diverge around 0.
  Consider the Fokker-Planck equation (\ref{FP}), it can be expressed in the following current conservation form,
  \begin{equation}
    \frac{\partial p(t,s)}{\partial t} + \frac{\partial j(t,s)}{\partial s} = 0,
  \end{equation}
  with the current given by
  \begin{equation}
    j(t,s) = -\frac{1}{2}\sigma^2\frac{\partial }{\partial s}\left(s^{2\beta}p(t,s)\right).
  \end{equation}
  With the asymptotics of $p(t,s)$, we can see that the current is a constant for $p_A(t,s)$, and vanishes for $p_R(t,s)$.
  These correspond to an absorbing boundary condition for $p_A(t,s)$, and a reflecting boundary condition for $p_R(t,s)$.

  Now, consider an arbitrary function $f(s)$. From the Fokker-Planck equation (\ref{FP}), we
  have
  \begin{equation}
    \frac{\partial}{\partial t}\left(\int f(s)p(t,s)ds\right) = \frac{1}{2}\sigma^2\int f(s)\frac{\partial^2}{\partial s^2}
    \left(s^{2\beta}p(t,s)\right)ds.
  \end{equation}
  Integrating the right hand side by parts twice, we have
  \begin{eqnarray}
    \frac{\partial}{\partial t}\left(\int f(s)p(t,s)ds\right) &=& \frac{1}{2}\sigma^2\int f^{\prime\prime}(s)s^{2\beta}p(t,s)ds\nonumber \\
          && + \left . \frac{1}{2}\sigma^2\left[f^{\prime}(s)s^{2\beta}p(t,s) - f(s)\frac{\partial}{\partial s}
                \left(s^{2\beta}p(t,s)\right)\right]\right|_{s=0}.
    \label{conservation}
  \end{eqnarray}
  Let $f(s)\equiv 1$, it can be seen that the right hand side of Eq. (\ref{conservation}) does not vanish for absorbing
  boundary condition, and is zero for reflecting boundary condition. Therefore, the total probability is not conserved for
  the absorbing boundary condition, and is conserved for the reflecting boundary condition.

  Now consider $f(s)\equiv s$. The right hand side of Eq. (\ref{conservation}) will vanish for absorbing
  boundary condition, implying that $S_t$ is a martingale under the absorbing boundary condition, but not for reflecting
  boundary condition.

  Finally, let us consider the call payoff, $f(s)=(s-K)^+$. Then, Eq. (\ref{conservation}) becomes
  \begin{equation}
    \frac{\partial}{\partial t}\left(\int f(s)p(t,s)ds\right) = \frac{1}{2}\sigma^2K^{2\beta}p(t,K),
  \end{equation}
  irrespective of the boundary condition, which gives the call option price as
  \begin{equation}
    C(t,S_t,T,K) = \int (s-K)^+p(T,s)ds = (S_t-K)^++\frac{1}{2}\sigma^2\int_t^T K^{2\beta}p(u,K)du.
    \label{TimeValue}
  \end{equation}
  However, for the put payoff, there will be an additional boundary term for the reflecting boundary condition.



\section{Absorbing boundary condition}

  The absorbing boundary condition implies that $B_{\nu}(\lambda)=0$, which leads to
  \begin{equation}
    g(\lambda, s)=s^{(1-4\beta)/2}A_{\nu}(\lambda)J_{\nu}\left(\frac{2\nu\sqrt{2\lambda}}{\sigma}s^{\frac{1}{2\nu}}\right).
  \end{equation}
  To determine the coefficient $A_{\nu}(\lambda)$, initial condition will be needed. From Eq. (\ref{Laplace}), the transition
  density is given by
  \begin{equation}
    p_A(t,s)=\int_0^{+\infty}e^{-\lambda t}g(\lambda,s)d\lambda
          =s^{(1-4\beta)/2}\int_0^{+\infty}e^{-\lambda t}
          A_{\nu}(\lambda)J_{\nu}\left(\frac{2\nu\sqrt{2\lambda}}{\sigma}s^{\frac{1}{2\nu}}\right)d\lambda.
  \end{equation}
  Take into account of the initial condition (\ref{initial}), we have
  \begin{equation}
    \delta(s-s_0)=s^{(1-4\beta)/2}\int_0^{+\infty}
          A_{\nu}(\lambda)J_{\nu}\left(\frac{2\nu\sqrt{2\lambda}}{\sigma}s^{\frac{1}{2\nu}}\right)d\lambda.
  \end{equation}
  Now, multiply both sides by
  $$
    \sqrt{s}J_{\nu}\left(\frac{2\nu\sqrt{2\lambda}}{\sigma}s^{\frac{1}{2\nu}}\right),
  $$
  and integrate over $s$, we will have
  \begin{eqnarray}
      \sqrt{s_0}J_{\nu}\left(\frac{2\nu\sqrt{2\lambda}}{\sigma}s_0^{\frac{1}{2\nu}}\right)
    &=&  \int_0^{+\infty}d\lambda^{\prime} A_{\nu}(\lambda^{\prime})
         \int_0^{+\infty}ds s^{1-2\beta}
         J_{\nu}\left(\frac{2\nu\sqrt{2\lambda^{\prime}}}{\sigma}s^{\frac{1}{2\nu}}\right)
         J_{\nu}\left(\frac{2\nu\sqrt{2\lambda}}{\sigma}s^{\frac{1}{2\nu}}\right)\nonumber\\
    &=& 2\nu \int_0^{+\infty}d\lambda^{\prime} A_{\nu}(\lambda^{\prime})
         \int_0^{+\infty}du u
         J_{\nu}\left(\frac{2\nu\sqrt{2\lambda^{\prime}}}{\sigma}u\right)
         J_{\nu}\left(\frac{2\nu\sqrt{2\lambda}}{\sigma}u\right) \nonumber\\
    &=& 2\nu \int_0^{+\infty}d\lambda^{\prime} A_{\nu}(\lambda^{\prime})
          \frac{\displaystyle \delta\left(\frac{2\nu\sqrt{2\lambda}}{\sigma} - \frac{2\nu\sqrt{2\lambda^{\prime}}}{\sigma}\right)}
            {\displaystyle \frac{2\nu\sqrt{2\lambda}}{\sigma}}\nonumber\\
    &=& \frac{\sigma^2}{2\nu}\int_0^{+\infty}d\lambda^{\prime} A_{\nu}(\lambda^{\prime})
          \delta(\lambda - \lambda^{\prime})\nonumber\\
    &=& \frac{\sigma^2}{2\nu}A_{\nu}(\lambda).
  \end{eqnarray}
  Here, we have used the Hankel transform
  \begin{equation}
    \int_0^{+\infty}rJ_{\nu}(kr)J_{\nu}(k^{\prime}r)=\frac{\delta(k-k^{\prime})}{k},
  \end{equation}
  and the property of delta function,
  \begin{equation}
    \delta\left(f(x)\right)=\frac{\delta(x)}{|f^{\prime}(0)|}.
  \end{equation}
  Now, we have
  $$
    A_{\nu}(\lambda)=\frac{2\nu}{\sigma^2}\sqrt{s_0}J_{\nu}\left(\frac{2\nu\sqrt{2\lambda}}{\sigma}s_0^{\frac{1}{2\nu}}\right),
  $$
  the transition density can be calculated,
  \begin{eqnarray}
    p_A(t,s) &=& \frac{2\nu}{\sigma^2}\left(s_0s^{1-4\beta}\right)^{1/2}\int_0^{+\infty}e^{-\lambda t}
                J_{\nu}\left(\frac{2\nu\sqrt{2\lambda}}{\sigma}s^{\frac{1}{2\nu}}\right)
                J_{\nu}\left(\frac{2\nu\sqrt{2\lambda}}{\sigma}s_0^{\frac{1}{2\nu}}\right)d\lambda\nonumber\\
           &=& \frac{4\nu}{\sigma^2}\left(s_0s^{1-4\beta}\right)^{1/2}\int_0^{+\infty}e^{-tu^2} u
                J_{\nu}\left(\frac{\sqrt{8}\nu}{\sigma}s^{\frac{1}{2\nu}}u\right)
                J_{\nu}\left(\frac{\sqrt{8}\nu}{\sigma}s_0^{\frac{1}{2\nu}}u\right)du\nonumber\\
           &=& \frac{2\nu}{\sigma^2t}\left(s_0s^{1-4\beta}\right)^{1/2}
               \exp\left(-\frac{2\nu^2}{\sigma^2t}\left(s_0^{\frac{1}{\nu}}+s^{\frac{1}{\nu}}\right)\right)
               I_{\nu}\left(\frac{4\nu^2}{\sigma^2t}\left(s_0s\right)^{\frac{1}{2\nu}}\right),
    \label{density1}
  \end{eqnarray}
  where we have used the result \cite{DLMF1}
  \begin{equation}
    \int_0^{+\infty}e^{-tu^2}uJ_{\nu}(au)J_{\nu}(bu)du=\frac{1}{2t}\exp\left(-\frac{a^2+b^2}{4t}\right)I_{\nu}\left(\frac{ab}{2t}\right),
  \end{equation}
  and $I_{\nu}(x)$ is the modified Bessel function of the first kind.

  Finally, the transition density is given by
  \begin{equation}
    p_A(t,s)=\frac{\left(s_0s^{1-4\beta}\right)^{1/2}}{(1-\beta)\sigma^2t}
    \exp\left(-\frac{s_0^{2(1-\beta)}+s^{2(1-\beta)}}{2(1-\beta)^2\sigma^2t}\right)
    I_{\nu}\left(\frac{\left(s_0s\right)^{1-\beta}}{(1-\beta)^2\sigma^2t}\right).
    \label{pA}
  \end{equation}



\section{Reflecting boundary condition}

  Consider the reflecting boundary condition. It can be shown, to impose this boundary condition, we must have
  \begin{equation}
    A_{\nu}(\lambda)+B_{\nu}(\lambda)\cot(\nu\pi)=0,
  \end{equation}
  and
  \begin{equation}
    \frac{1}{\nu}-1\geq 0,
  \end{equation}
  which is equivalent to
  \begin{equation}
    \beta \leq \frac{1}{2}.
  \end{equation}
  Now, we have
  \begin{equation}
    g(\lambda,x)=-s^{(1-4\beta)/2}\frac{B_{\nu}(\lambda)}{\sin(\nu\pi)}J_{-\nu}\left(\frac{2\nu\sqrt{2\lambda}}{\sigma}s^{\frac{1}{2\nu}}\right).
  \end{equation}
  Proceed in the same way as the absorbing boundary condition, the transition density is then given by
  \begin{equation}
    p_R(t,s)=\frac{\left(s_0s^{1-4\beta}\right)^{1/2}}{(1-\beta)\sigma^2t}
    \exp\left(-\frac{s_0^{2(1-\beta)}+s^{2(1-\beta)}}{2(1-\beta)^2\sigma^2t}\right)
    I_{-\nu}\left(\frac{\left(s_0s\right)^{1-\beta}}{(1-\beta)^2\sigma^2t}\right).
    \label{pR}
  \end{equation}

\section{Normalization}

  Given the transition densities, (\ref{pA}) and (\ref{pR}), the normalization can be obtained by a simple integration with respect to $s$.
  To see that the total density is preserved for the reflecting boundary condition, introduce the following variables,
  \begin{equation}
    \alpha = \frac{2\nu}{\sigma\sqrt{t}},\quad b = \alpha s_0^{{\frac{1}{2\nu}}}, \quad c = \alpha s^{{\frac{1}{2\nu}}}.
  \end{equation}
  Notice that
  \begin{equation}
    ds=\frac{2\nu}{\alpha^{2\nu}}c^{2\nu-1}dc,
  \end{equation}
  integration of (\ref{pR}) becomes
  \begin{eqnarray}
      \int_0^{+\infty}p_R(t,s)ds
    &=&  \frac{s_0^{1/2}}{(1-\beta)\sigma^2t}
         \exp\left(-\frac{s_0^{2(1-\beta)}}{2(1-\beta)^2\sigma^2t}\right) \nonumber\\
      && \quad\times \int_0^{+\infty}s^{1-4\beta}\exp\left(-\frac{s^{2(1-\beta)}}{2(1-\beta)^2\sigma^2t}\right)
         I_{-\nu}\left(\frac{\left(s_0s\right)^{1-\beta}}{(1-\beta)^2\sigma^2t}\right)ds \nonumber\\
    &=&  \frac{\alpha^2}{2\nu}\left(\frac{b}{\alpha}\right)^{\nu}e^{-b^2/2}\int_0^{+\infty}\left(\frac{c}{\alpha}\right)^{2-3\nu}
         e^{-c^2/2}I_{-\nu}\bigg(bc\bigg)\cdot \frac{2\nu}{\alpha^{2\nu}}c^{2\nu-1}dc \nonumber\\
    &=&  b^{\nu}e^{-b^2/2}\int_0^{+\infty}c^{1-\nu}
         e^{-c^2/2}I_{-\nu}\bigg(bc\bigg)dc.
  \end{eqnarray}
  From the identity,
  \begin{equation}
    \int_0^{+\infty}t^{\nu+1}e^{-p^2t^2}I_{\nu}\bigg(bt\bigg)dt = \frac{b^v}{\left(2p^2\right)^{\nu+1}}\exp\left(\frac{b^2}{4p^2}\right),
  \end{equation}
  it can be easily verified that the transition density under the reflecting boundary condition is indeed normalized to 1.

  Similarly, for (\ref{pA}), we have
  \begin{equation}
    \int_0^{+\infty}p_A(t,s)ds = b^{\nu}e^{-b^2/2}\int_0^{+\infty}c^{1-\nu}e^{-c^2/2}I_{\nu}\bigg(bc\bigg)dc.
  \end{equation}
  The only difference is the sign of the order of the modified Bessel function, which prevents us from applying the above identity.
  To this end, we can expand $I_{\nu}(x)$ into a series,
  \begin{equation}
    I_{\nu}\bigg(bc\bigg)=\sum_{k=0}^{\infty}\frac{1}{k!\Gamma\left(k+\nu+1\right)}\left(\frac{bc}{2}\right)^{2k+\nu},
  \end{equation}
  and the integral becomes
  \begin{eqnarray}
    \int_0^{+\infty}p_A(t,s)ds &=& b^{\nu}e^{-b^2/2}\sum_{k=0}^{\infty}\frac{1}{k!\Gamma\left(k+\nu+1\right)}\left(\frac{b}{2}\right)^{2k+\nu}
                                   \int_0^{+\infty}c^{2k+1}e^{-c^2/2}dc \nonumber\\
                               &=& e^{-b^2/2}\sum_{k=0}^{\infty}\frac{1}{\Gamma\left(k+\nu+1\right)}\left(\frac{b^2}{2}\right)^{k+\nu}\nonumber\\
                               &=& \frac{1}{\Gamma(\nu)}\gamma\left(\nu, \frac{b^2}{2}\right)\nonumber\\
                               &=& 1 - \frac{1}{\Gamma(\nu)}\Gamma\left(\nu, \frac{s_0^{2(1-\beta)}}{2(1-\beta)^2\sigma^2t}\right).
  \end{eqnarray}
  Here, the lower incomplete gamma function is defined as
  \begin{equation}
    \gamma(\nu,x) = \Gamma(\nu)e^{-x}\sum_{k=0}^{\infty}\frac{x^{k+\nu}}{\Gamma(k+\nu+1)},
  \end{equation}
  and is related to the upper incomplete gamma function by
  \begin{equation}
    \gamma(\nu,x) + \Gamma(\nu, x) = \Gamma(\nu).
  \end{equation}
  Therefore, under the absorbing boundary condition, the transition density is not normalized to 1, and there is a finite probability of
  absorption at 0.


\section{European option pricing}

  The forward, or undiscounted, European option prices can be obtained from the transition
  densities at the option epxiry by integrating with respect to the
  option payoff. For example, for a call option under absorbing boundary condition, we have
  \begin{equation}
    C_A(T,K)=\int_0^{+\infty}p_A(T,S)(S-K)^+dS = \int_K^{+\infty}Sp_A(T,S)dS - K\int_K^{+\infty}p_A(T,S)dS.
  \end{equation}
  Notice that the probability density function of a non-central $\chi^2$ distribution with $r$ degrees of freedom and the non-centrality
  parameter $\lambda$ is given by
  \begin{equation}
    f(x;r,\lambda) = \frac{1}{2}\left(\frac{x}{\lambda}\right)^{(r-2)/4}\exp\left(-\frac{x+\lambda}{2}\right)I_{(r-2)/2}\left(\sqrt{\lambda x}\right),
  \end{equation}
  and the transition density can be written as
  \begin{equation}
    p_A(T,S) = \frac{4\nu s_0 S^{1/\nu-2}}{\sigma^2T}f\left(\frac{4\nu^2S^{1/\nu}}{\sigma^2T};2\nu+2,\frac{4\nu^2s_0^{1/\nu}}{\sigma^2T}\right).
  \end{equation}
  With this, the first term in the above integral can be easily calculated by a change of variable
  \begin{eqnarray}
      && \int_K^{+\infty}\frac{4\nu s_0 S^{1/\nu-1}}{\sigma^2T}f\left(\frac{4\nu^2S^{1/\nu}}{\sigma^2T};2\nu+2,\frac{4\nu^2s_0^{1/\nu}}{\sigma^2T}\right)dS\nonumber\\
    = && s_0\int_K^{+\infty}f\left(\frac{4\nu^2S^{1/\nu}}{\sigma^2T};2\nu+2,\frac{4\nu^2s_0^{1/\nu}}{\sigma^2T}\right)d\left(\frac{4\nu^2S^{1/\nu}}{\sigma^2T}\right)\nonumber\\
    = && s_0\left(1 - F\left(\frac{4\nu^2K^{1/\nu}}{\sigma^2T};2\nu+2,\frac{4\nu^2s_0^{1/\nu}}{\sigma^2T}\right)\right),
  \end{eqnarray}
  where $F(x;r,\lambda)$ is the cumulative density function of the non-central $\chi^2$ distribution.
  For the second term, we can use the following symmetry property of the non-central $\chi^2$ distribution
  \begin{equation}
    \int_x^{+\infty}f(y;r-2,\lambda)dy + \int_{\lambda}^{+\infty}f(x;r,\mu)d\mu = 1,
  \end{equation}
  to show that
  \begin{eqnarray}
      && \int_K^{+\infty}\frac{4\nu s_0 S^{1/\nu-2}}{\sigma^2T}f\left(\frac{4\nu^2S^{1/\nu}}{\sigma^2T};2\nu+2,\frac{4\nu^2s_0^{1/\nu}}{\sigma^2T}\right)dS\nonumber\\
    = && \int_K^{+\infty}\frac{s_0}{S}f\left(\frac{4\nu^2S^{1/\nu}}{\sigma^2T};2\nu+2,\frac{4\nu^2s_0^{1/\nu}}{\sigma^2T}\right)d\left(\frac{4\nu^2S^{1/\nu}}{\sigma^2T}\right)\nonumber\\
    = && \int_q^{+\infty}\left(\frac{\lambda}{z}\right)^{\nu}f\left(z;2\nu+2,\lambda\right)dz \nonumber\\
    = && \int_q^{+\infty}f\left(\lambda;2\nu+2,z\right)dz \nonumber\\
    = && 1 - \int_{\lambda}^{+\infty}f\left(z;2\nu,q\right)dz \nonumber\\
    = && F(\lambda; 2\nu, q),
  \end{eqnarray}
  where we have defined
  \begin{equation}
    \lambda = \frac{4\nu^2s_0^{1/\nu}}{\sigma^2T}, \quad\quad q = \frac{4\nu^2K^{1/\nu}}{\sigma^2T}.
  \end{equation}
  Finally, the call option price under the absorbing boundary condition is given by
  \begin{equation}
    C_A(T,K)=s_0\left(1 - F\left(\frac{4\nu^2K^{1/\nu}}{\sigma^2T};2\nu+2,\frac{4\nu^2s_0^{1/\nu}}{\sigma^2T}\right)\right)
             - KF\left(\frac{4\nu^2s_0^{1/\nu}}{\sigma^2T};2\nu,\frac{4\nu^2K^{1/\nu}}{\sigma^2T}\right).
  \end{equation}
  The put option can be simply obtained by call-put parity
  \begin{equation}
    P_A(T,K)=s_0F\left(\frac{4\nu^2K^{1/\nu}}{\sigma^2T};2\nu+2,\frac{4\nu^2s_0^{1/\nu}}{\sigma^2T}\right)
             - K\left(1-F\left(\frac{4\nu^2s_0^{1/\nu}}{\sigma^2T};2\nu,\frac{4\nu^2K^{1/\nu}}{\sigma^2T}\right)\right).
  \end{equation}

  The option prices under the reflecting boundary condition can be calculated in a similar fashion. With repeating, they are given by
  \begin{equation}
    C_R(T,K)=s_0F\left(\frac{4\nu^2K^{1/\nu}}{\sigma^2T};-2\nu,\frac{4\nu^2s_0^{1/\nu}}{\sigma^2T}\right)
             - K\left(1-F\left(\frac{4\nu^2s_0^{1/\nu}}{\sigma^2T};-2\nu+2,\frac{4\nu^2K^{1/\nu}}{\sigma^2T}\right)\right),
  \end{equation}
  and
  \begin{equation}
    P_R(T,K)=s_0\left(1 - F\left(\frac{4\nu^2K^{1/\nu}}{\sigma^2T};-2\nu,\frac{4\nu^2s_0^{1/\nu}}{\sigma^2T}\right)\right)
             - KF\left(\frac{4\nu^2s_0^{1/\nu}}{\sigma^2T};-2\nu+2,\frac{4\nu^2K^{1/\nu}}{\sigma^2T}\right).
  \end{equation}


\section{Integral representation of European call option price}

  To preclude arbitrage opportunity, we are going to consider the European call option price under the absorbing boundary condition.
  The price for a European call option with expiry $T$, forward $S$, and strike $K$ is given by
  \begin{equation}
    C_A(T,S,K) = (S-K)^+ + \frac{\sigma^2}{2}K^{2\beta}\int_0^Tp_A(0,S;\tau,K)d\tau,
    \label{callprice}
  \end{equation}
  with the density $p(0,S;t,K)$ given by Eq. (\ref{density1}). To this end, we can use the integral representation of the modified
  Bessel function \cite{DLMF2},
  \begin{equation}
    I_{\nu}(z) = \frac{1}{\pi}\int_0^{\pi}e^{z\cos\phi}\cos(\nu\phi)d\phi
                - \frac{\sin(\nu\pi)}{\pi}\int_0^{+\infty}e^{-z\cosh \psi - \nu \psi}d\psi
  \end{equation}
  Define
  \begin{equation}
    q_S=\frac{S^{1-\beta}}{1-\beta}, \quad q_K=\frac{K^{1-\beta}}{1-\beta},
  \end{equation}
  the integration term of Eq. (\ref{callprice}) becomes
  \begin{eqnarray}
    && \nu\sqrt{SK}\int_0^T\frac{d\tau}{\tau}\exp\left(-\frac{q_K^2+q_S^2}{2\sigma^2\tau}\right)
    \Bigg(\frac{1}{\pi}\int_0^{\pi}\exp\left(\frac{q_Kq_S\cos\phi}{\sigma^2\tau}\right)\cos(\nu\phi)d\phi\nonumber\\
    && \quad\quad\quad\quad\quad\quad\quad\quad\quad\quad\quad\quad\quad\quad\quad\quad
            - \frac{\sin(\nu\pi)}{\pi}\int_0^{+\infty}\exp\left(-\frac{q_Kq_S\cosh \psi}{\sigma^2\tau}\right)e^{-\nu \psi}d\psi\Bigg)\nonumber\\
    = && \nu\sqrt{SK}\int_0^{\sigma^2 T}\frac{d\tau}{\tau}\exp\left(-\frac{q_K^2+q_S^2}{2\tau}\right)
    \Bigg(\frac{1}{\pi}\int_0^{\pi}\exp\left(\frac{q_Kq_S\cos\phi}{\tau}\right)\cos(\nu\phi)d\phi\nonumber\\
    && \quad\quad\quad\quad\quad\quad\quad\quad\quad\quad\quad\quad\quad\quad\quad\quad
            - \frac{\sin(\nu\pi)}{\pi}\int_0^{+\infty}\exp\left(-\frac{q_Kq_S\cosh \psi}{\tau}\right)e^{-\nu \psi}d\psi\Bigg).
  \end{eqnarray}
  Now, consider the first integration of the above expression. Change the variable $y=q_Kq_S/\tau$, define
  $b=(q_K^2+q_S^2)/2q_Kq_S$, and integrate by parts, we have
  \begin{eqnarray}
    && \frac{\nu\sqrt{SK}}{\pi}\int_0^{\sigma^2 T}\frac{d\tau}{\tau}\exp\left(-\frac{q_K^2+q_S^2}{2\tau}\right)
       \int_0^{\pi}\exp\left(\frac{q_Kq_S\cos\phi}{\tau}\right)\cos(\nu\phi)d\phi\nonumber\\
    = && \frac{\sqrt{SK}}{\pi}\int_{\frac{q_Kq_S}{\sigma^2 T}}^{+\infty}\frac{dy}{y}
       \int_0^{\pi}e^{-(b-\cos\phi)y}d\sin(\nu\phi)\nonumber\\
    = && \frac{\sqrt{SK}}{\pi}\int_{\frac{q_Kq_S}{\sigma^2 T}}^{+\infty}dy
         \left[\left.\frac{\sin(\nu\phi)}{y}e^{-(b-\cos\phi)y}\right|_0^{\pi}
              +\int_0^{\pi}\sin\phi\sin(\nu\phi)e^{-(b-\cos\phi)y}d\phi\right]\nonumber\\
    = && \frac{\sqrt{SK}}{\pi}\int_{\frac{q_Kq_S}{\sigma^2 T}}^{+\infty}\frac{\sin(\nu\pi)}{y}e^{-(b+1)y}dy\nonumber\\
      && \quad\quad\quad\quad+ \frac{\sqrt{SK}}{\pi}\int_0^{\pi}\frac{\sin\phi\sin(\nu\phi)}{b-\cos\phi}
       \exp\left(-\frac{q_Kq_S}{\sigma^2T}\left(b-\cos\phi\right)\right)d\phi.
  \end{eqnarray}
  Similarly, the second integration can be manipulated to give
  \begin{eqnarray}
    && -\frac{\sqrt{SK}}{\pi}\int_{\frac{q_Kq_S}{\sigma^2 T}}^{+\infty}\frac{\sin(\nu\pi)}{y}e^{-(b+1)y}dy\nonumber\\
      && \quad\quad\quad\quad+ \frac{\sqrt{SK}\sin(\nu\pi)}{\pi}\int_0^{+\infty}\frac{e^{-\nu \psi}\sinh \psi}{b+\cosh \psi}
       \exp\left(-\frac{q_Kq_S}{\sigma^2T}\left(b+\cosh \psi\right)\right)d\psi.
  \end{eqnarray}
  Finally, the call option price is
  \begin{eqnarray}
    C_A(T,S,K) &=& (S-K)^+ \nonumber\\
             &+&\frac{\sqrt{SK}}{\pi}\Bigg(\int_0^{\pi}\frac{\sin\phi\sin(\nu\phi)}{b-\cos\phi}
                        \exp\left(-\frac{q_Kq_S}{\sigma^2T}\left(b-\cos\phi\right)\right)d\phi\nonumber\\
             &&\quad\quad\quad\quad\quad + \sin(\nu\pi)\int_0^{+\infty}\frac{e^{-\nu \psi}\sinh \psi}{b+\cosh \psi}
       \exp\left(-\frac{q_Kq_S}{\sigma^2T}\left(b+\cosh \psi\right)\right)d\psi\Bigg).
    \label{CEVAbsorbingCall}
  \end{eqnarray}

  For completeness, we also record the result for reflecting boundary condition as
  \begin{eqnarray}
    C_R(T,S,K) &=& (S-K)^+ \nonumber\\
             &+&\frac{\sqrt{SK}}{\pi}\Bigg(\int_0^{\pi}\frac{\sin\phi\sin(\nu\phi)}{b-\cos\phi}
                        \exp\left(-\frac{q_Kq_S}{\sigma^2T}\left(b-\cos\phi\right)\right)d\phi\nonumber\\
             &&\quad\quad\quad\quad\quad + \sin(\nu\pi)\int_0^{+\infty}\frac{e^{\nu \psi}\sinh \psi}{b+\cosh \psi}
       \exp\left(-\frac{q_Kq_S}{\sigma^2T}\left(b+\cosh \psi\right)\right)d\psi\Bigg).
  \end{eqnarray}

\section{Free boundary CEV}

  The CEV process (\ref{CEV}) can be modified to allow negative forward in the following way,
  \begin{equation}
    dS_t=\sigma \left|S_t\right|^{\beta}dW_t,
    \label{FreeBoundaryCEV}
  \end{equation}
  with $0<\beta<1/2$. The corresponding Fokker-Planck equation is then given by
  \begin{equation}
    -\frac{\partial p(t,s)}{\partial t}+\frac{1}{2}\sigma^2\frac{\partial^2}{\partial s^2}
    \left(\left|s\right|^{2\beta}p(t,s)\right) = 0,
    \label{FBCEVFP}
  \end{equation}
  with initial condition (\ref{initial}). The solution to the Fokker-Planck equation (\ref{FBCEVFP})
  can be constructed as
  \begin{equation}
    p(t,s)=\frac{1}{2}\bigg(p_R\left(t,\left|s\right|\right)+{\rm sgn}(s)p_A\left(t,\left|s\right|\right)\bigg),
  \end{equation}
  where $p_A(t,s)$ and $p_R(t,s)$ are the solution to the Fokker-Plack equation (\ref{FP}) under
  the absorbing and reflecting boundary conditions, repectively. Following the time value of the option (\ref{TimeValue}),
  it can be shown that the call option price in the free boundary CEV model is
  \begin{eqnarray}
    C(T,S,K) &=& (S-K)^+ \nonumber\\
             &+&\frac{\sqrt{\left|SK\right|}}{\pi}\Bigg({\bf 1}_{K\geq 0}\int_0^{\pi}\frac{\sin\phi\sin(\nu\phi)}{b-\cos\phi}
                        \exp\left(-\frac{\bar{q}\left(b-\cos\phi\right)}{\sigma^2T}\right)d\phi\nonumber\\
             &&\quad\quad\quad\quad\quad + \sin(\nu\pi)\int_0^{+\infty}
                \frac{d\psi}{b+\cosh \psi}
       \exp\left(-\frac{\bar{q}\left(b+\cosh \psi\right)}{\sigma^2T}\right)\nonumber\\
             &&\quad\quad\quad\quad\quad\quad\quad\quad\quad\quad \times
                  \bigg({\bf 1}_{K\geq 0}\cosh(\nu \psi)+{\bf 1}_{K< 0}\sinh(\nu \psi)\bigg)\sinh \psi\Bigg),
  \end{eqnarray}
  where
  \begin{equation}
    \bar{q}=q_Kq_S,\quad b=\frac{q_K^2+q_S^2}{2q_Kq_S},\quad q_K=\frac{|K|^{1-\beta}}{1-\beta},
    \quad q_S=\frac{|S|^{1-\beta}}{1-\beta}.
  \end{equation}

\section{Implied volatility expansion}

  In this section, we will consider two expansion results for the implied volatilities of local volatility models,
  whose local volatility functions do not explicitly contain time dependence, {\it i.e.},
  \begin{equation}
    dS_t=\sigma\varphi(S_t)dW_t.
    \label{expansion1}
  \end{equation}
  The Black-Scholes equation for the forward call option price under this local volatility model is given by
  \begin{equation}
    \frac{\partial C}{\partial t} +\frac{\sigma^2}{2}S^2\frac{\partial^2C}{\partial S^2} = 0,
  \end{equation}
  where $C(t,S;T,K)$ is the forward value of a call option with terminal payoff of $f(S)=(S-K)^+$ at $T$. Write
  $C(t,S;T,K)$ as $g(\tau,S)$, with $\tau=\sigma^2(T-t)$, then the Black-Scholes equation becomes
  \begin{equation}
    \frac{\partial g}{\partial \tau} -\frac{1}{2}S^2\frac{\partial^2g}{\partial S^2} = 0.
    \label{expansion2}
  \end{equation}
  This will be the starting point for our expansion of the implied volatility.


  \subsection{Small expiry expansion}

    Let us expand around a shifted lognormal model,
    \begin{equation}
      dS_t=\lambda(\beta+\zeta S_t)dW_t,
    \end{equation}
    with $\lambda, \zeta\neq 0$. The forward call option price under this model is given by
    \begin{equation}
      C(t,S;T,K)=\left(S+\frac{\beta}{\zeta}\right)N(d_+) - \left(K+\frac{\beta}{\zeta}\right)N(d_-),
    \end{equation}
    where
    \begin{equation}
      d_{\pm} = \frac{\log\left(\frac{S+\beta / \zeta}{K+\beta / \zeta}\right)\pm \frac{1}{2}\lambda^2\zeta^2(T-t)}
                     {|\lambda\zeta|\sqrt{T-t}},
    \end{equation}
    and $N(x)$ is the cumulative density function of the standard normal distribution, which is related to the probability
    density function,
    \begin{equation}
      n(x) = \frac{1}{\sqrt{2\pi}}e^{-\frac{x^2}{2}},
    \end{equation}
    by $N(x)=\int_{-\infty}^xn(u)du$.

    For the local volatility model (\ref{expansion1}), we can guess the forward call option price to be given by the
    following form,
    \begin{equation}
      g(\tau,S)=\left(S+\frac{\beta}{\zeta}\right)N(z_+) - \left(K+\frac{\beta}{\zeta}\right)N(z_-),
    \end{equation}
    where
    \begin{equation}
      z_{\pm} = \frac{\log\left(\frac{S+\beta / \zeta}{K+\beta / \zeta}\right)\pm \frac{1}{2}\Omega(\tau,S)^2}
                     {\Omega(\tau,S)}.
      \label{expansion3}
    \end{equation}
    Notice that
    \begin{eqnarray}
      \frac{\partial g}{\partial \tau}
      &=& \left(S+\frac{\beta}{\zeta}\right)n(z_+)\frac{\partial z_+}{\partial\tau}
        - \left(K+\frac{\beta}{\zeta}\right)n(z_-)\frac{\partial z_-}{\partial\tau} \nonumber\\
      &=& \left(S+\frac{\beta}{\zeta}\right)n(z_+)\left(\frac{\partial z_+}{\partial\tau}
                                                     - \frac{\partial z_-}{\partial\tau}\right) \nonumber\\
      &=& \left(S+\frac{\beta}{\zeta}\right)n(z_+)\frac{\partial \Omega}{\partial\tau},
    \end{eqnarray}
    \begin{eqnarray}
      \frac{\partial g}{\partial S}
      &=& N(z_+) + \left(S+\frac{\beta}{\zeta}\right)n(z_+)\frac{\partial z_+}{\partial S}
        - \left(K+\frac{\beta}{\zeta}\right)n(z_-)\frac{\partial z_-}{\partial S} \nonumber\\
      &=& N(z_+) + \left(S+\frac{\beta}{\zeta}\right)n(z_+)\frac{\partial \Omega}{\partial S},
    \end{eqnarray}
    and
    \begin{eqnarray}
      \frac{\partial^2 g}{\partial S^2}
      &=& n(z_+)\frac{\partial z_+}{\partial S} + n(z_+)\frac{\partial \Omega}{\partial S}
        + \left(S+\frac{\beta}{\zeta}\right)n(z_+)\frac{\partial^2 \Omega}{\partial S^2}
        - \left(S+\frac{\beta}{\zeta}\right)n(z_+)z_+\frac{\partial z_+}{\partial S}\frac{\partial \Omega}{\partial S}\nonumber\\
      &=& n(z_+)\left(\frac{\partial z_+}{\partial S} + \frac{\partial \Omega}{\partial S}
        + \left(S+\frac{\beta}{\zeta}\right)\frac{\partial^2 \Omega}{\partial S^2}
        - \left(S+\frac{\beta}{\zeta}\right)z_+\frac{\partial z_+}{\partial S}\frac{\partial \Omega}{\partial S}\right),
    \end{eqnarray}
    where we have used the fact that
    \begin{equation}
      \left(S+\frac{\beta}{\zeta}\right)n(z_+) = \left(K+\frac{\beta}{\zeta}\right)n(z_-).
    \end{equation}
    Also, we have
    \begin{equation}
      \frac{\partial z_+}{\partial S} = \frac{1}{\Omega(S+\beta/\zeta)}-\frac{z_-}{\Omega}\frac{\partial\Omega}{\partial S},
    \end{equation}
    therefore
    \begin{eqnarray}
      \frac{\partial^2 g}{\partial S^2}
      &=& n(z_+)\Bigg(\frac{1}{\Omega(S+\beta/\zeta)}-\frac{z_-}{\Omega}\frac{\partial\Omega}{\partial S} + \frac{\partial \Omega}{\partial S}
          + \left(S+\frac{\beta}{\zeta}\right)\frac{\partial^2 \Omega}{\partial S^2}\nonumber\\
      &&  \quad\quad\quad - \left(S+\frac{\beta}{\zeta}\right)z_+\left(\frac{1}{\Omega(S+\beta/\zeta)}-\frac{z_-}{\Omega}
                            \frac{\partial\Omega}{\partial S}\right) \frac{\partial \Omega}{\partial S}\Bigg).
    \end{eqnarray}
    Substituting (\ref{expansion3}) into (\ref{expansion2}), and taking into account of the derivatives above, we have
    \begin{equation}
      \left(S+\frac{\beta}{\zeta}\right)^2\Omega\frac{\partial \Omega}{\partial\tau} =
      \frac{1}{2}\varphi(S)^2\left[\left(S+\frac{\beta}{\zeta}\right)^2\Omega\frac{\partial^2 \Omega}{\partial S^2}
                                   +(1-h_{-3})-h_1(1-h_{-1})\right],
      \label{expansion4}
    \end{equation}
    where
    \begin{equation}
      h_i=\left(S+\frac{\beta}{\zeta}\right)\frac{\partial \Omega}{\partial S}
          \left[\frac{1}{\Omega}\log\left(\frac{S+\beta / \zeta}{K+\beta / \zeta}\right)+\frac{1}{2}i\Omega\right],\quad
          i=-3,-1,1.
    \end{equation}
    Now, we seek a small expiry expansion in the form of
    \begin{equation}
      \Omega(\tau,S)=\sum_{i=0}^{\infty}\tau^{i+1/2}\Omega_i(S).
    \end{equation}
    Expanding to $O(\tau^2)$, we have
    \begin{eqnarray}
      \Omega\frac{\partial \Omega}{\partial\tau} &=& \left(\Omega_0\tau^{1/2}+\Omega_1\tau^{3/2}+\cdots\right)
                                                     \left(\frac{1}{2}\Omega_0\tau^{-1/2}+\frac{3}{2}\Omega_1\tau^{1/2}+\cdots\right)\nonumber\\
                                                 &=& \frac{1}{2}\Omega_0^2 + 2\Omega_0\Omega_1\tau + O(\tau^2),
    \end{eqnarray}
    \begin{eqnarray}
      \Omega\frac{\partial^2 \Omega}{\partial S^2} &=& \left(\Omega_0\tau^{1/2}+\Omega_1\tau^{3/2}+\cdots\right)
                                                     \left(\Omega_0^{\prime\prime}\tau^{1/2}
                                                     +\Omega_1^{\prime\prime}\tau^{3/2}+\cdots\right)\nonumber\\
                                                 &=& \Omega_0\Omega_0^{\prime\prime}\tau + O(\tau^2),
    \end{eqnarray}
    \begin{eqnarray}
      h_i &=& \left(S+\frac{\beta}{\zeta}\right)\left(\Omega_0^{\prime}\tau^{1/2}+\Omega_1^{\prime}\tau^{3/2}+\cdots\right) \nonumber\\
          && \times\left[\log\left(\frac{S+\beta / \zeta}{K+\beta / \zeta}\right)
                         \left(\frac{1}{\Omega_0}\tau^{-1/2}-\frac{\Omega_1}{\Omega_0^2}\tau^{1/2}+\cdots\right)
                         +\frac{1}{2}i\left(\Omega_0\tau^{1/2}+\Omega_1\tau^{3/2}+\cdots\right)\right] \nonumber\\
          &=& \left(S+\frac{\beta}{\zeta}\right)\Bigg[\log\left(\frac{S+\beta / \zeta}{K+\beta / \zeta}\right)
                                                      \frac{\Omega_0^{\prime}}{\Omega_0}\nonumber\\
          && \quad\quad\quad\quad\quad\quad +\left(\frac{1}{2}i\Omega_0\Omega_0^{\prime}
                                   +\log\left(\frac{S+\beta / \zeta}{K+\beta / \zeta}\right)
                                    \left(\frac{\Omega_1^{\prime}}{\Omega_0}-\frac{\Omega_1\Omega_0^{\prime}}{\Omega_0^2}\right)\right)\tau
                                   +O(\tau^2)\Bigg]
    \end{eqnarray}
    and
    \begin{equation}
      h_1h_{-1} =
    \end{equation}
    Matching terms of order $O(1)$, we have
    \begin{equation}
      \left(S+\frac{\beta}{\zeta}\right)^2\Omega_0^2 =
      \varphi(S)^2\left(1-\left(S+\frac{\beta}{\zeta}\right)\log\left(\frac{S+\beta / \zeta}{K+\beta / \zeta}\right)
      \frac{\Omega_0^{\prime}}{\Omega_0}\right)^2.
    \end{equation}
    Taking square root, and retain only the positive solution, the above ODE will become a Bernoulli one,
    \begin{equation}
      \Omega_0^{\prime} = \frac{\Omega_0}{\left(S+\frac{\beta}{\zeta}\right)\log\left(\frac{S+\beta / \zeta}{K+\beta / \zeta}\right)}
                        - \frac{\Omega_0^2}{\varphi(S)\log\left(\frac{S+\beta / \zeta}{K+\beta / \zeta}\right)}.
    \end{equation}
    Define $\psi=\Omega_0^{-1}$, the above ODE will be reduced to
    \begin{equation}
      \left(\log\left(\frac{S+\beta / \zeta}{K+\beta / \zeta}\right)\psi\right)^{\prime} = \frac{1}{\varphi(S)},
    \end{equation}
    which can be directly integrated,
    \begin{equation}
      \psi(S) = \left(\log\left(\frac{S+\beta / \zeta}{K+\beta / \zeta}\right)\right)^{-1}\int_K^S\frac{du}{\varphi(u)}.
    \end{equation}
    Thus, the leading order in the implied volatility expansion is given by
    \begin{equation}
      \Omega_0(S) = \log\left(\frac{S+\beta / \zeta}{K+\beta / \zeta}\right)\left(\int_K^S\frac{du}{\varphi(u)}\right)^{-1}.
    \end{equation}

    For the $O(\tau)$ term, we have the following ODE,
    \begin{eqnarray}
      2\left(S+\frac{\beta}{\zeta}\right)\Omega_1 &=&
      \frac{1}{2}\varphi(S)^2\left(\left(S+\frac{\beta}{\zeta}\right)\Omega_0^{\prime\prime}+\Omega_0^{\prime}\right) \nonumber\\
      &&\quad\quad\quad\quad - \varphi(S)\left(S+\frac{\beta}{\zeta}\right)\log\left(\frac{S+\beta / \zeta}{K+\beta / \zeta}\right)
                       \left(\frac{\Omega_1^{\prime}}{\Omega_0}-\frac{\Omega_1\Omega_0^{\prime}}{\Omega_0^2}\right),
    \end{eqnarray}
    whose solution is given by
    \begin{equation}
      \Omega_1(S) = -\frac{\Omega_0(S)}{\left(\int_K^S\frac{du}{\varphi(u)}\right)^2}\log\left(\Omega_0(S)
                    \left(\frac{(S+\beta / \zeta)(K+\beta / \zeta)}{\varphi(S)\varphi(K)}\right)^{1/2}\right).
    \end{equation}

    Up to $O\left(\tau^{5/2}\right)$, the small expiry expansion of the implied volatility is
    \begin{equation}
      \Omega(\tau,S)=\Omega_0(S)\tau^{1/2}+\Omega_1(S)\tau^{3/2}+O\left(\tau^{5/2}\right),
    \end{equation}
    where $\Omega_0(S)$ and $\Omega_1(S)$ are given above.




  \subsection{Singular perturbation expansion}



\section{A final remark}

  For all the above results, if we make the volatility parameter $\sigma$ time dependent, we can simply replace $\sigma^2(T-t)$
  by $\int_t^T\sigma(u)^2du$. This can be verified by checking the effect of this change of variable in the Black-Scholes equation.
  However, this is only valid when the local volatility function is seperable.




\begin{thebibliography}{99}
  \bibitem{Lesniewski}
    See \url{http://www.lesniewski.us/papers/working/NotesOnCEV.pdf}.

  \bibitem{Bessel}
    See \url{http://mathworld.wolfram.com/BesselDifferentialEquation.html}.

  \bibitem{DLMF1}
    See \url{http://dlmf.nist.gov/10.22.E67}.

  \bibitem{DLMF2}
    See \url{http://dlmf.nist.gov/10.32.E4}.
\end{thebibliography}


\end{document}
