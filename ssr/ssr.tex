\documentclass[12pt]{article}
\usepackage{amsfonts}
\usepackage[hscale=.8,vscale=.8]{geometry}
\usepackage{hyperref}

\begin{document}

\title{Notes on Skew Stickiness Ratio}
\date{Dec. 32, 2999}

\maketitle

\section{Power payoff}

  Consider the power payoff, $S(T)^p$, for maturity $T$. Under the Black-Scholes (BS) dynamics,
  \begin{equation}
    dS(t) = (r-q)S(t)dt+\sigma S(t)dW(t),
  \end{equation}
  where $r$ and $q$ are the constant risk free rate and repo rate, and $\sigma$ is the lognormal volatility of the
  underlying, we have
  \begin{equation}
    d\left(\log\left(S(t)^p\right)\right) = p\left(r-q-\frac{\sigma ^2}{2}\right)dt + p\sigma dW(t).
  \end{equation}
  This can be integrated to give
  \begin{equation}
    \log\left(\frac{S(T)}{S(t)}\right)^p = p\left(r-q-\frac{\sigma ^2}{2}\right)(T-t) + p\sigma\left(W(T)-W(t)\right),
  \end{equation}
  and the discounted price of the power payoff, is
  \begin{equation}
    Q_p(t,T)=e^{-r(T-t)}{\mathrm E}\left[S(T)^p\right]=e^{-r(T-t)}F(t,T)^p\exp\left(\frac{p(p-1)}{2}\sigma^2(T-t)\right),
    \label{power1}
  \end{equation}
  where we have defined the forward $F(t,T)=S(t)e^{(r-q)(T-t)}$.

  Given the market price of a power payoff, its implied volatility $\sigma_p(t,T)$ can be obtained by inverting
  Eq. (\ref{power1}). This quantity is well-defined for $p\in (0,1)$ when the power payoff is concave, or
  $p\in (-\infty, 0)\cup (1,+\infty)$ when the power payoff is convex. As long as $p\neq 0$ or 1, the implied volatility
  can always be obtained.

  \subsection{Implied volatility}

    The power payoff is European. According to the static replication method, it can be replicated by call and/or
    put options with the same maturity but a continuum of strikes. For $p\in (0,1)$, the price $Q_p(t,T)$ of a power payoff
    is always finite, and there exists an explicit formula which expresses the implied volatility as a weighted average of
    vanilla implied volatilities.

    Define the characteristic function for $S(t)$ as
    \begin{equation}
      L(p)={\mathrm E}[e^{px}],
    \end{equation}
    with $x=\log(S(T)/F(t,T))$, and the undiscounted price of a power payoff is related to this characteristic function by the
    following,
    \begin{equation}
      L(p)={\mathrm E}\left[\left(\frac{S(T)}{F(t,T)}\right)^p\right]=e^{r(T-t)}\frac{Q_p(t,T)}{F(t,T)^p}
          =\exp\left(\frac{p(p-1)}{2}\sigma_p(t,T)^2(T-t)\right).
    \end{equation}
    Using the terminal distribution of the underlying, the characteristic function can be expressed as an integral,
    \begin{equation}
      L(p)=\int_0^{+\infty}\rho(K)\exp\left(p\log\left(\frac{K}{F(t,T)}\right)\right)dK.
    \end{equation}
    Denote the time $t$ price of a call option with maturity $T$ and strike $K$ as $C_{{\rm BS}}(T,K)$,
    then the terminal distribution is given by the second derivative of the option price with respect to the strike, {\it i.e.},
    \begin{equation}
      \rho(K)=\frac{\partial^2V(T,K)}{\partial K^2},
    \end{equation}
    where $V(T,K) = e^{r(T-t)}C_{{\rm BS}}(T,K)$ is the undiscounted, or forward, option price. Now, the characteristic function becomes
    \begin{eqnarray}
      L(p) &=& \int_0^{+\infty}\frac{\partial^2V(T,K)}{\partial K^2}\exp\left(p\log\left(\frac{K}{F(t,T)}\right)\right)dK \nonumber \\
           &=& - \int_0^{+\infty}\frac{\partial V(T,K)}{\partial K }\frac{p}{K}\exp\left(p\log\left(\frac{K}{F(t,T)}\right)\right)dK,
    \end{eqnarray}
    by intgration by parts. Define
    \begin{equation}
      z(K)=\frac{\log\left(F(t,T)/K\right)}{\sigma_{{\rm BS}}(T,K)\sqrt{T}}-\frac{\sigma_{{\rm BS}}(T,K)\sqrt{T}}{2},
    \end{equation}
    where $\sigma_{{\rm BS}}(T,K)$ is the implied BS volatility of the vanilla option. This is a monotonic function, and allows an
    inverse for $K$ as a function of $z$. The vanilla option price and the greeks can be expressed with $z(K)$. In particular, for undiscounted
    call option price,
    \begin{equation}
      V(T,K) =  F(t,T)N\left(z(K)+\sigma_{{\rm BS}}(T,K)\sqrt{T}\right) - KN(z(K)),
    \end{equation}
    we have
    \begin{equation}
      \frac{\partial V(T,K)}{\partial K} = Kn(z(K))\frac{\partial\sigma_{{\rm BS}}(T,K)}{\partial K}\sqrt{T}-N(z(K)),
    \end{equation}
    where $n(x)$ and $N(x)$ are the Gaussian distribution and the cumulated Gaussian distribution, respectively. The characteristic function
    becomes
    \begin{eqnarray}
      L(p) &=& \int_0^{+\infty}\left[N(z(K))-Kn(z(K))\frac{\partial\sigma_{{\rm BS}}(T,K)}{\partial K}\sqrt{T}\right]\frac{p}{K}
                     \exp\left(p\log\left(\frac{K}{F(t,T)}\right)\right)dK \nonumber\\
           &=& -\int_0^{+\infty}n(z(K))\exp\left(p\log\left(\frac{K}{F(t,T)}\right)\right)\left[\frac{dz}{dK}
                                   + p\frac{\partial\sigma_{{\rm BS}}(T,K)}{\partial K}\sqrt{T}\right]dK,
    \end{eqnarray}
    where we have again performed the integration by parts for the first term. Define $w(z)=\sigma_{{\rm BS}}(T,z(K))\sqrt{T}$, and notice that
    \begin{equation}
      \log\left(\frac{K}{F(t,T)}\right)=-\left(\frac{w^2}{2}+wz\right),
    \end{equation}
    and
    \begin{equation}
      \frac{\partial\sigma_{{\rm BS}}(T,K)}{\partial K}\sqrt{T} = \frac{\partial w}{\partial z}\frac{dz}{dK},
    \end{equation}
    the characteristic function can be finally written as
    \begin{equation}
      L(p) = \int_{-\infty}^{+\infty}\frac{e^{-z^2/2}}{\sqrt{2\pi}}e^{-p(w^2/2+wz)}\left(1+p\frac{\partial w}{\partial z}\right)dz.
    \end{equation}
    Introduce another variable
    \begin{equation}
      y = z + pw(z),
    \end{equation}
    the characteristic function can be transformed to a simpler form,
    \begin{equation}
      L(p) = \int_{-\infty}^{+\infty}\frac{e^{-y^2/2}}{\sqrt{2\pi}}\exp\left(\frac{p(p-1)}{2}\sigma_{{\rm BS}}(T,K(y))^2(T-t)\right)dy.
    \end{equation}
    Finally, the relation between the vanilla and the power payoff implied volatilities is
    \begin{equation}
      \exp\left(\frac{p(p-1)}{2}\sigma_p(t,T)^2(T-t)\right)
      = \int_{-\infty}^{+\infty}\frac{e^{-y^2/2}}{\sqrt{2\pi}}
        \exp\left(\frac{p(p-1)}{2}\sigma_{{\rm BS}}(T,K(y,p))^2(T-t)\right)dy.
      \label{implied}
    \end{equation}

    \subsubsection{Limiting cases}

      Now, let us consider two limiting cases of the above result.

      {\it i) p=0.} For $p\rightarrow 0$, we have $S^p\rightarrow 1 + p\log S$, and $\sigma_p(t,T)$ becomes the implied volatility of the log contract.
      Expanding both sides of Eq. (\ref{implied}), we have
      \begin{equation}
        1 + \frac{p}{2}\sigma_0(t,T)^2(T-t)
        = \int_{-\infty}^{+\infty}\frac{e^{-y^2/2}}{\sqrt{2\pi}}
          \left(1 + \frac{p}{2}\sigma_{{\rm BS}}(T,K(y,0))^2(T-t)\right)dy.
      \end{equation}
      Thus, the implied volatility of a log contract is given by
      \begin{equation}
        \sigma_{\log}(t,T)^2
        = \int_{-\infty}^{+\infty}\frac{e^{-y^2/2}}{\sqrt{2\pi}}
          \sigma_{{\rm BS}}(T,K(y))^2dy,
      \end{equation}
      with
      \begin{equation}
        y(K)=\frac{\log\left(F(t,T)/K\right)}{\sigma_{{\rm BS}}(T,K)\sqrt{T}}-\frac{\sigma_{{\rm BS}}(T,K)\sqrt{T}}{2}.
      \end{equation}

      {\it ii) p=1.} For $p\rightarrow 1$, set $p=1-\epsilon$, with $\epsilon\rightarrow 0$. Then, we have $S^p\rightarrow S - \epsilon S\log S$,
      and $\sigma_p(t,T)$ becomes the implied volatility of a $S\log S$ contract. Expanding both sides of Eq. (\ref{implied}), we have
      \begin{equation}
        1 - \frac{\epsilon}{2}\sigma_0(t,T)^2(T-t)
        = \int_{-\infty}^{+\infty}\frac{e^{-y^2/2}}{\sqrt{2\pi}}
          \left(1 - \frac{\epsilon}{2}\sigma_{{\rm BS}}(T,K(y,0))^2(T-t)\right)dy.
      \end{equation}
      Thus, the implied volatility of a $S\log S$ contract is given by
      \begin{equation}
        \sigma_{S\log S}(t,T)^2
        = \int_{-\infty}^{+\infty}\frac{e^{-y^2/2}}{\sqrt{2\pi}}
          \sigma_{{\rm BS}}(T,K(y))^2dy,
      \end{equation}
      with
      \begin{equation}
        y(K)=\frac{\log\left(F(t,T)/K\right)}{\sigma_{{\rm BS}}(T,K)\sqrt{T}}+\frac{\sigma_{{\rm BS}}(T,K)\sqrt{T}}{2}.
      \end{equation}



  \subsection{Forward variance and its dynamics}

    For power payoffs with $p\neq 0$ or 1, the implied volatilities can always be solved from market prices, and satisfy the convex order
    condition (see appendix or reference), {\it i.e.},
    \begin{equation}
      \sigma_p(t,T_1)^2\left(T_1-t\right) \leq \sigma_p(t,T_2)^2\left(T_2-t\right),\quad\quad {\rm for} \quad T_1 \leq T_2.
    \end{equation}
    Therefore, we can define the continuous forward variance,
    \begin{equation}
      \xi_p(t,T) = \frac{d}{dT}\left(\sigma_p(t,T)^2\left(T-t\right)\right),
    \end{equation}
    which is positive by the convex order condition.

    In the following, we will consider the joint dynamics of the underlying and the forward variance. For simplicity, we will assume zero
    risk free rate and repo rate. In this case, the drift of the price of power payoffs will vanish. We can use this condition
    to determine the drift and diffusion of the forward variance.

    Let us assume the SDE for the underlying and the forward variance is given by
    \begin{eqnarray}
      dS(t) &=& \sigma(t)S(t)dW^S(t),\nonumber\\
      d\xi_p(t,T) &=& \mu_p(t,T)dt + \lambda_p(t,T)dW^T(t),\nonumber
    \end{eqnarray}
    with the correlations between the underlying and the forward variance and between different forward variances given by
    \begin{equation}
      \rho^{ST} = \frac{\langle dW^S(t)dW^T(t)\rangle}{dt}, \quad\quad \rho^{uv} = \frac{\langle dW^u(t)dW^v(t)\rangle}{dt},
    \end{equation}
    respectively.

    The price of a power payoff is
    \begin{equation}
      Q_p(t,T)=S(t)^p\exp\left(\frac{p(p-1)}{2}\int_t^T\xi_p(t,\tau)d\tau\right).
    \end{equation}
    Applying the Ito lemma,
    \begin{eqnarray}
      dQ_p(t,T) &=& \frac{Q_p(t,T)}{S(t)}\cdot pdS(t) + \frac{Q_p(t,T)}{S(t)^2}\cdot\frac{p(p-1)}{2}dS(t)dS(t)\nonumber\\
                &+& Q_p(t,T)\cdot \frac{p(p-1)}{2} d\left(\int_t^T\xi_p(t,\tau)d\tau\right)\nonumber\\
                &+& Q_p(t,T)\cdot \frac{p^2(p-1)^2}{8} d\left(\int_t^T\xi_p(t,u)du\right)d\left(\int_t^T\xi_p(t,v)dv\right) \nonumber\\
                &+& \frac{Q_p(t,T)}{S(t)}\cdot pdS(t) \cdot \frac{p(p-1)}{2} d\left(\int_t^T\xi_p(t,\tau)d\tau\right),
    \end{eqnarray}
    which is equivalent to
    \begin{eqnarray}
      \frac{dQ_p(t,T)}{Q_p(t,T)}
                &=& p\sigma(t)dW^S(t) + \frac{p(p-1)}{2}\sigma(t)^2dt\nonumber\\
                &+& \frac{p(p-1)}{2}\left(-\xi_p(t,t) + \int_t^T\left(d\xi_p(t,\tau)\right)d\tau\right)\nonumber\\
                &+& \frac{p^2(p-1)^2}{8} \int_t^Tdu\int_t^Tdvd\xi_p(t,u)d\xi_p(t,v) \nonumber\\
                &+& \frac{p^2(p-1)}{2}\int_t^TdS(t)d\xi_p(t,\tau)d\tau \nonumber\\
                &=& \Bigg[\frac{p(p-1)}{2}\left(\sigma(t)^2 - \xi_p(t,t)\right) + \frac{p(p-1)}{2}\int_t^T \mu_p(t,\tau)d\tau \nonumber\\
                && \quad + \quad \frac{p^2(p-1)}{2}\sigma(t)\int_t^T\rho^{S\tau}\lambda_p(t,\tau)d\tau \nonumber\\
                && \quad + \quad \frac{p^2(p-1)^2}{8} \int_t^Tdu\int_t^Tdv\rho^{uv}\lambda_p(t,u)\lambda_p(t,v)\Bigg]dt\nonumber\\
                &+& \left[p\sigma(t)dW^S(t) + \frac{p(p-1)}{2}\left(\int_t^T\lambda_p(t,\tau)d\tau\right)dW^{\tau}(t)\right].
    \end{eqnarray}
    Since the drift term needs to vanish for all $T$, we can take the derivative of the drift term with respect to $T$, which leads to
    \begin{equation}
      \frac{p(p-1)}{2}\mu_p(t,T) + \frac{p^2(p-1)}{2}\rho^{ST}\sigma(t)\lambda_p(t,T)
           + \frac{p^2(p-1)^2}{4} \int_t^T\rho^{Tv}\lambda_p(t,T)\lambda_p(t,v)dv = 0,
    \end{equation}
    or,
    \begin{equation}
      \mu_p(t,T) = -\left(p\rho^{ST}\sigma(t)\lambda_p(t,T) + \frac{p(p-1)}{2} \int_t^T\rho^{Tv}\lambda_p(t,T)\lambda_p(t,v)dv\right).
    \end{equation}
    Also, in the limit of $T\rightarrow t$, we are left with a term proportional to $\sigma(t)^2 - \xi_p(t,t)$. Therefore, the diffusion of
    the underlying must satisfy an extra condition,
    \begin{equation}
      \sigma(t) = \sqrt{\xi_p(t,t)},
    \end{equation}
    for all values of $p$.

    The final joint dynamics of the underlying and the forward variance is now given by
    \begin{eqnarray}
      dS(t) &=& \sqrt{\xi_p(t,t)}S(t)dW^S(t),\\
      d\xi_p(t,T) &=& -\left(p\rho^{ST}\sqrt{\xi_p(t,t)}\lambda_p(t,T) + \frac{p(p-1)}{2} \int_t^T\rho^{Tv}\lambda_p(t,T)\lambda_p(t,v)dv\right)dt \nonumber\\
                  &&  + \quad\lambda_p(t,T)dW^T(t).
    \end{eqnarray}


  \section{Variance swap}




  \section{Heston model}





  \section{Forward variance model}




\begin{thebibliography}{99}
  \bibitem{Bergomi}
    Lorenzo Bergomi, {\it Stochastic Volatility Modeling}, Chapman Hall/CRC Press, 2016.
\end{thebibliography}


\end{document}
