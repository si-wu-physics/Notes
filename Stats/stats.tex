\documentclass[12pt]{article}
\usepackage{amsfonts}
\usepackage[hscale=.8,vscale=.8]{geometry}
\usepackage{hyperref}

\usepackage{amsthm}
\usepackage{enumitem}

\newtheorem{theorem}{Theorem}[section]
\newtheorem{corollary}{Corollary}[theorem]
\newtheorem{lemma}[theorem]{Lemma}
\newtheorem{remark}{Remark}

\begin{document}

\title{Notes on Statistics}
\date{Dec. 32, 2999}

\maketitle

\section{Distributions from random samples}

  \subsection{Random samples and statistics}

    The random variables $X_1,\cdots,X_n$ are called a ramdom sample of 
    size $n$ from the population $f(x)$ if they are independent and identically
    distributed with pdf or pmf $f(x)$, or iid random
    variables. Let $T(x_1,\cdots,x_n)$ be a real-valued or vector-valued
    function whose domain includes of sample space of $(X_1,\cdots,X_n)$,
    then the random variable or random vector $Y=T(X_1,\cdots,X_n)$ is called 
    a statistic, whose distribution is called the sampling distribution of $Y$.


\section{Point estimation}

\section{Hypothesis testing}

\section{Interval estimation}

\section{A complete example}

\section{More examples}

\section{Analysis of variance}

\section{Linear regression}

\appendix

  \section{Distribution of transformations of random variables}
  \label{tf}

    \begin{equation}
      f_{{\bf U}}(u_1,\cdots,u_n)=\sum_{i}f_{{\bf X}}
                 (h_{1i}(u_1,\cdots,u_n),\cdots,h_{ni}(u_1,\cdots,u_n))\left|J_i\right|,
    \end{equation}

  \section{Two useful relatios}

\begin{thebibliography}{99}
\end{thebibliography}


\end{document}