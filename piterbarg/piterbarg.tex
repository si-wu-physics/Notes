\documentclass[12pt]{article}
\usepackage{amsfonts}
\usepackage[hscale=.8,vscale=.8]{geometry}
\usepackage{hyperref}

\usepackage{amsthm}
\usepackage{enumitem}

\newtheorem{theorem}{Theorem}[section]
\newtheorem{corollary}{Corollary}[theorem]
\newtheorem{lemma}[theorem]{Lemma}
\newtheorem{remark}{Remark}

\begin{document}

\title{Notes on Piterbarg's Formulation of Heston Model}
\date{Dec. 32, 2999}

\maketitle

\section{Shifted Lognormal Model}

  Consider the following shifted lognormal process for underlying $S$,
  \begin{equation}
    dS_t=\lambda \bigg(bS_t+\left(1-b\right)L\bigg)dW_t,
  \end{equation}
  the price of a call option written on $S$ is given by
  \begin{equation}
    C(t,S_t,T,K) = \frac{1}{b}\Big[\bigg(bS_t+\left(1-b\right)L\bigg)N\left(d_+\right)
                 - \bigg(bK+\left(1-b\right)L\bigg)N\left(d_-\right)\Big],
  \end{equation}
  where
  \begin{equation}
    d_{\pm} = \frac{1}{\left|\lambda b\right|\sqrt{T-t}}\left(\log\left(\frac{bS_t+\left(1-b\right)L}{bK+\left(1-b\right)L}\right)
            \pm \frac{1}{2}\lambda^2b^2(T-t)\right),
  \end{equation}
  $S_t$ is the spot price of the underlying at time $t$, and the strike and the expiry
  of the call option are $K$ and $T$, respectively. The parameters $\lambda$ and $b$ are
  responsible for the overall level and the slope of the implied volatility smile, respectively.

\section{Stochastic Volatility Model}

  We consider a stochastic volatility model in the following form,
  \begin{eqnarray}
    &&dS_t=\lambda \bigg(bS_t+\left(1-b\right)L\bigg)\sqrt{z_t}dW_t,\\
    &&dz_t=\theta(z_0-z_t)dt+\eta\sqrt{z_t}dZ_t,
  \end{eqnarray}
  with $\langle dW_t, dZ_t \rangle = \rho dt$. If $b=1$, this model will reduce to the
  Heston model. All the parameters have similar meaning as their counterparts of the
  Heston model. Since we have already explicitly included the volatility $\lambda$ of the underlying
  process, we can set $z_0=1$. The volatlity of variance $\eta$ is responsible for the
  convexity of the implied volatility smile.

  The volatility follows the Cox-Ingersoll-Ross (CIR) process. Its properties have
  been discussed in \cite{Feller}, and will not be repeated here.

  Define
  \begin{equation}
    X_t=\frac{bS_t+\left(1-b\right)L}{bS_0+\left(1-b\right)L},
  \end{equation}
  we have
  \begin{equation}
    \frac{dX_t}{X_t}=\lambda b\sqrt{z_t}dW_t,
  \end{equation}
  which has the following solution,
  \begin{equation}
    X_t=\exp\left(-\frac{1}{2}\lambda^2b^2\int_0^tz_udu+\lambda b\int_0^t\sqrt{z_u}dW_u\right).
  \end{equation}


\begin{thebibliography}{99}
  \bibitem{Feller}
    {\it Notes on Feller Condition}, personal notes.

\end{thebibliography}


\end{document}