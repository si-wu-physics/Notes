\documentclass[12pt]{article}
\usepackage{amsfonts}
\usepackage[hscale=.8,vscale=.8]{geometry}
\usepackage{hyperref}

\begin{document}

\title{Notes on SABR Model}
\date{Dec. 32, 2999}

\maketitle

  The SABR model is characterized by the following stochastic process,
  \begin{eqnarray}
    && dS_t=\sigma_t S_t^{\beta}dW_t,\nonumber\\
    && d\sigma_t=\gamma\sigma_tdZ_t,\\
    && {\mathrm E}[dW_tdZ_t]=\rho dt,\nonumber
  \end{eqnarray}
  where the volatility of the forward process is now stochastic, $\gamma$ is the volatility of volatility (volvol),
  and $\rho$ is the correlation. The initial volatility of $\sigma_t$ is assumed to be $\sigma_0$. We also assume that
  the absorbing boundary condition is imposed for the forward process to preserve the martingale property.
  In the following, we will present several approximate results.


\section{Projection to SABR model with zero correlation}

  When the correlation between the underlying and the volatility vanishes, the SABR model admits a (semi) closed form solution
  for the option prices. We can start from this and build an approximation for general correlations.

  \subsection{Zero correlation}

    As in the last section, define
    \begin{equation}
      q_S=\frac{S^{1-\beta}}{1-\beta}, \quad q_K=\frac{K^{1-\beta}}{1-\beta},
      \quad \bar{q}=q_Sq_K, \quad b=\frac{q_S^2+q_K^2}{2q_Sq_K},
    \end{equation}
    the call option price is given by
    \begin{eqnarray}
      C(T,S,K) &=& (S-K)^+ \nonumber\\
               &+&\frac{\sqrt{SK}}{\pi}\Bigg(\int_0^{\pi}\frac{\sin\phi\sin(\nu\phi)}{b-\cos\phi}
                          \frac{G(\gamma^2T,s(\phi))}{\cosh\left(s(\phi)\right)}d\phi\nonumber\\
               &&\quad\quad\quad\quad\quad + \sin(\nu\pi)\int_0^{+\infty}\frac{e^{-\nu \psi}\sinh \psi}{b+\cosh \psi}
         \frac{G(\gamma^2T,s(\psi))}{\cosh\left(s(\psi)\right)}d\psi\Bigg),
      \label{SABRZC}
    \end{eqnarray}
    where
    \begin{eqnarray}
      && \sinh\left(s(\phi)\right) =\frac{\gamma}{\sigma_0}\sqrt{2\bar{q}(b-\cos\phi)}, \nonumber\\
      && \sinh\left(s(\psi)\right) =\frac{\gamma}{\sigma_0}\sqrt{2\bar{q}(b+\cosh\psi)}.
    \end{eqnarray}
    Here,
    \begin{equation}
      G(t,s)=\frac{2e^{-t/8}}{t\sqrt{\pi t}}\int_s^{+\infty}ue^{-u^2/2t}\sqrt{\cosh u - \cosh s}du,
    \end{equation}
    is an integration kernel, which has a highly accurate series expansion approximation,
    \begin{equation}
      G(t,s)=\sqrt{\frac{\sinh s}{s}}e^{-\frac{s^2}{2t}-\frac{t}{8}}\left(R(t,s)+\delta R(t,s)\right),
    \end{equation}
    where
    \begin{equation}
      R(t,s)=1+\frac{3tg(s)}{8s^2}-\frac{5t^2(-8s^2+3g^2(s)+24g(s))}{128s^4}
              +\frac{35t^3(-40s^2+3g^3(s)+24g^2(s)+120g(s))}{1024s^6},
    \end{equation}
    with $g(s)=s\coth(s)-1$, and
    \begin{equation}
      \delta R(t,s)=e^{\frac{t}{8}}-\frac{3072+384t+24t^2+t^3}{3072}.
    \end{equation}
    In the limit of $s\rightarrow 0$, $G(t,s)$ has the following MacLaurin expansion,
    \begin{equation}
      G(t,s)=e^{-\frac{s^2}{2t}}\left(1+\left(\frac{1}{12}-\frac{2688+80t^2+21t^3}{322560}e^{-\frac{t}{8}}\right)s^2\right).
    \end{equation}



  \subsection{Non-zero correlation}

    For non-zero correlation model, we need to map it to a zero correlation model. Denote the mapped parameters with a tilde,
    then the mapped $\beta$ and $\gamma$ can be chosen as
    \begin{equation}
      \tilde{\beta} = \beta, \quad
      \tilde{\gamma}^2 = \gamma^2 - \frac{3}{2}\bigg(\gamma^2\rho^2+\sigma_0\gamma\rho(1-\beta)S^{\beta-1}\bigg).
    \end{equation}
    Then, the mapped initial volatility can be expressed as an expansion in maturity
    \begin{equation}
      \tilde{\sigma}_0=\tilde{\sigma}_0^{(0)}+T\tilde{\sigma}_0^{(1)}.
    \end{equation}
    The zeroth order term is given by
    \begin{equation}
      \tilde{\sigma}_0^{(0)}=\frac{2\Phi\delta\tilde{q}\tilde{\gamma}}{\Phi^2-1},
    \end{equation}
    where
    \begin{equation}
      \Phi=\left(\frac{\sigma_{{\rm min}}+\rho\sigma_0+\gamma\delta q}{(1+\rho)\sigma_0}\right)^{\frac{\tilde{\gamma}}{\gamma}},
    \end{equation}
    and
    \begin{equation}
      \delta q = \frac{K^{1-\beta}-S^{1-\beta}}{1-\beta},\quad
      \delta \tilde{q} = \frac{K^{1-\tilde{\beta}}-S^{1-\tilde{\beta}}}{1-\tilde{\beta}} = \delta q,
    \end{equation}
    since $\tilde{\beta} = \beta$. Also,
    \begin{equation}
      \sigma_{{\rm min}}^2=\gamma^2\delta q^2+2\rho\gamma\delta q\sigma_0+\sigma_0^2.
    \end{equation}

    The first order term is given by
    \begin{equation}
      \frac{\tilde{\sigma}_0^{(1)}}{\tilde{\sigma}_0^{(0)}}=\tilde{\gamma}^2\sqrt{1+\tilde{R}^2}
      \frac{\displaystyle \frac{1}{2}\log\left(\frac{\sigma_0\sigma_{{\rm min}}}{\tilde{\sigma}_0^{(0)}\tilde{\sigma}_{{\rm min}}}\right)
                                                -\mathcal{B}_{{\rm min}}}
      {\tilde{R}\log\bigg(\tilde{R}+\sqrt{\tilde{R}^2+1}\bigg)},
    \end{equation}
    where
    \begin{equation}
      \tilde{R} = \frac{\delta q\tilde{\gamma}}{\tilde{\sigma}_0^{(0)}},\quad
      \tilde{\sigma}_{{\rm min}}^2 = \tilde{\gamma}^2\delta q^2+\left(\tilde{\sigma}_0^{(0)}\right)^2,
    \end{equation}
    and $\mathcal{B}_{{\rm min}}$ is given by
    \begin{equation}
      \mathcal{B}_{{\rm min}}=-\frac{1}{2}\frac{\beta}{1-\beta}\frac{\rho}{\sqrt{1-\rho^2}}
      \left(\pi-\varphi_0-\arccos\rho-I\right).
    \end{equation}
    Here,
    \begin{equation}
      \varphi_0=\arccos\left(-\frac{\gamma\delta q+\rho\sigma_0}{\sigma_{{\rm min}}}\right),\quad
      L=\frac{\sigma_{{\rm min}}}{\gamma q_K\sqrt{1-\rho^2}},
    \end{equation}
    and
    \begin{eqnarray}
      I=\frac{2}{\sqrt{1-L^2}}\left(\arctan\left(\frac{u_0+L}{\sqrt{1-L^2}}\right)
                    -\arctan\left(\frac{L}{\sqrt{1-L^2}}\right)\right)       && \quad L < 1 \\
      I=\frac{1}{\sqrt{L^2-1}}\log\left(\frac{1+u_0\left(L+\sqrt{L^2-1}\right)}
                                               {1+u_0\left(L-\sqrt{L^2-1}\right)}\right)  && \quad L > 1
    \end{eqnarray}
    with the limit $I(L=1)=2u_0/(1+u_0)$, where
    \begin{equation}
      u_0=\frac{\delta q \gamma\rho+\sigma_0-\sigma_{{\rm min}}}{\delta q \gamma\sqrt{1-\rho^2}}.
    \end{equation}

    Finally, the call option price of the non-zero correlation model can be obtained by replace the parameters
    in Eq. (\ref{SABRZC}) with the mapped parameters.

    \subsubsection{Limiting cases}

      In the following limiting cases, the mapped $\gamma$ parameter will be simplified.

      {\it i) At-the-money strike.} For $K=S$, we have
      \begin{equation}
        \tilde{\sigma}_0^{(0)}=\sigma_0,
      \end{equation}
      and
      \begin{equation}
        \frac{\tilde{\sigma}_0^{(1)}}{\tilde{\sigma}_0^{(0)}}=\frac{1}{12}
        \left(1-\frac{\tilde{\gamma}^2}{\gamma^2}-\frac{3}{2}\rho^2\right)\gamma^2
        +\frac{1}{4}\beta\rho\gamma\sigma_0S^{\beta -1}.
      \end{equation}

      {\it ii) Negative $\tilde{\gamma}^2$.} If the mapping of $\tilde{\gamma}$ leads to a
      negative $\tilde{\gamma}^2$, we can set $\tilde{\gamma}=0$, and the mapped $\tilde{\sigma}_0$
      will take the following limiting form,
      \begin{equation}
        \tilde{\sigma}_0^{(0)}=\frac{\gamma\delta\tilde{q}}
        {\displaystyle \log\left(\frac{\sigma_{{\rm min}}+\rho\sigma_0+\gamma\delta q}{(1+\rho)\sigma_0}\right)},
      \end{equation}
      and
      \begin{equation}
        \frac{\tilde{\sigma}_0^{(1)}}{\tilde{\sigma}_0^{(0)}}=\frac{\gamma^2}
        {\displaystyle \log^2\left(\frac{\sigma_{{\rm min}}+\rho\sigma_0+\gamma\delta q}{(1+\rho)\sigma_0}\right)}
        \left(\frac{1}{2}\log\left(\frac{\sigma_0\sigma_{{\rm min}}}{\tilde{\sigma}_0^{(0)}\tilde{\sigma}_{{\rm min}}}\right)
                                                -\mathcal{B}_{{\rm min}}\right).
      \end{equation}
      Since $\tilde{\gamma}=0$, the mapped model will degenerate to the CEV model, and the call option price
      can be obtained by replacing the parameters in Eq. (\ref{CEVAbsorbingCall}) with the mapped parameters.



\section{Free boundary SABR}

  Similar to the free boundary CEV process, the forward process of the SABR model can also be modified to allow negative
  forward in the following way,
  \begin{equation}
    dS_t=\sigma \left|S_t\right|^{\beta}dW_t,
    \label{FreeBoundarySABR}
  \end{equation}
  with $0<\beta<1/2$, which becomes the free boundary SABR model. When the correlation is zero, $\rho=0$, the call option
  value has the similar form to the SABR model,
  \begin{eqnarray}
    C(T,S,K) &=& (S-K)^+ \nonumber\\
             &+&\frac{\sqrt{\left|SK\right|}}{\pi}\Bigg({\bf 1}_{K\geq 0}\int_0^{\pi}\frac{\sin\phi\sin(\nu\phi)}{b-\cos\phi}
                        \frac{G(\gamma^2T,s(\phi))}{\cosh\left(s(\phi)\right)}d\phi\nonumber\\
             &&\quad\quad\quad\quad\quad + \sin(\nu\pi)\int_0^{+\infty}
                \frac{d\psi}{b+\cosh \psi}
       \frac{G(\gamma^2T,s(\psi))}{\cosh\left(s(\psi)\right)}\nonumber\\
             &&\quad\quad\quad\quad\quad\quad\quad\quad\quad\quad \times
                  \bigg({\bf 1}_{K\geq 0}\cosh(\nu \psi)+{\bf 1}_{K< 0}\sinh(\nu \psi)\bigg)\sinh \psi\Bigg).
  \end{eqnarray}
  For non-zero correlation, the parameters can be mapped to those of a zero correlation model as in the SABR case. The only difference is
  that
  \begin{equation}
      \delta q = \frac{|k|^{1-\beta}-|S|^{1-\beta}}{1-\beta},\quad
      \delta \tilde{q} = \frac{|k|^{1-\tilde{\beta}}-|S|^{1-\tilde{\beta}}}{1-\tilde{\beta}} = \delta q,
  \end{equation}
  where
  \begin{equation}
    k={\rm max}(K,0.1S)
  \end{equation}
  and
  \begin{equation}
    L=\frac{\sigma_{{\rm min}}(1-\beta)}{\gamma k^{1-\beta}\sqrt{1-\rho^2}},
  \end{equation}
  If the mapped $\tilde{\gamma}^2$ is negative, we can similarly set $\tilde{\gamma}$ to 0, and the free boundary SABR model
  will degenerate to the corresponding free boundary CEV model.



\begin{thebibliography}{99}
  \bibitem{Lesniewski}
    See \url{http://www.lesniewski.us/papers/working/NotesOnCEV.pdf}.

  \bibitem{Bessel}
    See \url{http://mathworld.wolfram.com/BesselDifferentialEquation.html}.

  \bibitem{DLMF1}
    See \url{http://dlmf.nist.gov/10.22.E67}.

  \bibitem{DLMF2}
    See \url{http://dlmf.nist.gov/10.32.E4}.
\end{thebibliography}


\end{document}
