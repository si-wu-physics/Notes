\documentclass[12pt]{article}
\usepackage{amsfonts}
\usepackage[hscale=.8,vscale=.8]{geometry}
\usepackage{hyperref}

\usepackage{amsthm}
\usepackage{enumitem}

\newtheorem{theorem}{Theorem}[section]
\newtheorem{lemma}[theorem]{Lemma}
\newtheorem{corollary}[theorem]{Corollary}
\newtheorem{remark}[theorem]{Remark}

\theoremstyle{definition}
\newtheorem{definition}[theorem]{Definition}
\newtheorem{example}[theorem]{Example}

\begin{document}

\title{Notes on Statistics}
\date{Dec. 32, 2999}

\maketitle

\section{Distributions from random samples}

  \subsection{Random samples and statistics}

    \begin{definition}
      The random variables $X_1,\cdots,X_n$ are called a ramdom sample of 
      size $n$ from the population $f(x)$ if they are independent and identically
      distributed with pdf or pmf $f(x)$, or iid random
      variables. Let $T(x_1,\cdots,x_n)$ be a real-valued or vector-valued
      function whose domain includes of sample space of $(X_1,\cdots,X_n)$,
      then the random variable or random vector $Y=T(X_1,\cdots,X_n)$ is called 
      a statistic, whose distribution is called the sampling distribution of $Y$.
    \end{definition}

    \begin{theorem}
      This is another theorem.
    \end{theorem}
    \begin{proof}
      This proves the theorem.
    \end{proof}

    \begin{example}
      This is an example.
    \end{example}


\section{Point estimation}

  \begin{example}
    This is another example.
  \end{example}

  \begin{remark}
    This is just a remark.
  \end{remark}

\section{Hypothesis testing}

\section{Interval estimation}

\section{A complete example}

\section{More examples}

\section{Analysis of variance}

\section{Linear regression}

\appendix

  \section{Distribution of transformations of random variables}
  \label{tf}

    \begin{equation}
      f_{{\bf U}}(u_1,\cdots,u_n)=\sum_{i}f_{{\bf X}}
                 (h_{1i}(u_1,\cdots,u_n),\cdots,h_{ni}(u_1,\cdots,u_n))\left|J_i\right|,
    \end{equation}

  \section{Cochran's theorem}

    \begin{theorem}[Cochran]
      Let $X_1$, $X_2$, $\cdots$, $X_n$ be i.i.d. $N(0,\sigma^2)$ distributed random
      variables, and suppose that
      \[\sum_{i=1}^n X_i^2 = \sum_{j=1}^k Q_j, \]
      where $Q_1$, $Q_2$, $\cdots$, $Q_k$ are positive semi-definite quadratic forms in
      $X_1$, $X_2$, $\cdots$, $X_n$, {\it i.e.},
      \[ Q_j = {\bf X}^{\top}{\bf A}_j{\bf X},\quad j=1,2,\cdots,k. \]
      Let $r_j={\rm rank}({\bf A}_j)$ be the rank of the matrix ${\bf A}_j$. If $\sum_{j=1}^kr_j=n$,
      then
      \begin{itemize}
        \item $Q_1$, $Q_2$, $\cdots$, $Q_k$ are independent;
        \item $Q_j \sim \sigma^2\chi^2(r_j)$.
      \end{itemize}
    \end{theorem}

    From linear algebra, we know that if a matrix \({\bf A}\) is symmetric and idempotent,
    then \({\rm rank}({\bf A}) = {\rm tr}({\bf A})\). Since \({\bf A}\) is idempotent, {\it i.e.},
    \({\bf A}^2 = {\bf A}\), the eigenvalues of the matrix \({\bf A}\) are either 0 or 1. Therefore,
    the rank of the matrix is equal to the sum of its eigenvalues. We will use this fact to determine
    the degrees of freedom for the \(\chi^2\) distributions corresponding to the quadratic forms.
    In this case, we will only need to verify that the matrices are idempotent.

  \section{Two useful relations}

\begin{thebibliography}{99}
\end{thebibliography}


\end{document}