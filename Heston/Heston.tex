\documentclass[12pt]{article}
\usepackage{amsfonts}
\usepackage[hscale=.8,vscale=.8]{geometry}
\usepackage{hyperref}

\begin{document}

\title{Notes on Characteristic Functions}
\date{Dec. 32, 2999}

\maketitle

For an underlying $S_t$ following certain stochastic process, its characteristic function is defined as
\begin{equation}
  \label{Char01}
  \varphi(u)=E[\left.e^{iuX_T}\right|X_t],
\end{equation}
where $X_t=\log S_t$.

\section{Fourier Inversion Formula}

  Consider the problem of finding the in-the-money (ITM) probability of the underlying asset at maturity, $P(\log S_T > \log K)$, which is given by
  \begin{equation}
    P(\log S_T > k)=\int_k^{+\infty}f(x)dx,
  \end{equation}
  with $k=\log K$. Here, $f(x)$ is the probability density function of $X_T=\log S_T$, and can be recovered from the characteristic function (\ref{Char01})
  by the inverse Fourier transform,
  \begin{equation}
    f(x)=\frac{1}{2\pi}\int_{-\infty}^{+\infty}e^{-iux}\varphi(u)du.
  \end{equation}
  Therefore,
  \begin{eqnarray}
    P(\log S_T > k)&=&\frac{1}{2\pi}\int_k^{+\infty} \left(\int_{-\infty}^{+\infty}e^{-iux}\varphi(u)du\right)dx
                        =\frac{1}{2\pi}\int_{-\infty}^{+\infty}\varphi(u) \left(\int_k^{+\infty}e^{-iux}dx\right)du\nonumber\\
                   &=&\frac{1}{2\pi}\int_{-\infty}^{+\infty}\varphi(u)\frac{e^{-iuk}}{iu}du
                        - \frac{1}{2\pi}\lim_{R\rightarrow +\infty}\int_{-\infty}^{+\infty}\varphi(u)\frac{e^{-iuR}}{iu}du.
  \end{eqnarray}
  To evaluate the second term, we again use the definition of the characteristic function,
  \begin{eqnarray}
    \frac{1}{2\pi}\lim_{R\rightarrow +\infty}\int_{-\infty}^{+\infty}\varphi(u)\frac{e^{-iuR}}{iu}du
    &=& \frac{1}{2\pi}\lim_{R\rightarrow +\infty}\int_{-\infty}^{+\infty}\left(\int_{-\infty}^{+\infty}e^{iux}f(x)dx\right)\frac{e^{-iuR}}{iu}du\nonumber\\
    &=& \frac{1}{2\pi}\lim_{R\rightarrow +\infty}\int_{-\infty}^{+\infty}f(x)\left(\int_{-\infty}^{+\infty}\frac{e^{iu(x-R)}}{iu}du\right)dx\nonumber\\
    &=& \frac{1}{2\pi}\lim_{R\rightarrow +\infty}\int_{-\infty}^{+\infty}f(x)\cdot\pi {\rm sgn}(x-R)dx\nonumber\\
    &=& \frac{1}{2}\lim_{R\rightarrow +\infty}\left(1-2F(R)\right) = -\frac{1}{2}.
  \end{eqnarray}
  Here, we have used the fact that
  \begin{equation}
    \int_{-\infty}^{+\infty}{\rm sgn}(x-y)f(x)dx=\int_y^{+\infty}f(x)dx - \int_{-\infty}^yf(x)dx = 1-2F(x),
  \end{equation}
  where $F(x)$ is the cumulative distribution function of $X_T$.

  Finally, the ITM probability is given by
  \begin{equation}
    \label{inversion}
    P(\log S_T > k)=\frac{1}{2}+\frac{1}{2\pi}\int_{-\infty}^{+\infty}\varphi(u)\frac{e^{-iuk}}{iu}du.
  \end{equation}


\section{Vanilla option pricing with characteristic functions}

  In the following, we are going to explore several formulations of vanilla option prices in terms of
  characteristic functions.

  \subsection{Heston \cite{Heston}}

    Consider the call option price
    \begin{equation}
      C(K)=e^{-r(T-t)}{\rm E}^{\mathbb{Q}}\left[\left(S_T-K\right)^+\right],
    \end{equation}
    under the risk neutral measure.
    We are seeking a representation similar to the Black-Scholes formula. To this end, we can write the
    call option prices as
    \begin{equation}
      C(K)=e^{-r(T-t)}{\rm E}^{\mathbb{Q}}\left[\left(S_T-K\right){\mathcal I}_{S_T>K}\right],
    \end{equation}
    where ${\mathcal I}$ is the indicator function. Now, we have
    \begin{equation}
      C(K)=e^{-r(T-t)}{\rm E}^{\mathbb{Q}}\left[S_T{\mathcal I}_{S_T>K}\right]-Ke^{-r(T-t)}{\rm E}^{\mathbb{Q}}\left[{\mathcal I}_{S_T>K}\right]
          =S_tP_1 - Ke^{-r(T-t)}P_2,
    \end{equation}
    where
    \begin{equation}
      P_1 = {\rm E}^{\mathbb{Q}}\left[\frac{S_T/S_t}{B_T/B_t}{\mathcal I}_{S_T>K}\right],\quad\quad
      P_2 = {\rm E}^{\mathbb{Q}}\left[{\mathcal I}_{S_T>K}\right].
    \end{equation}
    It is obvious that $P_2$ is the risk neutral probability for the underlying asset maturing in-the-money. Also, $P_1$ can
    be represented as the ITM probability in another measure. Notice that
    \begin{equation}
      \frac{d{\mathbb{Q}}}{d{\mathbb{Q}}^S}=\frac{B_T/B_t}{S_T/S_t}=\frac{{\rm E}^{\mathbb{Q}}[e^{X_T}]}{e^{X_T}},
    \end{equation}
    where we have used the fact that
    \begin{equation}
      {\rm E}^{\mathbb{Q}}[S_T]={\rm E}^{\mathbb{Q}}[e^{X_T}]=S_t\frac{B_T}{B_t},
    \end{equation}
    then
    \begin{equation}
      P_1 = {\rm E}^{{\mathbb{Q}^S}}\left[\frac{S_T/S_t}{B_T/B_t}{\mathcal I}_{S_T>K}\frac{dQ}{dQ^S}\right]
          = {\rm E}^{{\mathbb{Q}^S}}\left[{\mathcal I}_{S_T>K}\right].
    \end{equation}
    From Eq. (\ref{inversion}), the two probabilities $P_1$ and $P_2$ can be written as
    \begin{equation}
      P_j=\frac{1}{2}+\frac{1}{2\pi}\int_{-\infty}^{+\infty}\varphi_j(u)\frac{e^{-iuk}}{iu}du, \quad\quad j=1,2.
    \end{equation}
    It seems that two characteristic functions corresponding to the two different measures are required in the valuation of the option
    price. However, these two characteristic functions are related due to the measure change. To see this, notice that
    \begin{equation}
      \varphi_1(u)=\int_{-\infty}^{+\infty}e^{iux}f^S(x)dx,
    \end{equation}
    where $f^S(x)$ is the probability density function of $X_T$ under the $\mathbb{Q}^S$-measure. It is related to $f(x)$, the probability density
    function of $X_T$ under the $\mathbb{Q}$-measure, through the measure change,
    \begin{equation}
      f^S(x)=f(x)\frac{d{\mathbb{Q}}^S}{d{\mathbb{Q}}}=f(x)\frac{e^{X_T}}{{\rm E}^{\mathbb{Q}}[e^{X_T}]}.
    \end{equation}
    Now,
    \begin{equation}
      \varphi_1(u)=\frac{1}{{\rm E}^{\mathbb{Q}}[e^{X_T}]}\int_{-\infty}^{+\infty}e^{ix(u-i)}f(x)dx=\frac{\varphi(u-i)}{\varphi(-i)},
    \end{equation}
    since
    \begin{equation}
      {\rm E}^{\mathbb{Q}}[e^{X_T}]=\int_{-\infty}^{+\infty}e^{x}f(x)dx=\varphi(-i),
    \end{equation}
    where $\varphi(u)$ is the characteristic function as defined in (\ref{Char01}), and coincides with $\varphi_2(u)$. Therefore,
    \begin{eqnarray}
      P_1 &=& \frac{1}{2}+\frac{1}{2\pi}\int_{-\infty}^{+\infty}\frac{e^{-iu\log K}\varphi(u-i)}{iu\varphi(-i)}du, \\
      P_2 &=& \frac{1}{2}+\frac{1}{2\pi}\int_{-\infty}^{+\infty}\frac{e^{-iu\log K}\varphi(u)}{iu}du. \\
    \end{eqnarray}

  \subsection{Carr and Madan \cite{CarrMadan}}

    The call option price can be represented as
    \begin{equation}
      C(k)=e^{-r(T-t)}\int_k^{+\infty}(e^x-e^k)f(x)dx,
    \end{equation}
    where $k=\log K$, $x=\log S_T$, and $f(x)$ is the probability density function for the distribution of $x$ at maturity. We want to find the
    Fourier transform of the above call option price, but it is not integrable. To remedy this, we introduce a damping factor and modify the
    call option price accodingly,
    \begin{equation}
      c(k)=e^{\alpha k}C(k),
    \end{equation}
    and the corresponding Fourier transform is given by
    \begin{equation}
      \hat{c}(v)=\int_{-\infty}^{+\infty}e^{ivk}c(k)dk.
    \end{equation}
    Rearrange the integration order, we have
    \begin{eqnarray}
      \hat{c}(v) &=& e^{-r(T-t)}\int_{-\infty}^{+\infty}e^{ivk}\left(\int_k^{+\infty}e^{\alpha k}(e^x-e^k)f(x)dx\right)dk \nonumber\\
                 &=& e^{-r(T-t)}\int_{-\infty}^{+\infty}f(x)\left(\int_{-\infty}^x\left(e^{(\alpha+iv)k}e^x-e^{(\alpha+iv+1)k}\right)dk\right)dx \nonumber\\
                 &=& e^{-r(T-t)}\int_{-\infty}^{+\infty}f(x)\left.\left[\frac{e^{(\alpha+iv)k}e^x}{\alpha+iv}
                                                        -\frac{e^{(\alpha+iv+1)k}}{\alpha+iv+1}\right]\right|_{-\infty}^{x}dx \nonumber\\
                 &=& e^{-r(T-t)}\int_{-\infty}^{+\infty}f(x)\left[\frac{e^{(\alpha+iv+1)x}}{\alpha+iv} -\frac{e^{(\alpha+iv+1)x}}{\alpha+iv+1}\right]dx \nonumber\\
                 &=& e^{-r(T-t)}\int_{-\infty}^{+\infty}f(x)\frac{e^{(\alpha+iv+1)x}}{(\alpha+iv)(\alpha+iv+1)}dx \nonumber\\
                 &=& \frac{e^{-r(T-t)}\varphi(v-(1+\alpha)i)}{(\alpha+iv)(\alpha+iv+1)},
    \end{eqnarray}
    where $\alpha > 0$ is necessary to ensure convergence at $k=-\infty$. Once the Fourier transform of the modified call option price is known,
    we can use the inverse Fourier transform to recover the call option price,
    \begin{equation}
      C(k)=\frac{e^{-\alpha k}}{2\pi}\int_{-\infty}^{+\infty}e^{-ivk}\hat{c}(v)dv.
    \end{equation}

  \subsection{Lewis \cite{Lewis}}

    Denote the payoff the contingent claim at maturity as $g(x)$, then the value of the derivative is given by
    \begin{equation}
      V = \int_{-\infty}^{+\infty}f(x)g(x)dx.
    \end{equation}


\begin{thebibliography}{99}
  \bibitem{Heston}
    S. Heston, {\it A Closed-Form Solution for Options with Stochastic Volatility with Applications to Bond and Currency Options}, Review of Financial Studies {\bf 6}, 327 (1993).

  \bibitem{CarrMadan}
    P. Carr and D. Madan, {\it Option valuation using the fast Fourier transform}, The Journal of Computational Finance {\bf 2}, 61 (1999).

  \bibitem{Lewis}
    A. Lewis, {\it A Simple Option Formula for General Jump-Diffusion and Other Exponential Levy Processes}, \url{https://ssrn.com/abstract=282110}.

\end{thebibliography}


\end{document}