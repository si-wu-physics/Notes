\documentclass[12pt]{article}
\usepackage{amsfonts}
\usepackage[hscale=.8,vscale=.8]{geometry}
\usepackage{hyperref}

\usepackage{amsthm}
\usepackage{enumitem}

\newtheorem{theorem}{Theorem}[section]
\newtheorem{corollary}{Corollary}[theorem]
\newtheorem{lemma}[theorem]{Lemma}
\newtheorem{remark}{Remark}

\begin{document}

\title{Notes on Characteristic Functions}
\date{Dec. 32, 2999}

\maketitle

\section{Characteristic Function}

  A stochastic process $X_t$ is called a {\it L\'{e}vy process} if it has the following properties:
  \begin{enumerate}[noitemsep]
    \item $X_0=0$ almost surealy;
    \item {\it Independence of increments}. For any $0\le t_1 < t_2 < \cdots < t_n < +\infty$, $X_{t_2}-X_{t_1}$, 
          $X_{t_3}-{t_2}$, $\cdots$, $X_{t_n}-X_{t_{n-1}}$ are independent;
    \item {\it Stationary increments}. For any $s<t$, $X_t-X_s$ is equal in distribution to $X_{t-s}$;
    \item {\it Continuity in probability}. For any $\epsilon>0$ and $t\ge 0$, $\lim_{h\rightarrow 0}P(|X_{t+h}-X_t>\epsilon|)=0$.
  \end{enumerate}
  For the L\'{e}vy process, the characteristic function is given by
  \begin{equation}
    \varphi(u)={\rm E}\left[e^{iuX_t}\right]=\exp\left\{t\left(iua-\frac{1}{2}u^2\sigma^2
              +\int_{\mathbb{R}\textbackslash\{0\}}\left(e^{iux}-1-iux\mathcal{I}_{|x|<1}\right)\Pi(dx)\right)\right\}.
    \label{char}
  \end{equation}

  In the following, we will give several examples of L\'{e}vy process and their corresponding characteristic functions.

  \subsection{Black-Scholes Model}

    For the Black-Scholes model, 
    \begin{equation}
      dS_t=(r-q)S_tdt+\sigma S_tdW_t,
    \end{equation}
    we can define 
    \begin{equation}
      X_t=\log\left(\frac{S_t}{S_0e^{(r-q)t}}\right),
    \end{equation}
    it can be shown that
    \begin{equation}
      dX_t = -\frac{1}{2}\sigma^2 dt + \sigma dW_t.
    \end{equation}
    Therefore, $X_t\sim N(-\sigma^2t/2, \sigma^2t)$, where $N(\mu,\sigma^2)$ is a Gaussian distribution with mean and variance
    as $\mu$ and $\sigma^2$, respectively. The characteristic function can be found by direct integration
    \begin{equation}
      \varphi(u)=\frac{1}{2\pi\sigma^2 t}\int \exp\left(iux-\frac{\left(x+\frac{1}{2}\sigma^2t\right)^2}{2\sigma^2t}\right)
                =\exp\left(-\frac{\sigma^2t}{2}u(u+i)\right).
      \label{BS}
    \end{equation}

    \begin{remark}
      In many applications, the characteristic function is alternatively defined as ${\rm E}\left[e^{iu\log S_t}\right]$, which is related
      to (\ref{BS}) by a factor, $\exp\left[iu(\log S_0 +(r-q)t)\right]$. We will use these two definitions interchangeably, and which 
      definition is used is clear from the context.
    \end{remark}

  \subsection{Heston Model}

    The state variable dynamics of the Heston model is given by
    \begin{eqnarray}
      dS_t &=& (r-q)S_tdt+\sqrt{v_t}S_tdW_t, \\
      dv_t &=& \kappa(\theta - v_t)dt + \sigma\sqrt{v_t}dB_t,
    \end{eqnarray}
    with $\langle dW_t,dB_t\rangle=\rho dt$. Consider the bivariate characteristic function,
    \begin{equation}
      \varphi(u_1,u_2; x_t, v_t)={\rm E}\left[\left.\exp\left(iu_1x_T+iu_2v_T\right)\right|x_t,v_t\right],
    \end{equation}
    where $u_t=\log S_t$. Define $\tau = T-t$, we are seeking a solution in the form of
    \begin{equation}
      \varphi(u_1,u_2; x_t, v_t) = \exp\left(A(\tau,u_1,u_2)+B(\tau,u_1,u_2)x_t+C(\tau,u_1,u_2)v_t\right).
    \end{equation}
    According to the Feynman-Kac theorem, the bivariate characteristic function satisfies the following PDE,
    \begin{equation}
      -\frac{\partial \varphi}{\partial \tau} + \left(r-q-\frac{1}{2}v\right)\frac{\partial \varphi}{\partial x}
      + \kappa(\theta - v)\frac{\partial \varphi}{\partial x} + \frac{1}{2}v\frac{\partial^2 \varphi}{\partial x^2}
      + \rho\sigma v\frac{\partial^2 \varphi}{\partial x\partial v}+\frac{1}{2}\sigma^2v\frac{\partial^2 \varphi}{\partial v^2}=0,
    \end{equation}
    with initial conditions $A(0,u_1,u_2)=0$, $B(0,u_1,u_2)=iu_1$, and $C(0,,u_1,u_2)=iu_2$. Matching terms, the
    PDE becomes coupled ODEs,
    \begin{eqnarray}
      \frac{dA}{d\tau} &=& (r-q)B+\kappa\theta C, \\
      \frac{dB}{d\tau} &=& 0, \\
      \frac{dC}{d\tau} &=& \frac{1}{2}\sigma^2C^2+(\rho\sigma B-\kappa)C+\frac{1}{2}B^2-\frac{1}{2}B.
    \end{eqnarray}
    Using the initial condtion, we have $B(\tau,u_1,u_2)=iu_1$, and the other two equations become
    \begin{eqnarray}
      \frac{dA}{d\tau} &=& i(r-q)u_1+\kappa\theta C, \label{A} \\
      \frac{dC}{d\tau} &=& \frac{1}{2}\sigma^2C^2+(i\rho\sigma u_1-\kappa)C-\frac{1}{2}u_1(u_1+i).
      \label{C}
    \end{eqnarray}
    The Riccati equation for $C(\tau,u_1,u_2)$ has the following solution
    \begin{equation}
      C(\tau,u_1,u_2)=-\frac{w^{\prime}(\tau)}{w(\tau)}\frac{1}{R(\tau)},
    \end{equation}
    where $w(\tau)$ satisfies the following equation,
    \begin{equation}
      w^{\prime\prime}-\left[\frac{R^{\prime}}{R}+Q\right]w^{\prime}+PRw=0,
    \end{equation}
    and
    \begin{equation}
      P(\tau) = -\frac{1}{2}u_1(u_1+i), \quad\quad Q(\tau)=i\rho\sigma u_1-\kappa,
      \quad\quad R(\tau)=\frac{1}{2}\sigma^2.
    \end{equation}
    Define
    \begin{equation}
      \beta = \kappa-i\rho\sigma u_1, \quad\quad  d=\sqrt{\beta^2+\sigma^2u_1(u_1+i)},
    \end{equation}
    it can be shown that the solution to Eq. (\ref{C}) is
    \begin{equation}
      C(\tau,u_1,u_2)=\frac{\beta+d}{\sigma^2}\frac{e^{d\tau}-1}{Ke^{d\tau}-1},
      \label{C1}
    \end{equation}
    where
    \begin{equation}
      K=\frac{\beta+d-iu_2\sigma^2}{\beta-d-iu_2\sigma^2}.
    \end{equation}
    Then, Eq. (\ref{A}) can be directly integrated,
    \begin{equation}
      A(\tau,u_1,u_2)=iu_1(r-q)\tau+\frac{\kappa\theta}{\sigma^2}\left[(\beta+d)\tau
                      -2\log\left(\frac{Ke^{d\tau}-1}{K-1}\right)\right].
      \label{A1}
    \end{equation}

    \begin{remark}
      When the characteristic function for the log stock price is needed, we can simply set $u_2=0$ in the 
      above bivariate characteristic function.
    \end{remark}

    \begin{remark}
      For numerical integration, the {\it Little Heston Trap} trick is generally applied to the argument of the 
      complex logarithm, to remove the discountinuities. Define
      \begin{equation}
        G=\frac{1}{K}=\frac{\beta-d-iu_2\sigma^2}{\beta+d-iu_2\sigma^2},
      \end{equation}
      we can rewrite Eqs. (\ref{C1}) and (\ref{A1}) equivalently as
      \begin{eqnarray}
        A(\tau,u_1,u_2) &=& iu_1(r-q)\tau+\frac{\kappa\theta}{\sigma^2}\left[(\beta-d)\tau
                      -2\log\left(\frac{1-Ge^{-d\tau}}{1-G}\right)\right], \\
        C(\tau,u_1,u_2) &=& \frac{\beta+d}{\sigma^2K}\frac{1-e^{-d\tau}}{1-Ge^{-d\tau}}.
      \end{eqnarray}
    \end{remark}

  \subsection{Cox-Ingersoll-Ross Process}


  % \subsection{3/2 Model}

  \subsection{Orenstein-Uhlenbeck Process}


\section{Fourier Inversion Formula}

  Consider the problem of finding the cumulative density function from characteristic function (\ref{char}). 
  In particular, we want to find the probability of $X>k$, {\it i.e.}, $P(X>k)$, which is given by
  \begin{equation}
    P(X > k)=\int_k^{+\infty}f(x)dx.
  \end{equation}
  Here, $f(x)$ is the probability density function of $X$, and can be recovered from the characteristic function (\ref{char})
  by the inverse Fourier transform,
  \begin{equation}
    f(x)=\frac{1}{2\pi}\int_{-\infty}^{+\infty}e^{-iux}\varphi(u)du.
  \end{equation}
  Therefore,
  \begin{eqnarray}
    P(X > k)&=&\frac{1}{2\pi}\int_k^{+\infty} \left(\int_{-\infty}^{+\infty}e^{-iux}\varphi(u)du\right)dx
                        =\frac{1}{2\pi}\int_{-\infty}^{+\infty}\varphi(u) \left(\int_k^{+\infty}e^{-iux}dx\right)du\nonumber\\
                   &=&\frac{1}{2\pi}\int_{-\infty}^{+\infty}\varphi(u)\frac{e^{-iuk}}{iu}du
                        - \frac{1}{2\pi}\lim_{R\rightarrow +\infty}\int_{-\infty}^{+\infty}\varphi(u)\frac{e^{-iuR}}{iu}du.
  \end{eqnarray}
  To evaluate the second term, we again use the definition of the characteristic function,
  \begin{eqnarray}
    \frac{1}{2\pi}\lim_{R\rightarrow +\infty}\int_{-\infty}^{+\infty}\varphi(u)\frac{e^{-iuR}}{iu}du
    &=& \frac{1}{2\pi}\lim_{R\rightarrow +\infty}\int_{-\infty}^{+\infty}\left(\int_{-\infty}^{+\infty}e^{iux}f(x)dx\right)\frac{e^{-iuR}}{iu}du\nonumber\\
    &=& \frac{1}{2\pi}\lim_{R\rightarrow +\infty}\int_{-\infty}^{+\infty}f(x)\left(\int_{-\infty}^{+\infty}\frac{e^{iu(x-R)}}{iu}du\right)dx\nonumber\\
    &=& \frac{1}{2\pi}\lim_{R\rightarrow +\infty}\int_{-\infty}^{+\infty}f(x)\cdot\pi {\rm sgn}(x-R)dx\nonumber\\
    &=& \frac{1}{2}\lim_{R\rightarrow +\infty}\left(1-2F(R)\right) = -\frac{1}{2}.
  \end{eqnarray}
  Here, we have used the fact that
  \begin{equation}
    \int_{-\infty}^{+\infty}{\rm sgn}(x-y)f(x)dx=\int_y^{+\infty}f(x)dx - \int_{-\infty}^yf(x)dx = 1-2F(x),
  \end{equation}
  where $F(x)$ is the cumulative distribution function of $X_T$.

  Finally, the probability is given by
  \begin{equation}
    \label{inversion}
    P(X > k)=\frac{1}{2}+\frac{1}{2\pi}\int_{-\infty}^{+\infty}\varphi(u)\frac{e^{-iuk}}{iu}du.
  \end{equation}


\section{Vanilla option pricing with characteristic functions}

  In the following, we are going to explore several formulations of vanilla option prices in terms of
  characteristic functions.

  \subsection{Heston \cite{Heston}}

    Consider the call option price
    \begin{equation}
      C(K)=e^{-r(T-t)}{\rm E}^{\mathbb{Q}}\left[\left(S_T-K\right)^+\right],
    \end{equation}
    under the risk neutral measure.
    We are seeking a representation similar to the Black-Scholes formula. To this end, we can write the
    call option prices as
    \begin{equation}
      C(K)=e^{-r(T-t)}{\rm E}^{\mathbb{Q}}\left[\left(S_T-K\right){\mathcal I}_{S_T>K}\right],
    \end{equation}
    where ${\mathcal I}$ is the indicator function. Now, we have
    \begin{equation}
      C(K)=e^{-r(T-t)}{\rm E}^{\mathbb{Q}}\left[S_T{\mathcal I}_{S_T>K}\right]-Ke^{-r(T-t)}{\rm E}^{\mathbb{Q}}\left[{\mathcal I}_{S_T>K}\right]
          =S_tP_1 - Ke^{-r(T-t)}P_2,
    \end{equation}
    where
    \begin{equation}
      P_1 = {\rm E}^{\mathbb{Q}}\left[\frac{S_T/S_t}{B_T/B_t}{\mathcal I}_{S_T>K}\right],\quad\quad
      P_2 = {\rm E}^{\mathbb{Q}}\left[{\mathcal I}_{S_T>K}\right].
    \end{equation}
    It is obvious that $P_2$ is the risk neutral probability for the underlying asset maturing in-the-money. Also, $P_1$ can
    be represented as the ITM probability in another measure. Notice that
    \begin{equation}
      \frac{d{\mathbb{Q}}}{d{\mathbb{Q}}^S}=\frac{B_T/B_t}{S_T/S_t}=\frac{{\rm E}^{\mathbb{Q}}[e^{X_T}]}{e^{X_T}},
    \end{equation}
    where we have used the fact that
    \begin{equation}
      {\rm E}^{\mathbb{Q}}[S_T]={\rm E}^{\mathbb{Q}}[e^{X_T}]=S_t\frac{B_T}{B_t},
    \end{equation}
    then
    \begin{equation}
      P_1 = {\rm E}^{{\mathbb{Q}^S}}\left[\frac{S_T/S_t}{B_T/B_t}{\mathcal I}_{S_T>K}\frac{dQ}{dQ^S}\right]
          = {\rm E}^{{\mathbb{Q}^S}}\left[{\mathcal I}_{S_T>K}\right].
    \end{equation}
    From Eq. (\ref{inversion}), the two probabilities $P_1$ and $P_2$ can be written as
    \begin{equation}
      P_j=\frac{1}{2}+\frac{1}{2\pi}\int_{-\infty}^{+\infty}\varphi_j(u)\frac{e^{-iuk}}{iu}du, \quad\quad j=1,2,
    \end{equation}
    with $k=\log K$.
    It seems that two characteristic functions corresponding to the two different measures are required in the valuation of the option
    price. However, these two characteristic functions are related due to the measure change. To see this, notice that
    \begin{equation}
      \varphi_1(u)=\int_{-\infty}^{+\infty}e^{iux}f^S(x)dx,
    \end{equation}
    where $f^S(x)$ is the probability density function of $X_T$ under the $\mathbb{Q}^S$-measure. It is related to $f(x)$, the probability density
    function of $X_T$ under the $\mathbb{Q}$-measure, through the measure change,
    \begin{equation}
      f^S(x)=f(x)\frac{d{\mathbb{Q}}^S}{d{\mathbb{Q}}}=f(x)\frac{e^{X_T}}{{\rm E}^{\mathbb{Q}}[e^{X_T}]}.
    \end{equation}
    Now,
    \begin{equation}
      \varphi_1(u)=\frac{1}{{\rm E}^{\mathbb{Q}}[e^{X_T}]}\int_{-\infty}^{+\infty}e^{ix(u-i)}f(x)dx=\frac{\varphi(u-i)}{\varphi(-i)},
    \end{equation}
    since
    \begin{equation}
      {\rm E}^{\mathbb{Q}}[e^{X_T}]=\int_{-\infty}^{+\infty}e^{x}f(x)dx=\varphi(-i),
    \end{equation}
    where $\varphi(u)$ is the characteristic function as defined in (\ref{char}), and coincides with $\varphi_2(u)$. Therefore,
    \begin{eqnarray}
      P_1 &=& \frac{1}{2}+\frac{1}{2\pi}\int_{-\infty}^{+\infty}\frac{e^{-iu\log K}\varphi(u-i)}{iu\varphi(-i)}du, \\
      P_2 &=& \frac{1}{2}+\frac{1}{2\pi}\int_{-\infty}^{+\infty}\frac{e^{-iu\log K}\varphi(u)}{iu}du. \\
    \end{eqnarray}

  \subsection{Carr and Madan \cite{CarrMadan}}

    The call option price can be represented as
    \begin{equation}
      C(k)=e^{-r(T-t)}\int_k^{+\infty}(e^x-e^k)f(x)dx,
    \end{equation}
    where $k=\log K$, $x=\log S_T$, and $f(x)$ is the probability density function for the distribution of $x$ at maturity. We want to find the
    Fourier transform of the above call option price, but it is not integrable. To remedy this, we introduce a damping factor and modify the
    call option price accodingly,
    \begin{equation}
      c(k)=e^{\alpha k}C(k),
    \end{equation}
    and the corresponding Fourier transform is given by
    \begin{equation}
      \hat{c}(v)=\int_{-\infty}^{+\infty}e^{ivk}c(k)dk.
    \end{equation}
    Rearrange the integration order, we have
    \begin{eqnarray}
      \hat{c}(v) &=& e^{-r(T-t)}\int_{-\infty}^{+\infty}e^{ivk}\left(\int_k^{+\infty}e^{\alpha k}(e^x-e^k)f(x)dx\right)dk \nonumber\\
                 &=& e^{-r(T-t)}\int_{-\infty}^{+\infty}f(x)\left(\int_{-\infty}^x\left(e^{(\alpha+iv)k}e^x-e^{(\alpha+iv+1)k}\right)dk\right)dx \nonumber\\
                 &=& e^{-r(T-t)}\int_{-\infty}^{+\infty}f(x)\left.\left[\frac{e^{(\alpha+iv)k}e^x}{\alpha+iv}
                                                        -\frac{e^{(\alpha+iv+1)k}}{\alpha+iv+1}\right]\right|_{-\infty}^{x}dx \nonumber\\
                 &=& e^{-r(T-t)}\int_{-\infty}^{+\infty}f(x)\left[\frac{e^{(\alpha+iv+1)x}}{\alpha+iv} -\frac{e^{(\alpha+iv+1)x}}{\alpha+iv+1}\right]dx \nonumber\\
                 &=& e^{-r(T-t)}\int_{-\infty}^{+\infty}f(x)\frac{e^{(\alpha+iv+1)x}}{(\alpha+iv)(\alpha+iv+1)}dx \nonumber\\
                 &=& \frac{e^{-r(T-t)}\varphi(v-(1+\alpha)i)}{(\alpha+iv)(\alpha+iv+1)},
    \end{eqnarray}
    where $\alpha > 0$ is necessary to ensure convergence at $k=-\infty$. Once the Fourier transform of the modified call option price is known,
    we can use the inverse Fourier transform to recover the call option price,
    \begin{equation}
      C(k)=\frac{e^{-\alpha k}}{2\pi}\int_{-\infty}^{+\infty}e^{-ivk}\hat{c}(v)dv.
    \end{equation}

  \subsection{Lewis \cite{Lewis}}

    Denote the payoff the contingent claim at maturity as $g(x)$, then the value of the derivative is given by
    \begin{equation}
      V = \int_{-\infty}^{+\infty}f(x)g(x)dx.
    \end{equation}


\begin{thebibliography}{99}
  \bibitem{Heston}
    S. Heston, {\it A Closed-Form Solution for Options with Stochastic Volatility with Applications to Bond and Currency Options}, Review of Financial Studies {\bf 6}, 327 (1993).

  \bibitem{CarrMadan}
    P. Carr and D. Madan, {\it Option valuation using the fast Fourier transform}, The Journal of Computational Finance {\bf 2}, 61 (1999).

  \bibitem{Lewis}
    A. Lewis, {\it A Simple Option Formula for General Jump-Diffusion and Other Exponential Levy Processes}, \url{https://ssrn.com/abstract=282110}.

\end{thebibliography}


\end{document}